\documentclass{article}
\usepackage[landscape, margin=1cm]{geometry}
\usepackage{array}
\usepackage{tabularx}
\usepackage{makecell}
\usepackage{apacite}
\usepackage{longtable}
\usepackage{tabu}

% \newcolumntype{L}{>{\arraybackslash}m{3cm}}
% \renewcommand{\cellalign/theadalign}{vh}

\title{Ontogenetic Growth Model Literature}
\date{}
\begin{document}
\maketitle

% \begin{table}[] % not inside table environment so it doesn't float
\begin{longtabu} to \textwidth {
        |X[3,l]
        |X[2,l]
        |X[2,l]
        |X[1.7,l]
        |X[10,l]|} %{\textwidth}{|l|l|l|l|X|X|}%{|l|l|l|l|L|l|}
    \hline
    \textbf{Study}                          & \textbf{Scaling} & \textbf{P or A}    & \textbf{LHT or MT}    & \textbf{Model(s)} \\ \hline
    \cite{Gadgil1970}                       & Hyperallometric  & Prediction         & LHT                   & \makecell[l]{Probability of survival from birth to age $x$, $l_x = \prod\limits_0^{x-1} \alpha_i \cdot f_{1}(\theta_i) \cdot g_{1}(\psi_i) \cdot \eta_i$ \\ Size at age $x$ is $w_x = w_0 + \sum\limits_0^{x-1}\delta_i \cdot f_{2}(\theta_i) \cdot g_{2}(\psi_i)$ \\$\alpha_i = $probabiliy of survival from age $i$ to $i+1$ for individual making no \\ repro effort at age $i$ \\ $\delta_i = $incrememnet in size from age $i$ to $i+1$ no repro effort \\ $w_i = $ size individual age $i$ \\ $\theta_i = $repro effort of individual at age $i$ $0 < \theta_i < 1$ \\ $f_1$ is monotonically decreasing function [0,1]} \\ \hline
    \cite{Roff1983}                         & Isometric        & Assumption         & LHT                   & \makecell[l]{Assumes fixed amount surplus energy which may be channeled into either \\ somatic or gonadal tissue. Uses GSI. "Energy channeled into the gonads \\ detracts from somatic growth and hence future fecundity." \\ $W_{t+1} = W_t + \Delta{S_t} - G_{t+1}$ (1) \qquad $W_{t+1} =$ somatic weight, $\Delta{S_t} =$ max \\ potential increase in somatic weight/energy surplus. $G =$ is weight of the \\gonads. \\ In general, $G_{t+1} = \alpha{W_{t+1}^\beta}$ (2)\\ "Fecundity varies as power function of length, exp $\approx 3$" \\ Data show $1 < \beta < 1.9$ \\ Sub (1) $\rightarrow$ (2) \\ $W_{t+1} = W_t + \Delta{S_t} - \alpha{W_{t+1}^\beta}$ \\ $W_{t+1} + \alpha{W_{t+1}^\beta} = W_t + \Delta{S_t}$ \\ $W_{t+1}(1+\alpha{W_{t+1}^{\beta{-1}}}) = W_t + \Delta{S_t}$ \\ $W_{t+1} = \frac{W_t + \Delta{S_t}}{1 + \alpha{W_{t+1}^{\beta{-1}}}}$} \\ \hline
    Roff (1984)                             & Isometric        & Assumption         & LHT                   & dm                                                                                                                                                    \\ \hline
    Reiss (1985)                            & Hypoallometric   & Prediction         & LHT                   & dm                                                                                                                                                    \\ \hline
    \cite{Kozowski1987-determinate}         & Variable         & Prediction         & LHT                   & \makecell[l]{Maximising no. offspring in lifetime equivalent to maximising volume \\ $V = HS$ \\ where $S$ is area of base (determined by $\alpha$, max lifespan and $L(x)$) and \\ $H$ is height (determined by $m_\alpha$) \\ Rate of volume change with $\alpha$ is: \\ $\frac{dV}{dt} = \frac{dH}{dt}S + H\frac{dS}{dt}$} \\ \hline
    \cite{Kozowski1987-indeterminate}       & Variable         & Prediction         & LHT                   & \makecell[l]{To maxmize lifetime allocation of resources to reproduction in seasonal \\ environments it is optimal to have indeterminate growth and to allocate \\ increasing proportions of surplus energy to repro year upon year.} \\ \hline
    \cite{kozlowski1996}                    & Variable         & Prediction         & LHT                   & \makecell[l]{}                                                                                                                                                    \\ \hline
    \cite{West2001}                         & Isometric        & Assumption         & MT                    & \makecell[l]{$\frac{dm}{dt} = am^{3/4} - bm$ \\ No growth at size M when $\frac{dm}{dt} = 0 $ \\ $am^{3/4} - bm = 0 $ \\ $am^{3/4} = bm$ \\ $b = \frac{a}{M^{1/4}}$ and substitute \\ $\frac{dm}{dt} = am^{3/4} - \Big(\frac{a}{M^{1/4}}\Big)m = am^{3/4}\Big[1 - (\frac{m}{M})^{1/4}\Big]$ \\ $\int\frac{dm}{dt} = \int am^{3/4}\Big[1 - (\frac{m}{M})^{1/4}\Big]$ using $u = 1 - \big(\frac{m}{M}\big)^{1/4}$ substitution\\ $\big(\frac{m}{M}\big)^{1/4} = 1 - \bigg[1 - \left(\frac{m_0}{M}\right)^{1/4}\bigg]e^{-at/4M^{1/4}}$ \\ Plotted dimensionless mass ratio, $r = 1 - R \equiv (m/M)^{1/4}$ against dimensionless \\ time variable, $\tau = (at/4M^{1/4}) - ln[1-(m_{0}/M)^{1/4}]$ \\ "during a spawning period, the mass of a clutch is a constant fraction, $\lambda$ of body \\ mass: $m_k \approx \lambda{m}$} \\ \hline
    \cite{Charnov2001}                      & Isometric        & Assumption         & LHT                   & \makecell[l]{$\frac{dm}{dt} = am^{3/4} - bm$ when $m < m_\alpha$ \\ $\frac{dm}{dt} = am^{3/4} - bm - cm$ when $m > m_\alpha$ \\ Terminal growth asymptote translates down since $M = \Big(\frac{a}{b+c}\Big)^4$}                                                                                                                                                    \\ \hline
    \cite{Charnov2004}                      & Isometric        & Assumption         & LHT                   & \makecell[l]{$\frac{dm}{dt} = am^{3/4} - bm$ when $m < m_\alpha$ \\ $\frac{dm}{dt} = am^{3/4} - bm - cm$ when $m > m_\alpha$ \\ Terminal growth asymptote translates down since $M = \Big(\frac{a}{b+c}\Big)^4$}                                                                                                                                                    \\ \hline
    \cite{Lester2004}                       & Isometric*       & Assumption         & LHT                   & \makecell[l]{No theoretical support for \cite{vonBert1938} because does not "cleanly \\ account for change in energy allocation [...] at maturity". Build on \\ \cite{Charnov2001} i.e. $\alpha$ and $c$ adjust given a mortality rate. Use VB equation \\ as exact description of post-maturation somatic growth. Evaluate trade-offs \\ between somatic and repro determine LH traits. Potential somatric growth \\ $\frac{dW}{dt} = c_{1}W_t^{m_2} - c_{1}W_t^{m_2} \rightarrow$ same as $\frac{dm}{dt} = am^{3/4} - bm$ \\ but mean $\bar{m}_1 = 0.69$ and $\bar{m}_2 = 0.75$ \\ Assume that the equation holds true \\ $\frac{dW}{dt} = (c_1 - c_2)W_t^{2/3}$ \\ ``error negligible provided $m_1$ and $m_2$ have similar values in range 0.67-0.70" \\ cite Piazza \textit{et al.} 2002 regarding ontogenetic diet shifts to larger prey as size \\increases. \\ Post-maturation: \\ $L_t = L_{\infty}(1 - e^{-k(t-t_0)})$}\\ \hline
    Roff \textit{et al}. (2006)             & Isometric        & Assumption         & LHT                   & dm                                                                                                                                                    \\ \hline
    \cite{Quince2008a}                      & Isometric*       & Assumption         & LHT                   & dm                                                                                                                                                    \\ \hline
    \cite{Quince2008b}                      & Hyperallometric* & Prediction         & LHT                   & dm                                                                                                                                                    \\ \hline
    Pecquerie \textit{et al}. (2009)        & Isometric        & Assumption         & LHT                   & dm                                                                                                                                                    \\ \hline
    \cite{kooijman2010dynamic}              & Isometric*       & Assumption         & LHT                   & dm                                                                                                                                                    \\ \hline
    Arendt (2004)                           & Isometric        & Assumption         & LHT                   & dm                                                                                                                                                    \\ \hline
    \cite{Ohnishi2010}                      & Hyperallometric  & Assumption         & LHT                   & dm                                                                                                                                                    \\ \hline
    \cite{Brunel2013}                       & Isometric        & Assumption         & LHT                   & dm                                                                                                                                                    \\ \hline
    \cite{Charnov2013}                      & Isometric        & Assumption         & LHT                   & dm                                                                                                                                                    \\ \hline
    \cite{Boukal2014}                       & Isometric*       & Assumption         & LHT                   & dm                                                                                                                                                    \\ \hline
    \cite{Kooijman2014a}                    & Isometric*       & Assumption         & LHT                   & dm                                                                                                                                                    \\ \hline
    \cite{Minte-Vera2016}                   & Isometric*       & Assumption         & LHT                   & dm                                                                                                                                                    \\ \hline
    \cite{Jusup2017}                        & Isometric        & Assumption         & LHT                   & dm                                                                                                                                                    \\ \hline
    \cite{Mangel2017}                       & Hyperallometric  & Assumption         & LHT                   & dm                                                                                                                                                    \\ \hline
    Smallegauge \textit{et al}. (2017)      & Isometric        & Assumption         & LHT                   & dm                                                                                                                                                    \\ \hline
    \cite{Manabe2018}                       & Isometric        & Assumption         & LHT                   & dm                                                                                                                                                    \\ \hline
\end{longtabu}
% \end{table}


\bibliographystyle{apacite}
\bibliography{/home/lvassor/Documents/bibtex_sync/CMEE_Thesis.bib}
\end{document}\grid
