\documentclass{article}
\usepackage[landscape, margin=1cm]{geometry}
\usepackage{array}
\usepackage{tabularx}
\usepackage{makecell}
\usepackage{apacite}
\usepackage{longtable}
\usepackage{tabu}
\usepackage{amssymb}

% \newcolumntype{L}{>{\arraybackslash}m{3cm}}
% \renewcommand{\cellalign/theadalign}{vh}

\title{Ontogenetic Growth Model Literature}
\date{}
\begin{document}
\maketitle

% \begin{table}[] % not inside table environment so it doesn't float
\begin{longtabu} to \textwidth {
        |X[3,l]
        |X[2,l]
        |X[2,l]
        |X[1.3,l]
        |X[10,l]|} %{\textwidth}{|l|l|l|l|X|X|}%{|l|l|l|l|L|l|}
    \hline
    \textbf{Study}                              & \textbf{Scaling} & \textbf{P or A}    & \textbf{LHT/MT}    & \textbf{Model(s)} \\ \hline
    \cite{Gadgil1970}                           & Hyperallometric  & Prediction         & LHT                   & \makecell[l]{Probability of survival from birth to age $x$, $l_x = \prod\limits_0^{x-1} \alpha_i \cdot f_{1}(\theta_i) \cdot g_{1}(\psi_i) \cdot \eta_i$ \\ Size at age $x$ is $w_x = w_0 + \sum\limits_0^{x-1}\delta_i \cdot f_{2}(\theta_i) \cdot g_{2}(\psi_i)$ \\$\alpha_i = $probabiliy of survival from age $i$ to $i+1$ for individual making no \\ repro effort at age $i$ \\ $\delta_i = $incrememnet in size from age $i$ to $i+1$ no repro effort \\ $w_i = $ size individual age $i$ \\ $\theta_i = $repro effort of individual at age $i$ $0 < \theta_i < 1$ \\ $f_1$ is monotonically decreasing function [0,1]} \\ \hline
    \cite{Roff1983}                             & Isometric        & Assumption         & LHT                   & \makecell[l]{Assumes fixed amount surplus energy which may be channeled into either \\ somatic or gonadal tissue. Uses GSI. "Energy channeled into the gonads \\ detracts from somatic growth and hence future fecundity." \\ $W_{t+1} = W_t + \Delta{S_t} - G_{t+1}$ (1) \qquad $W_{t+1} =$ somatic weight, $\Delta{S_t} =$ max \\ potential increase in somatic weight/energy surplus. $G =$ is weight of the \\gonads. \\ In general, $G_{t+1} = \alpha{W_{t+1}^\beta}$ (2)\\ "Fecundity varies as power function of length, exp $\approx 3$" \\ Data show $1 < \beta < 1.9$ \\ Sub (1) $\rightarrow$ (2) \\ $W_{t+1} = W_t + \Delta{S_t} - \alpha{W_{t+1}^\beta}$ \\ $W_{t+1} + \alpha{W_{t+1}^\beta} = W_t + \Delta{S_t}$ \\ $W_{t+1}(1+\alpha{W_{t+1}^{\beta{-1}}}) = W_t + \Delta{S_t}$ \\ $W_{t+1} = \frac{W_t + \Delta{S_t}}{1 + \alpha{W_{t+1}^{\beta{-1}}}}$} \\ \hline
    Roff (1984)                                 & Isometric        & Assumption         & LHT                   & dm                                                                                                                                                    \\ \hline
    Reiss (1985)                                & Hypoallometric   & Prediction         & LHT                   & dm                                                                                                                                                    \\ \hline
    \cite{Kozlowski1987-determinate}            & Variable         & Prediction         & LHT                   & \makecell[l]{``Organism is limited by energy, either absolute amount available or by limits on \\ acquisition/processing time" \\ ``Energy allocated to current repro must lower growth rate" \\ ``small body size [...] cause decreased future repro rate and low survivability" \\ ``Dynamic optimisation technique in literature shows that to maximise repro \\ output, somatic and reproductive growth should be seperated in time" \\ ``Separation observable if time unit short enough" (link to MT which considers \\ lifespan) \\ Optimisation of $\alpha$ and M but assumes immediate switch from growth to repro to \\ allow calculus (versus dynamic optimisation) \\ Permit variation of mortality with time. \\ Assume fixed population size \\ \textbf{Optimal $\alpha$} \\ Assume ATR is continuous - approximated by v small clutches \\ Assume growth pause and surplus energy ATR \\ DISADV: Delayed maturity lowers chance of surviving to reproduce and time \\available (seasonal environment) \\ ADV: ``fecundity gain hypothesis" - larger mothers = increase ATR (constant \\ fraction) after $\alpha$ but delay $\alpha$ in order to be bigger \\ Maximising no. offspring in lifetime equivalent to maximising volume \\ $V = HS = \alpha \cdot l(x) \cdot x$ \\ axes: age ($x$) x P(surviving to age) $l(x)$ x rate reproduction $m(x)$ [function of age \\ and $\therefore$ size at $\alpha$] \\ With no mortality then could just integrate $m(x)$ up to $x$ to find $R_0$. $Z$ introduces \\ $l(x)$, a third dimension. Height of volume is constant beyond $\alpha$. \\ where $S$ is area of base (determined by $\alpha$, max lifespan and $L(x)$) and \\ $H$ is height (determined by $m_\alpha$) \\ Rate of volume change ($R_0$ with $\alpha$ (varying maturation age) is: \\ $\frac{dV}{d\alpha} = \frac{dH}{d\alpha}S + H\frac{dS}{d\alpha}$ ($u'v + v'u$) \\ Assumption that age influences fecundity because is a correlate of size so: \\ $\frac{dH(w)}{d\alpha} = \frac{dH(w)}{dw}f(w)$ where $f(w) = \frac{dw}{d\alpha} =$ growth rate of individual. \\ Because age and size correlate, mortality rate can be considered size-dependent so \\ S is function of $m$ and $\alpha$: \\ $\frac{dS(m,\alpha)}{dt} = \frac{\partial{S(m,\alpha)}}{\partial{m}}f(w) + \frac{\partial{S(m,\alpha)}}{\partial{\alpha}}$} \\ \hline
    \cite{Kozowski1987-indeterminate}           & Variable         & Prediction         & LHT                   & \makecell[l]{To maxmize lifetime allocation of resources to reproduction in seasonal \\ environments it is optimal to have indeterminate growth and to allocate \\ increasing proportions of surplus energy to repro year upon year.} \\ \hline
    \cite{kozlowski1996}                        & Variable         & Prediction         & LHT                   & \makecell[l]{}                                                                                                                                                    \\ \hline
    Kozlowski in: \cite{brown2000-scaling-book} & Variable         & Prediction         & LHT                   & \makecell[l]{}                                                                                                                                                    \\ \hline
    \cite{West2001}                             & Isometric        & Assumption         & MT                    & \makecell[l]{$\frac{dm}{dt} = am^{3/4} - bm$ \\ No growth at size M when $\frac{dm}{dt} = 0 $ \\ $am^{3/4} - bm = 0 $ \\ $am^{3/4} = bm$ \\ $b = \frac{a}{M^{1/4}}$ and substitute \\ $\frac{dm}{dt} = am^{3/4} - \Big(\frac{a}{M^{1/4}}\Big)m = am^{3/4}\Big[1 - (\frac{m}{M})^{1/4}\Big]$ \\ $\int\frac{dm}{dt} = \int am^{3/4}\Big[1 - (\frac{m}{M})^{1/4}\Big]$ using $u = 1 - \big(\frac{m}{M}\big)^{1/4}$ substitution\\ $\big(\frac{m}{M}\big)^{1/4} = 1 - \bigg[1 - \left(\frac{m_0}{M}\right)^{1/4}\bigg]e^{-at/4M^{1/4}}$ \\ Plotted dimensionless mass ratio, $r = 1 - R \equiv (m/M)^{1/4}$ against dimensionless \\ time variable, $\tau = (at/4M^{1/4}) - ln[1-(m_{0}/M)^{1/4}]$ \\ "during a spawning period, the mass of a clutch is a constant fraction, $\lambda$ of body \\ mass: $m_k \approx \lambda{m}$} \\ \hline
    \cite{Charnov2001}                          & Isometric        & Assumption         & LHT                   & \makecell[l]{``A prominent feature of comparative life histories in fish is the approximate \\ \textbf{invariance} across species of certain dimensionless numbers made up from \\ reproductive and timing variables, $\frac{\alpha}{E}$ and $c \cdot E$ (age at maturity/average adult \\ lifespan and proportion body mass ATR/year x E)'' \\ Acknowledges that various literature shows disagreements in how strong the \\ correlation between $E$ and $\alpha$ has to be to claim invariance but quantifies $r \geq 0.8$ \\ Combines \cite{West2001} with ``general yet specific constrains on ATR'' \\ ``Optimal life histories accurately predict the numeric value of dimensionless \\ numbers that combine $\alpha$ or $m_\alpha$, $Z$ and $c$'' \\ $\frac{dm}{dt} = am^{3/4} - bm$ when $m < m_\alpha$ \\ $\frac{dm}{dt} = am^{3/4} - bm - cm$ when $m > m_\alpha$ \\ Terminal growth asymptote translates down since $M = \Big(\frac{a}{b+c}\Big)^4$ \\ New growth model predicts fastest growth rate under indeterminate schedule will be \\ near size of first repro. \\ Lifetime ATR is time integreal $\int c\cdot m$ \\ ``It seems reasonable that $\alpha$ and $c$ are [...] most easily adjusted by natural selection \\ across species'' \\ ``$a$ and $c$ are chosen to maximise a quantity proportional to limetime production of \\ offspring in the face of a mortality rate $Z$'' \\ Contradict Kozlowski - ``$Z$ is probably not body-size (or age)-dependent over \\ ontogeny within a species'' \\ Discuss that many formal evo models predict that growth should cease with \\ onset of repro ($m_\alpha \rightarrow m_\infty$), but this isn't found outside birds, mammals, insects so \\ must be ignoring something crucial \\ Why have a high $b$? Why build body with high-cost cells? \\ Hypothesise that high $b$ allows greater $c$ \\ Morphology may constrain \textit{how many} cells you can devote to $c$, but $b$ may allow \\ greater per unit production, $\therefore c = bq$ where q includes constraints. \\ Suggestion that q itself is fixed, therefore evo adjustment of c means adjusting b  \\ \textbf{Reproductive Allometry} \\ None of the dimensionless numbers rely on a} \\ \hline
    \cite{Charnov2004}                          & Isometric        & Assumption         & LHT                   & \makecell[l]{$\frac{dm}{dt} = am^{3/4} - bm$ when $m < m_\alpha$ \\ $\frac{dm}{dt} = am^{3/4} - bm - cm$ when $m > m_\alpha$ \\ Terminal growth asymptote translates down since $M = \Big(\frac{a}{b+c}\Big)^4$}                                                                                                                                                    \\ \hline
    \cite{Lester2004}                           & Isometric*       & Assumption         & LHT                   & \makecell[l]{No theoretical support for \cite{vonBert1938} because does not "cleanly \\ account for change in energy allocation [...] at maturity". Build on \\ \cite{Charnov2001} i.e. $\alpha$ and $c$ adjust given a mortality rate. Use VB equation \\ as exact description of post-maturation somatic growth. Evaluate trade-offs \\ between somatic and repro determine LH traits. Potential somatric growth \\ $\frac{dW}{dt} = c_{1}W_t^{m_2} - c_{1}W_t^{m_2} \rightarrow$ same as $\frac{dm}{dt} = am^{3/4} - bm$ \\ but mean $\bar{m}_1 = 0.69$ and $\bar{m}_2 = 0.75$ \\ Assume that the equation holds true \\ $\frac{dW}{dt} = (c_1 - c_2)W_t^{2/3}$\\ ``error negligible provided $m_1$ and $m_2$ have similar values in range 0.67-0.70" \\ cite Piazza \textit{et al.} 2002 regarding ontogenetic diet shifts to larger prey as size \\increases. \\ Post-maturation: \\ $L_t = L_{\infty}(1 - e^{-k(t-t_0)})$}\\ \hline
    Roff \textit{et al}. (2006)                 & Isometric        & Assumption         & LHT                   & dm                                                                                                                                                    \\ \hline
    \cite{Quince2008a}                          & Isometric*       & Assumption         & LHT                   & dm                                                                                                                                                    \\ \hline
    \cite{Quince2008b}                          & Hyperallometric* & Prediction         & LHT                   & dm                                                                                                                                                    \\ \hline
    Pecquerie \textit{et al}. (2009)            & Isometric        & Assumption         & LHT                   & dm                                                                                                                                                    \\ \hline
    \cite{kooijman2010dynamic}                  & Isometric*       & Assumption         & LHT                   & dm                                                                                                                                                    \\ \hline
    Arendt (2004)                               & Isometric        & Assumption         & LHT                   & dm                                                                                                                                                    \\ \hline
    \cite{Ohnishi2010}                          & Hyperallometric  & Assumption         & LHT                   & dm                                                                                                                                                    \\ \hline
    \cite{Brunel2013}                           & Isometric        & Assumption         & LHT                   & dm                                                                                                                                                    \\ \hline
    \cite{Charnov2013}                          & Isometric        & Assumption         & LHT                   & dm                                                                                                                                                    \\ \hline
    \cite{Boukal2014}                           & Isometric*       & Assumption         & LHT                   & dm                                                                                                                                                    \\ \hline
    \cite{Kooijman2014a}                        & Isometric*       & Assumption         & LHT                   & dm                                                                                                                                                    \\ \hline
    \cite{Minte-Vera2016}                       & Isometric*       & Assumption         & LHT                   & dm                                                                                                                                                    \\ \hline
    \cite{Jusup2017}                            & Isometric        & Assumption         & LHT                   & dm                                                                                                                                                    \\ \hline
    \cite{Mangel2017}                           & Hyperallometric  & Assumption         & LHT                   & \makecell[l]{Size-dependent mortality function leads to the prediction that $\alpha$ depends upon \\ asymptotic size. Discusses inverse life-history wrt mortality rate \\Beverton's GML theory: Growth, Maturation, Longevity \\ ``It is now generally agreed that mortality of fish is an explicit function of size \\ (and thus an implicity function of age)'' \\ Gislason et al. (2010) used stats study to conclude rate of mortality depends on size \\ AND asymptotic size $L_\infty$ and von Bert growth rate k \\ $M_{g}(L) = e^{0.55} \cdot k \cdot \frac{L_{\infty}^{1.44}}{L^{1.61}}$ \\ $M_{c}(L) = e^{-0.05} \cdot k \cdot \big(\frac{L_\infty}{L}\big)^1.46$ \quad as found by Charnov \textit{et al.} (2013) who then argues \\ that 1.46 and 0.05 are not statistically different from 1.5 and 0, respectively. Hence, \\ $M_c(L) = k \cdot \big(\frac{L_\infty}{L}\big)^{1.5}$} \\ \hline
    Smallegauge \textit{et al}. (2017)          & Isometric        & Assumption         & LHT                   & dm                                                                                                                                                    \\ \hline
    \cite{Manabe2018}                           & Isometric        & Assumption         & LHT                   & \makecell[l]{Generalised q-VGBF using q-exponentials defined in Tsallis statistics \\ Can track growth trajectories in different life history strategies \\ Previous functions are special case of Putter - assumes growth rates proportional to \\ surplus energy production rates given by difference between anabolism and catabolism \\ $\frac{dw}{dt} = \lambda{w}^m - \gamma{w}^n$ \\ ``Since specific metabolic rate described by decreasing allometric function of body \\ mass, assume rate of surplus energy production per body mass is expressed as follows: \\ $\frac{1}{w}\frac{dS}{dt} = kw^{-\frac{1}{r}}$  [assume powers are identical between anabolism and catabolism] \\ $w =$ body mass, $S =$ surplus energy (same unit as body mass),\\ $t =$ time or age, $r$ and $k$ are coefficients \\ $\therefore$ specific rate of somatic growth and repo are: \\ $\frac{1}{w}\frac{dw}{dt} = kw^-\frac{1}{r}g(w)$ \quad and \quad $\frac{1}{w}\frac{dF}{dt} = kw^{-\frac{1}{r}}[1 - g(w)]$ \\ $g(w)$ is some function describing the rate of surplus energy ATR \\ $F$ is the cumulative energy allocation to reproduction \\ $g(w) = \Bigg[1 - \Big(\frac{w}{w_{\infty}}\Big)^{\frac{1}{r}}\Bigg]^q$ \quad ($w \leq w_{\infty}, q > 0, r > 0$)} \\ \hline
\end{longtabu}
% \end{table}


\bibliographystyle{apacite}
\bibliography{/home/lvassor/Documents/bibtex_sync/CMEE_Thesis.bib}
\end{document}\grid
