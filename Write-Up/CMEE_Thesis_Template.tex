\documentclass[a4paper]{article} % twoside for paper submission but remove for electronic submission
\author{Dr Deraj Wilson-Aggarwall}

%%%% Packages %%%

% Page Formatting packages
\usepackage[margin=2cm]{geometry}
\usepackage{lipsum} % Generates dummy text
\usepackage{titlesec}
    \titleformat*{\section}{\LARGE\bfseries}
    \titleformat*{\subsection}{\Large\bfseries}
    \titleformat*{\subsubsection}{\large\bfseries}
    \titleformat*{\paragraph}{\large\bfseries}
    \titleformat*{\subparagraph}{\large\bfseries}
    \newcommand{\sectionbreak}{\newpage} % forces new page for each section
    \titlespacing*{\section}
{0pt}{20cm plus 1ex minus .2ex}{4.3ex plus .2ex}
\usepackage{etoolbox} % adds numbers to sections
\makeatletter
\patchcmd{\ttlh@hang}{\parindent\z@}{\parindent\z@\leavevmode}{}{}
\patchcmd{\ttlh@hang}{\noindent}{}{}{}
\makeatother
\usepackage{fancyhdr} % rule line and current section displayed at top of pages % \thispagestyle{empty} removes it on desired pages
    \pagestyle{fancy}
    \renewcommand{\sectionmark}[1]{\markboth{#1}{}} % set the \leftmark to section name
    \fancyhead[L]{\nouppercase{\leftmark}} %  sets the left head element ONLY EVER section name, otherwise mark is section/subsection/subsubsection
    \fancyhead[R]{} % sets the right head element to the page number if you use \thepage as argument
    \fancyfoot[C]{\thepage} % sets footer as page number

% Plotting packages
\usepackage{graphicx}
\begin{document}

\begin{titlepage}
    % Square brackets control size of vertical gap AFTER group
    %----------------------------------------------------------------------------------------
    %	LOGO SECTION
    %----------------------------------------------------------------------------------------
    
    \includegraphics[width=4cm]{./Images/logo.png}\\%[1cm] % Include a department/university logo - this will require the graphicx package
     
    %----------------------------------------------------------------------------------------
    
    \center % Center everything on the page
    
    
    %----------------------------------------------------------------------------------------
    %	TITLE SECTION
    %----------------------------------------------------------------------------------------
    \makeatletter
    \linespread{1.5} % controls line spacing of title
        {\huge{SOME SHIT ABOUT \\ MOSQUITO ABUNDANCE}\par} % leave \par here
    \vspace{2.5cm} % Title of your document

    %----------------------------------------------------------------------------------------
    %	DESCRIPTION SECTION
    %----------------------------------------------------------------------------------------
    % If you want consistent line spacing then need to typeset paragraphy as a single {} group and use \\ for line break
    % Strangely, you need to leave the \ character at the end of the group so that the spacing of the last line from the penultimate line in a group remains constant
    % textsc command sets small capitals (sc)
    \textsc{A thesis submitted in partial fulfilment \\ of the requirements for the degree of \\ Master of Science at Imperial College London \\ by \\ \ }\\[2.5cm]
    \textsc{\Large \@author}\\[2.5cm]
    \textsc{Submitted for the \\ M.Sc. Computational Methods in Ecology \& Evolution \\ Formatted in the journal style of the Potato Journal \\ \ }\\[2cm]

    %----------------------------------------------------------------------------------------
    %	HEADING SECTIONS
    %----------------------------------------------------------------------------------------
    \textsc{Department of Life Sciences \\ Imperial College London \\ \ }\\[1cm]
    \textsc{\today}\\[2cm] % Date, change the \today to a set date if you want to be precise
    
    \vfill % Fill the rest of the page with whitespace
    
\end{titlepage}

%%% Declaration %%%
\section*{Declaration of Originality}\thispagestyle{empty}
    Declaration: The first page inside the cover must provide a brief declaration of the contributions
    made by you and by others to your project. Key points to address are:
    \begin{itemize}
        \item Was the data provided to you or did you collect or assemble it?
        \item Were you responsible for data processing or cleaning, if required?
        \item Were any mathematical models developed by you or by your supervisor?
        \item What role, if any, did your supervisor play in developing the analyses presented?
    \end{itemize}
    I certify that this thesis, and the research to which it refers, are the product of my own work, conducted during the current year of the \emph{M.Sc. Computational Methods in Ecology \& Evolution} at Imperial College London. Any ideas or quotations from the work of other people, published or otherwise, or from my own previous work are fully acknowledged in accordance with the standard referencing practices of the discipline and this institution.
    \vspace{3cm}
    \begin{flushright}
        \makeatletter
        \@author \\
        \today
    \end{flushright}

%%% Abstract %%%
\section*{Abstract}\thispagestyle{empty}
    Your abstract goes here. The abstract is a very brief summary of the dissertation's contents. It should be about half a page long. Somebody unfamiliar with your project should have a good idea of what it's about having read the abstract alone and will know whether it will be of interest to them.

%%% Acknowledgements %%%
\section*{Acknowledgements}\thispagestyle{empty}
    It is usual to thank those individuals who have provided particularly useful assistance, technical or otherwise, during your project.

%%% Table of Contents %%%
\newpage\tableofcontents\thispagestyle{empty}

%%% List of Figures %%%
\newpage\listoffigures\thispagestyle{empty}
\addcontentsline{toc}{section}{List of Figures}

%%% Listof Tables %%%
\newpage\listoftables\thispagestyle{empty}
\addcontentsline{toc}{section}{List of Tables}

%%% Introduction %%%
\newpage
\section{Introduction}\thispagestyle{empty}
\lipsum

%%% Methods %%%
\newpage
\section{Methods}\thispagestyle{empty}
\lipsum

%%% Results %%%
\newpage
\section{Results}\thispagestyle{empty}
\lipsum

%%% Discussion %%%
\newpage
\section{Discussion}\thispagestyle{empty}
\lipsum

%%% Conclusion %%%
\newpage
\section{Conclusion}\thispagestyle{empty}
\lipsum

%%% Bibliography %%%
\addcontentsline{toc}{section}{Bibliography}
\newpage

\end{document}