\documentclass[a4paper]{article} % twoside for paper submission but remove for electronic submission
\title{Reconciling Resource Supply and Ontogenetic Growth (Models)}
\author{Luke Joseph Vassor}

%%%% Packages %%%

% Page Formatting packages
\usepackage[margin=2cm]{geometry}
\usepackage{lipsum} % Generates dummy text
\usepackage{titlesec}
    \titleformat*{\section}{\LARGE\bfseries}
    \titleformat*{\subsection}{\normalsize\bfseries}
    \titleformat*{\subsubsection}{\large\bfseries}
    \titleformat*{\paragraph}{\large\bfseries}
    \titleformat*{\subparagraph}{\large\bfseries}
    \newcommand{\sectionbreak}{\newpage} % forces new page for each section
    \titlespacing*{\section}
{0pt}{20cm plus 1ex minus .2ex}{4.3ex plus .2ex}
\usepackage{etoolbox}

\makeatletter
\patchcmd{\ttlh@hang}{\parindent\z@}{\parindent\z@\leavevmode}{}{}
\patchcmd{\ttlh@hang}{\noindent}{}{}{}
\makeatother
\usepackage{fancyhdr} % rule line and current section displayed at top of pages % \thispagestyle{plain} removes it on desired pages
    \pagestyle{fancy}
    \renewcommand{\footrulewidth}{0.4pt}% default is 0pt
    \renewcommand{\headrulewidth}{0.4pt}% default is 0pt
    \renewcommand{\sectionmark}[1]{\markboth{#1}{}} % set the \leftmark to section name
    \fancyhead[L]{\nouppercase{\leftmark}} %  sets the left head element ONLY EVER section name, otherwise mark is section/subsection/subsubsection
    \fancyhead[R]{} % sets the right head element to the page number if you use \thepage as argument
    \fancyfoot[C]{\thepage} % sets footer as page number

\usepackage{pdfpages}
% \usepackage{lscape}
% Math packages
\usepackage{amsmath}
\usepackage{amssymb}
% \usepackage{mathpazo}
\usepackage{xpatch}
\usepackage{todonotes}
% Language packages
% \usepackage[english]{babel}
\usepackage[utf8]{inputenc}    % utf8 support       %!!!!!!!!!!!!!!!!!!!!
\usepackage[T1]{fontenc}       % code for pdf file  %!!!!!!!!!!!!!!!!!!!!
% \usepackage{listings}

% Other packages
% \usepackage[font=small,skip=0pt]{caption}
\usepackage{lmodern}% http://ctan.org/pkg/lmodern
\usepackage{slantsc}
\usepackage{xcolor}
\usepackage{parskip}
\usepackage{tocloft}
    \renewcommand{\cftsecleader}{\cftdotfill{\cftdotsep}} % for sections
% \usepackage[colorinlistoftodos]{todonotes}

% Plotting packages
\makeatletter
\newcommand{\thickhline}{%      % for thick lines in table
    \noalign {\ifnum 0=`}\fi \hrule height 1pt
    \futurelet \reserved@a \@xhline
}
% \usepackage{subfigure}
\usepackage[pyplot]{juliaplots}
\usepackage{wrapfig}
\usepackage{graphicx}
\usepackage[font=small,labelfont=bf]{caption} % Required for specifying captions to tables and figures
\usepackage{tabularx}
% Citation packages
\usepackage[maxcitenames=2, backend=biber, dashed=false, style=imperialharvard]{biblatex}
    \addbibresource{./CMEE_Thesis.bib}
    \setlength\bibitemsep{0.5\baselineskip}
    % \renewcommand*\finalnamedelim{\addspace\&\space}

% Force \textit{et al}. to be italicised

    \xpatchbibmacro{name:andothers}{%
        \bibstring{andothers}%
    }{%
    \bibstring[\emph]{andothers}%
    }
    \xpatchbibmacro{Name}{%
        \bibstring{name}
    }{%
    \bibstring[\textbf]{name}%
    }{}{} % leave these empty arguments here, seem to cause issues when ommited
    
% Kill "Accessed on" lines in bibliography
\AtEveryBibitem{
    \clearfield{urlyear}
    \clearfield{urlmonth}
    \clearfield{doi}
    \clearfield{url}
    \clearfield{eprint}
    \clearfield{eprinttype}
}
\usepackage[capitalise]{cleveref}

\begin{document}

\begin{titlepage}
    % Square brackets control size of vertical gap AFTER group
    %----------------------------------------------------------------------------------------
    %	LOGO SECTION
    %----------------------------------------------------------------------------------------
    
    \includegraphics[width=4cm]{./Images/logo.png}\\%[1cm] % Include a department/university logo - this will require the graphicx package
     
    %----------------------------------------------------------------------------------------
    
    \center % Center everything on the page
    
    
    %----------------------------------------------------------------------------------------
    %	TITLE SECTION
    %----------------------------------------------------------------------------------------
    \makeatletter
    \linespread{1.5} % controls line spacing of title
        {\huge{RECONCILING RESOURCE SUPPLY AND ONTOGENETIC GROWTH IN FISH}\par} % leave \par here
    \vspace{2.5cm} % Title of your document

    %----------------------------------------------------------------------------------------
    %	DESCRIPTION SECTION
    %----------------------------------------------------------------------------------------
    % If you want consistent line spacing then need to typeset paragraphy as a single {} group and use \\ for line break
    % Strangely, you need to leave the \ character at the end of the group so that the spacing of the last line from the penultimate line in a group remains constant
    % textsc command sets small capitals (sc)
    \textsc{A thesis submitted in partial fulfilment \\ of the requirements for the degree of \\ Master of Science at Imperial College London \\ by \\ \ }\\[2.5cm]
    \textsc{\Large \@author}\\[2.5cm]
    \textsc{Submitted for the \\ M.Sc. Computational Methods in Ecology \& Evolution \\ Formatted in the journal style of \textsl{\textsc{Ecology Letters}} \\ \ }\\[2cm]

    %----------------------------------------------------------------------------------------
    %	HEADING SECTIONS
    %----------------------------------------------------------------------------------------
    \textsc{Department of Life Sciences \\ Imperial College London \\ \ }\\[1cm]
    \textsc{\today}\\[2cm] % Date, change the \today to a set date if you want to be precise
    
    \vfill % Fill the rest of the page with whitespace
    
\end{titlepage}

%%% Declaration %%%
\section*{Declaration of Originality}\thispagestyle{plain}
    Declaration: The first page inside the cover must provide a brief declaration of the contributions
    made by you and by others to your project. Key points to address are:
    \begin{itemize}
        \item Was the data provided to you or did you collect or assemble it?
        \item Were you responsible for data processing or cleaning, if required?
        \item Were any mathematical models developed by you or by your supervisor?
        \item What role, if any, did your supervisor play in developing the analyses presented?
    \end{itemize}
    I certify that this thesis, and the research to which it refers, are the product of my own work, conducted during the current year of the \emph{M.Sc. Computational Methods in Ecology \& Evolution} at Imperial College London. Any ideas or quotations from the work of other people, published or otherwise, or from my own previous work are fully acknowledged in accordance with the standard referencing practices of the discipline and this institution.

    \textbf{Contributions:} Dr Samraat Pawar, Dr Diego Barneche, Dr Van Savage and Tom Clegg conceived the study. Diego Barneche compiled data on fish growth. S.P., T.C. and L.V. developed the idea.
    \vspace{3cm}
    \begin{flushright}
        Luke Joseph Vassor \\
        \today
    \end{flushright}

%%% Abstract %%%
\section*{Abstract}\thispagestyle{plain}
    The amount of metabolic energy an organism can sequester from its environment fundamentally determines its scope for growth and reproduction. Here I use an energetics and life history theoretical model to estimate the optimum mass-proportion and mass-scaling values of fecundity rate in fish, under different energy intake regimes, to test a recent claim that it scales hyperallometrically with mass. Theoretically, I show that a truly continuous rate of allocation to fecundity cannot scale hyperallometrically with size. I also derive and validate the use of a new intake rate value, based on fish energetics data and field metabolic rate mass-scaling, by demonstrating that the onset of reproductive allocation at maturity is only possible given these intake rate values. Overall, the study highlights the importance of correctly characterising growth and the fitness contribution of different sized individuals to a population.

    \textbf{Keywords} \\
    Biomass; energetics; productivity; allometry; life history; fisheries
%%% Acknowledgements %%%
\section*{Acknowledgements}\thispagestyle{plain}
    % Special thanks goes to Dr Samraat Pawar for providing excellent supervision and facilitating  highly-stimulating, always comical, dialogue throughout the 5-month thesis period, and the CMEE course in general. Likewise to Tom Clegg, for his part in this, and his unwavering patience in providing frequent additional support and discussion outside of scheduled meetings. Additional thanks goes to Dr Diego Barneche, Prof Van Savage and Prof Dustin Marshall for allocating time to this problem on their visits to Silwood Park, to Francis Windram and Alex Christensen for coding and mathematics assistance, respectively, to Deraj Wilson-Aggarwal for moral support and frequent Barista skills, and to the remaining members of PawarLab for providing such an open, intellectual and hilarious environment. Lastly, the completion of this thesis, or the degree in general, would not have been possible without the efforts and support of my family, who have assisted in any way possible to make life easier, be that via food or transport, I will always be grateful to them. Nor would it have been possible without the support of my partner, Tiffany, who, despite her passion for a different field, has tolerated my endless utterances and vocalised cognitive grapplings with metabolic, life history and game theory for the last 5 months - something tells me I should hold on tight.

%%% Table of Contents %%%
\newpage\tableofcontents\thispagestyle{plain}

%%% List of Figures %%%
\newpage\listoffigures\thispagestyle{plain}
\addcontentsline{toc}{section}{List of Figures}

%%% Listof Tables %%%
% \newpage\listoftables\thispagestyle{plain}
% \addcontentsline{toc}{section}{List of Tables}

%%% Notation %%%
\newpage\section*{Notation}\thispagestyle{plain}
\addcontentsline{toc}{section}{Notation}

\begin{table}[h]
    \centering
    % \caption{Parameterisations for the scaling relationships underlying the growth model. All units are in SI.}
    \begin{tabularx}{0.7\linewidth}{ll}
    \thickhline
    \textbf{Description}                                                & \textbf{Symbol}       \\ \thickhline
    Change in body mass with (continuous) time                          & $\frac{dm}{dt}$       \\
    Body mass                                                           & $m$                   \\ \hline
    Energy Intake/Acquisition Rate \textsc{coeff}                       & $a$                   \\ 
    \textsc{scaling exponent}                                           & $y$                   \\ \hline
    Maintenance Metabolic Rate \textsc{coeff}                           & $b$                   \\ 
    \textsc{scaling exponent}                                           & $z$                   \\ \hline
    Allocation to reproduction Rate (estimated by GSI*)\textsc{coeff}   & $c$                   \\  %\autocite{peters1983,Blueweiss1978}
    \textsc{scaling exponent}                                           & $\rho$                \\ \hline %$1.29$ \autocite{Barneche2018-reproductive_output}
    ATR = allocation to reproduction                                    & ATR                   \\ \hline
    Age at maturity (onset of reproduction)                             & $\alpha$              \\ \hline
    Consumption rate                                                    & $C(t)$                \\ 
    \textsc{scaling exponent}                                           & $\gamma$              \\ \hline
    Foraging length \textsc{coeff}                                      & $t_0$                 \\
    \textsc{scaling exponent}                                           & $\psi$                \\ \hline
    Mass at maturity                                                    & $m_{\alpha}$          \\ \hline
    Asymptotic/terminal size                                            & $M$ or $m_{\infty}$   \\ \hline
    Probability of survival to time $t$                                 & $L_t$                 \\ \hline
    Rate of instantaneous mortality                                     & $Z$                   \\ \thickhline
    \end{tabularx}
\end{table}

*Gonadosomatic Index; proportion of somatic body mass given to repro per year

%%% Introduction %%%
\newpage
\section{Introduction}\thispagestyle{plain}
    Organisms must grow by synthesising biomass from consumed resources in order to progress through ontogeny, or stages of life. The rate of biomass production directly influences fitness at the individual level by constraining the speed at which maturity is reached. Beyond the individual, growth has scalable, measurable impacts at multiple ecological levels, by constraining food-web trophic structure and energy transfer efficiency \autocite{Barneche2018}. Biologists have sought to gain a proximate and ultimate understanding ontogenetic growth for over a century. \textit{Why do organisms grow at a specific rate during a specific stage? Why do they stop growing? What causes this? Can growth be controlled, and if so, why? What is the optimal size to grow to? When is the optimal age to mature?} These are all valid and relevant questions, answers to which require a theoretical understanding of the energetics of growth and how evolution has selected for the patterns we observe given these constraints. 
    
    Energetic constraints dictate the variety of sizes and, consequently, the fantastic spectrum of niches resident in the biosphere since most key physiological, ecological and life history traits covary with body size \autocite{peters1983, brown2000-scaling-book,schmidt1984scaling,Marshall2019b}. To quote \cite{Bartholomew1981} ``It is only a slight overestimate to say that the most important attribute of an animal, both physiologically and ecologically, is its size''. As such, growth modelling has historically been a popular endeavor in the biological sciences, demanding a knowledge of the mechanisms which shape resource allocation and evolutionary strategy. This prerequisit has attracted theorists, field ecologists and applied scientists, in a joint endeavor to translate growth as a quantifiable, mathematisable and testable idea into a theoretical, predictive framework \autocite{popper1962,popper1972,peters1983, West2011}. Geometric scaling laws are especially relevant to this pursuit since many biological traits scale with body size, governed by a power law: $Y = Y_0 M^{\beta}$, where $Y$ is the trait to be predicted, $M$ is body mass, and $Y_0$ and $\beta$ are empirically-derived constants. These laws have been formalised into ``allometric" equations. If $\beta = 1$, the scaling is said to be ``isometric", while if $\beta \neq 1$, the relationship is called allometric, and plots as a curve on linear axes \autocite{brown2000-scaling-book}.

    Historically, the approaches used to model ontogenetic growth bifurcate into two major branches. Evolutionary Life History Theory, which relates growth, phenomenologically, to the optimum timing of fundamental life history events, and Ecological Metabolic Theory, which relates growth, mechanistically, to fundamental cellular and energetic processes which constrain the scope for growth.
        
    Evolutionary life history theorists have typically employed optimisation techniques to solve for the age and size values of given life history events which maximise fitness, e.g. age-at-maturation. The \textit{modus operandi} is to assume that evolution selects for timing and growth strategies which do this by optimising trade-offs among competing traits \autocite{Day1997, Stearns1989, stearns1992evolution}. Typically, then, simplifying assumptions are made with regard to energetic mechanisms \autocite{Day1997, Kozowski1987-indeterminate}, which are view as the end evolutionary result of selection on body size, an exemplar of a ``top-down'' approach.
        
    Converseley, metabolic theorists utilise laws from thermodynamics and enzyme kinetics as a first-principles approach to growth problems, starting with energetics and ending with body size, i.e. a ``bottom-up'' approach \autocite{Brown2004}. Paradigmatically, lifetime growth is governed by the distribution and transformation of available energy. These bioenergetic growth models are predicated on the premise that growth is constrained by an energy scope or profit. An organism garners energy from its environment at a certain rate (revenue), some of which it expends on internal maintenance (metabolism) at a certain rate (cost). Any surplus energy remaining after this expenditure, can then be used to synthesise new biomass for growth \autocite{Holdway1984, Rochet2001}. Growth slows because maintenance rate scales with mass to a larger exponent than intake rate (different power laws), so this energy profit decreases with size (see \cref{growth_schedules}). Extant models agree upon a mass exponent of 1 for maintenance rate (double the body mass = double the number of cells), whereas the mass exponent of intake rate has been debated through time, given its derivation from different geometric mechanisms. Early models suggested a value of $2/3$ \autocite{Putter1920,vonBert1938, VonBertalanffy1957}, due to intake being governed by absorbtion across a membrane surface (rearrangement of $\pi r^2$). However, recent models tend to use a value of $3/4$, proposed by \cite{West1997}, instead due to energy transport begin governed by a fractal-like supply network (e.g. capillaries), the result being their ``general ontogenetic growth model'' (see \cref{west_ogm}).

    \begin{center}
        \begin{minipage}{\linewidth}
        \includegraphics[width=4cm,height=4cm,keepaspectratio]{../Other/Presentation/determinate.pdf}
        \includegraphics[width=4cm,height=4cm,keepaspectratio]{../Other/Presentation/indeterminate.pdf}
        \includegraphics[height=4.3cm]{../Other/Presentation/logistic_growth_curve.pdf}  
        \label{growth_schedules}
        \captionof{figure}{Since power laws have the convenient property of the exponent being equal to the slope after a logarithmic transformation, $\log{(am^y)} = y\cdot \log{(am)}$. Determinate growth evolves from having energy intake and maintenance rates which intercect before death. Indeterminate growers do have a theoretical asymptotic size, due to the finite nature of energy resource, but this is never reached.}
        \end{minipage}%
    \end{center}

    \cite{Charnov2001} continued this energetics approach in a fish life history framework by developing the \cite{West2001} model to include a fecundity rate term, acknowledging that this only appears at maturity. Fish are ideal model organisms for growth modelling given their immense ecological and economic importance. Across a global range of freshwater and marine habitats, they represent the highest vertebrate species richness and exceed 8 orders of magnitude in their range of body masses \autocite{Barneche2018c}. In fisheries management, knowledge of time  and food required to reach maturity is integral to sustainable management of stocks \textcolor{red}{(Szuwalski et al. 2017)}
    
    While \cite{Charnov2001} assumed that fecundity rate scales isometrically with mass, recently, \cite{Barneche2018-reproductive_output} showed that fish reproductive output scales hyperallometrically with body mass, i.e. larger mothers are disproportionately more fecund. This opened the question of whether the rate of allocation to fecundity also scales with mass in this way. As such, a consequent review extended the \cite{Charnov2001} model to include this hyperallometry \autocite{Marshall2019b} (see SI for full derivation). However the data presented by \cite{Barneche2018-reproductive_output} is problematic as they represent batch fecundity at discrete time points, a consequence of the typical sampling technique used to measure fecundity, which involves catch and dissection to take gonad measurements (Barneche, 2019, pers comm). Unlike intake rate and maintenance rate, this fecundity output term is not based on an instantaneous rate. Instead it captures snapshots of fecundity at different sizes and it remains unclear whether this hyperallometry would apply to an instantaneous rate of allocation across continuous time. Should this rate scale hyperallometrically with mass, how they increase their energy surplus to permit this extra cost comes into question.

    Resource intake rate scaling has previously been shown to exhibit environment and dimensionality-dependence, which most extant models ignore \autocite{Pawar2012}. Many assume a simple relationship between energy intake rate and resting metabolic rate, such that per-capita resource supply scales with consumer body size ($m$) to an exponent of $3/4$ (see above), irrespective of taxon or environment \autocite{Pawar2012}. Environment and dimension-dependence means that consumption rate can, in fact, scale to an exponent as large as $1.06$ and as low as $0.85$, the latter being the approximate mass-scaling exponent of field metabolic rate \autocite{peters1983,Weibel2004, Pawar2012}. Since fish experience field conditions throughout life, it may, in fact, be more prudent to assume a relationship between intake rate and field metabolic rate, rather than resting \autocite{Boisclair1993}. It logically follows that the steeper scaling of intake rate may provide the energetic compensation required for a hyperallometric fecundity rate scaling.

    Together, these new results on fecundity rate and intake rate scaling reveal disconnects in the ontogenetic growth modelling literature. In this paper, I show that allocation to fecundity, as an instantaneous rate, is theoretically unlikely to exhibit hyperallometric mass-scaling in fish mothers, under the traditional intake rate scaling regime which uses the canonical $3/4$ exponent. When I increase this exponent, causing intake rate to scale steeper, I then show that reproductive hyperallometry can emerge, due to the compensation effect of the larger intake rate.
    
    These theoretical findings will be crucial in generating new hypotheses for testing empirical data and for highlighting the distinction between continuous rate data and discrete time data. Further, these findings likely warrant update of how existing models consider intake rate, since the reason we do not observe shrinking trajectories in reality may be because larger intake rate scaling permits larger (hyperallometric) scaling exponents of fecundity. They will also be in creating space for new theory to develop in understanding the energetic mechanisms behind fecundity rate allocation. Finally, bridging these questions will  Quantifying the effects of these new findings for fish growth models will have profound implications for fisheries management and for growth modelling as a scientific practice in general.
%%% Methods %%%
\section{Materials and Methods}\thispagestyle{plain}
\subsection{Can instantaneous fecundity rate scale hyperallometrically with mass in fish mothers?}

To test whether hyperallometry is possible for an instantaneous rate of fecundity allocation, I developed the \cite{Charnov2001} approach by using a biphasic, hybrid model which utilises Ecological Metabolic Theory and Life History Theory which captures the energetics of growth during two distinct ontogenetic stages. Mature fish experience continuous diversion of resources to fecundity which scales to the $\rho$ exponent of mass (see \cref{growth_curve}). I also used an updated value of the intake rate coefficient, $a$, derived from fish-specific energetic values. In order to theoretically endorse their model-fit results for the value of $a$, \cite{West2001} used fundamental cellular properties to derive an approximate value of $a$. Investigation of their calculation of $a$ revealed it was flawed when applied to fish data, warranting update (see SI).
\begin{align}
    \frac{dm}{dt} &= am^{3/4} - bm \ \ \ \ \ \ \ \ \ \ \ \ \ \ m < m_{\alpha} \label{luke_model_juvenile}\\
    \frac{dm}{dt} &= am^{3/4} - bm - cm^{\rho} \ \ \ \ \ m \geq m_{\alpha} \label{luke_model}
\end{align}

The model is based on the Life History Theory concept that natural selection optimises strategies, e.g. $c$ and $\rho$, to maximise fitness, where lifetime reproductive output can be used as a proxy for fitness, denoted $R_0$, which can be derived from theoretical evolution studies \autocite{Charnov2001, stearns1992evolution}. To this end, I tested the model via simulations which allowed the fecundity rate parameters $c$ and $\rho$ to vary in order to maximise $R_0$, which is calculated using a life history model, developed from \cite{Charnov2001}. Since the scope for reproduction is determined by how much energy is available, it follows that the optimal values of $c$ and $\rho$ are dependent on the value and scaling of intake rate $am^{x}$, addressed in \textit{H3}. 

At any time $t$, $b_{t}$ is the \textit{effective} energy allocated by fish to reproduction, the product of the physiological allocation of resources $cm^{\rho}$ and an efficiency term $h(m)$ representing a declining efficiency of this allocation, known as reproductive senescence, the natural decline in fecundity as fish age \autocite{Stearns2000, Benoit2018, Vrtilek2018}. This decline begins at maturity ($\alpha$) and is controlled by a variable rate parameter $\kappa$. Fish also experience an extrinsic mortality rate, or actuarial senescence, contained in a surivorship function, $l_t$, which is effectively a declining $\mathbb{P}$(survival to $t$) \autocite{Beverton1959, Peterson1984, Charnov1993,Walters1993, Charnov2001, Benoit2018, Laird2010, Reznick2002, Reznick2006}. To the best of my knowledge, this study is the first intance of this incorporation of reproductive senescence into a growth and life history model. It is important to note than reproductive and acturarial senescence are functions of time or age, whilst allocation to reproduction is a function of mass. Thus, the instanteous reproductive output at time $t$ is the product $l_{t}b_{t}$ and the lifetime (cumulative) reproductive output is:
\begin{equation}
    R_{0} = \int_{\alpha}^{\infty}l_{t}b_{t} dt
\end{equation}
Since fish live in a juvenile and adult phase, they are subject to varying mortality rates through ontogeny \autocite{Charnov2001}. Juvenile mortality ($t_0 \rightarrow t_{\alpha}$) controls how many fish are alive at $\alpha$ and recruited into the adult phase. Since this follows an exponential distribution, $l_t = e^{-Z(t)}$ bounded [0,1], it acts as a scaling factor, denoted $L_{\alpha}$, for the mature population ($t_{\alpha} \rightarrow t_{\infty}$), which controls how many individuals reach maturity \autocite{Charnov1990-agematurity}. For adults survival is relative to when maturity is reached, $l_{t} = e^{-Z(t-\alpha)}$. Therefore the lifetime survivorship of fish is the product of the juvenile survivosrhip, $L_{\alpha}$ and adult survivorhsip. Together these form the ``characteristic equation'' \todo{CITE ROFF1992,2002;STEARNS 1992; ARENDT; TSOULAKI}
\begin{equation}
    R_{0} = c\int_{0}^{\alpha}e^{-Z(t)}dt\int_{\alpha}^{\infty} m(t)^{\rho} e^{-(\kappa+Z)(t-\alpha)} dt \label{LHT_optimisation}
\end{equation}
Common in comparative life histories in fish is the of use invariant dimensionless quantities derived from the timing of life history events. That is, across species but within a taxon, certain life history variables, representing the timing and magnitude of reproduction form, dimensionless, invariant ratios \autocite{Charnov1990-invariant, Charnov1993}. It has been shown for fish that the ratio of age-at-maturity and mortality rate, $\alpha\cdot Z \approx 2$ \autocite{Charnov1993}. Logically, this invariant makes sense since delaying maturation, or increasing $\alpha$ ($\alpha \approx 2/Z$) is only a feasible strategy if the risk of dying is low enough, or increases less quickly. Rearranging this for $Z$ estimates mortality rate for a given $\alpha$ value $Z = 2/\alpha$.
See SI for full derivation of \cref{LHT_optimisation}.

Maximising $R_0$ requires analytically solving \cref{LHT_optimisation} for values of $c$ and $\rho$ which do so. Since \cref{LHT_optimisation} has no closed-form solution, I simulated this numerically using the \texttt{DifferentialEquations} and \texttt{DiffEqCallbacks} packages in Julia v1.1.1 \autocite{Bezanson2017}, which ran the Rosenbrock optimisation function \autocite{Rosenbrock1960}. The following parameter space was simulated: $0.001 < c < 0.4$ and $0.001 < \rho < 1.25$ \todo{justification for these ranges} with 100 linearly-spaced values over a lifespan of $1e6$ days, to ensure all growth trajectory simulations reached asymptotic size (\textcolor{red}{LV Note: but fish don't reach their asymptotic size? Isn't this unrealistic?}). I produced a heatmap of the fecundity rate parameter space, with an optimum $c, \rho$ combination, for a fixed intake rate mass-scaling ($x$) and reproductive senescence rate ($\kappa$). Since the evolutionary goal is to maximise lifetime reproductive output, natural selection in fish will inevitably tend towards these optima across time, and thus optimum value combinations theoretically estimate if hyperallometrically scaled fecundity rate is possible.

Preliminary simulations of the model resulted in some growth curves which exhibit shrinking (i.e. loss of mass at maturity), due to large values of $c$ and $\rho$ causing too much loss, resulting in $dm/dt < 0$. Since shedding of somatic mass to reproduce is not biologically realistic, I first screened for these shrinking curves by only preserving the feasible parameter space of $c$ and $\rho$ which did not cause shrinking.
By considering fecundity allocation as a rate across the entire fish lifetime which accounts for time in between fecundity output measurements, I expect the mass-scaling to go down because ..

\subsection{Is fecundity rate hyperallometry more likely when intake rate mass-scaling is steeper?}

The $3/4$ scaling of intake rate is set by resting metabolic rate, which scales to the $3/4$ exponent of mass \autocite{Kleiber1947, peters1983, niklas1994plant} due to the approximate fractal architecture of supply networks which become more deeply nested with branches as body size increases \autocite{West1997}. This geometry has evolved due to natural selection optimising energy transport to the cells and consequently, as size increases, the number of terminal units (capillaries) scales to the $3/4$ exponent of mass \autocite{West1997, West2005}. As fractals are mathematically considered to have non-integer dimensions \autocite{Hausdorff1918, Mandelbrot1982}, this gives rise to non-integer size-scaling. 

Resource consumption rate has been shown to scale with mass more steeply than the canonical $3/4$ exponent, argued to be more likely related to field metabolic rate mass-scaling (exponent = 0.85), versus resting metabolic rate. Given the restrictive assumptions underlying resting metabolic rate of no foraging (food is provided \textit{ad libitum}), growing or reproducing, it seems far more prudent to relate intake rate to the mass-scaling of field metabolic rate. This is especially so for the last two assumptions, no growth or reproduction, which are both violated as part of this exercise. Shrinking curves are caused by an inability of intake rate to compensate for the large costs borne by maintenance rate and high values of $c$ and $\rho$. Therefore, I predict that increasing the scaling of intake rate to a more biologically realistic value (0.85) will open up the parameter space for larger values of $c$ and $\rho$, since fish will have more available energy to use.

% \begin{table}[h]
%     \caption{Parameter values for model fitting from \cite{West2001}}
%     \begin{center}
%     \begin{tabular}{|l|l|l|l|l|}
%     \hline
%     Organism            & $a$   & $m_0$   & $M$     & Slope \\ \hline
%     Cow                 & 0.28  & 33.3 kg & 442kg   & 1.08  \\ \hline
%     Pig                 & 0.31  & 0.90 kg & 320kg   & 1.08  \\ \hline
%     Rabbit              & 0.36  & 0.12 kg & 1.35 kg & 1.34  \\ \hline
%     Guinea pig          & 0.21  & 5 g     & 840g    & 0.91  \\ \hline
%     Rat                 & 0.23  & 8 g     & 280g    & 1.07  \\ \hline
%     Shrew               & 0.83  & 0.3 g   & 4.2 g   & 0.98  \\ \hline
%     Salmon              & 0.026 & 0.01 g  & 2.4 kg  & 1.01  \\ \hline
%     Guppy               & 0.10  & 0.008 g & 0.15 g  & 1.04  \\ \hline
%     Shrimp              & 0.027 & 0.0008g & 0.075g  & 0.82  \\ \hline
%     \end{tabular}
%     \end{center}
% \end{table}

%%% Results %%%
\section{Results and Discussion}\thispagestyle{plain}
Production of a fecundity-rate feasibility heatmap proved necessary to obtain growth curves which did not shrink, but instead experienced an inflection, where growth rate remained positive ($dm/dt > 0$) as energy is diverted to reproduction at age $\alpha$ (see \cref{growth_curve}).
\begin{center}
    \begin{minipage}{0.8\linewidth}
        \includegraphics[width=0.8\linewidth]{../Results/single_curve_with_onset.pdf}
        \captionof{figure}{Growth undergoes an inflection, as it approaches an asymptote, at maturity when $\alpha = 200$, as opposed to shrinking, which would be rejected by the optimisation algorithm since shrinking in fish is not biologically possible. At maturity, growth rate slows (inflection) due to the diversion of resource to fecundity, leaving less scope for growth. Simultaneously, lifetime reproduction (cumulative) increases to an asymptote, when this quantity is less and less likely to increase given the exponential decay of survivorship probability, $l_t$.} 
        \label{growth_curve}
    \end{minipage}\\
\end{center}
\vspace{0.5cm}

\subsection{Can instantaneous fecundity rate scale hyperallometrically with mass in fish mothers?}
The life history optimisation results show that, theoretically, instantaneous fecundity rate in mature fish is unlikely to scale hyperallometrically with mass. These results highlight the importance of age-at-maturity in a fish's ability to then devote a large proportion of body mass to reproduction, under the paradigm of the model. In order for fecundity rate to scale hyperallometrically, of which there was a single instance, a fish must mature very early. Juvenile fish are not subject to a fecundity rate cost in the model, only maintenance cost (see \cref{luke_model_juvenile}), thus they grow far more rapidly than in the mature phase (see \cref{growth_curve}) when this extra cost is incurred (see logistic growth). Moreover, if this growth rate is too rapid, fish will reach a size at maturity $m_{\alpha}$ which is too large \autocite{Arendt2011} to permit hyperallometric scaling of fecundity rate, as this is an exponent of their mass. If mass is too large when this extra fecundity cost $cm^{\rho}$ commences, since this cost is related to their (now) large mass, raised to an exponent, their overall cost quickly exceeds their energy intake, causing $dm/dt < 0$ and shrinking. Therefore, as maturity age increases, the optimum $\rho$ decreases (see \cref{alpha_sensitivity_0.75}). This highlights the mathematical dynamics of the model which play out given the non-linearity of the intake and fecundity rate terms. Exponents dominate coefficients, so as a result have far narrower ranges of feasible values, evidenced by the shape of the feasible parameter space in \cref{heatmap_0.75}. Since I filtered out values which caused shrinking fish, in maintaining biological realism, the only scaling exponent values which remained were hypoallometric, except for the one instance at a very young maturity age (see \cref{alpha_sensitivity_0.75}). If $\alpha$ is, in reality, in the order of hundreds of days, or years, then these results support previous suggestions that larger adults invest relatively less in reproduction, with mass-scaling exponents of 0.5 - 0.9 \autocite{Reiss1985, Stearns2000}. At these older maturity ages, fish will have reached much larger sizes (given the lack of fecundity cost), and so only small $c$ and $\rho$ values are permitted to avoid shrinking. If hyperallometric scaling of fecundity rate does in reality, occur in fish, then these results corroborate the idea that relative to the timescale of a fish's lifetime, the sexually immature phase is negligible, therefore growth is well approximated by a single equation which encompasses the costs incurred by sexual maturity (\cref{luke_model}) in their lifetime \autocite{West2001}. The speed at which fish reach maturity in early life means they can essentially be viewed as being born mature. By maturing very early, fish effectively do not ``let'' themselves grow too large.

\begin{center}
    \begin{minipage}{0.7\linewidth}
    \includegraphics[width=0.9\linewidth]{{../Results/opt_hm_Alph=200.00_a=2.15_x=0.75_k=0.01_tex}.pdf}
    \captionof{figure}{Optimum fecundity rate parameters $\rho,c$ when $am^x = (2.15)m^{0.75}, \kappa = 0.01$. White 'O' locates optimum combination.}
    \label{heatmap_0.75}
    \end{minipage}%
\end{center} 
\vspace{0.5cm}

\subsection{Is fecundity rate hyperallometry more likely when intake rate mass-scaling is steeper?}
Increasing the value of the intake rate scaling exponent to 0.85 theoretically makes fecundity rate hyperallometry more likely. Furthermore, it endorses the above claim regarding maturity age by permitting hyperallometry at older maturity ages (see \cref{alpha_sensitivity_0.85}). In essence, under a higher intake regime, fish can delay maturity and reproductive hyperallometry can still emerge at these large sizes. Increased intake rate translates the curved fecundity rate parameter space upwards in the $\rho$ plane (\cref{heatmap_0.85} vs \cref{heatmap_0.75}), pushing several optimum $\rho$ values above 1 (see \cref{alpha_sensitivity_0.85}). This is an intuitive result, since the steeper scaling of intake rate now provides the energy needed to compensate for the extra loss incurred by higher fecundity rates, which will be optimally increased to maximise lifetime reproduction. Mathematically, a small alteration to an exponent like this can substantially alter the behaviour of such a model, given that when $m > 1 g$, the exponent will dominate any changes in the coefficient. As such, even slight increases in intake rate scaling will permit greater values of $c$ and $\rho$. 

These results extend a dilemma known formerly as ``the general life history problem'', in which indeterminate growers are subject to a trade-off between increased mortality at maturity and maximising lifetime reproductive output \autocite{Stearns2000}. Given these results, if a fish can benefit disproportionately in fecundity by delaying maturity (which is permitted by higher intake scaling) and growing larger, they face the decision of doing so while simultaneously decreasing their probability of surviving that long. Essentially, the reward in delayed fecundity (resulting from hyperallometry) must be greater than the risk of mortality \autocite{Arendt2011}. Given that, empirically, we know that some fish mature after years of growth, rather than instantly, (Cod: 2-4 years, reference. Salmon x years, reference), if fecundity rate hyperallometry is true, then we can predict that one of two things must be occuring: either intake rate scales steeper than the canonical $3/4$ to allow them to mature that late, or their adult mortality rate is less than expected. The latter is already captured by the $\alpha Z$ invariant, since Z has to decrease to allow a greater $\alpha$, however this exists under the assumption high mortality rate disincentivises fish from delaying maturity without consideration of the benefits, which include greater fecundity, more competitive and less vulnerable \autocite{Arendt2011} FORWARD CITE. It may be that even when $Z$ remains high, in a game theoretic sense, it is still optimal to delay maturity to benefit from the hyperallometric fecundity.

\todo{discuss the questions raised and empirical data we need for foraging - mention our attempt and the struggle with timescale}
The results raise questions surrounding our theoretical knowledge of intake rate.
it may be that following a glut of food, fish can differentially decide whether to invest in rapid growth or reproducing - even in a glut big fish would benefit more.
\todo{make clear the fecundity rate data will be very difficult to obtain}
\todo{develop SI with a derivation and cut shit out}
If anything it is actually necessary to have higher allometry scalings (assuming rho has a physiological maximum) at later maturity ages because they must compensate for the lost time to maximise lifetime repro. Delay wouldn't be a strategy if they couldn't do this (counter argument is that they delay it for the other reason, competition and less vulnerable \autocite{Arendt2011})
\listoftodos
Fecundity generally increases with body size for ectotherms \autocite{roff2002} see arendt
\begin{center}
    \begin{minipage}{0.7\linewidth}
        \includegraphics[width=0.9\linewidth]{{../Results/opt_hm_Alph=200.00_a=2.15_x=0.85_k=0.01_tex}.pdf}  
        \captionof{figure}{Optimum fecundity rate parameters $\rho,c$ when $am^x = (2.15)m^{0.85}, \kappa = 0.01$. White 'O' locates optimum combination.}
        \label{heatmap_0.85}
    \end{minipage}%
\end{center}

\todo{Sort out this secion:}
The inability of the canonical $3/4$ scaling of intake rate to compensate for values of $\rho > 1$ causes growth rate to drop below 0, rendering all these curves unfeasible and hence why no optimum values of $\rho > 1$ were observed in the results. Since fish face the constraint of not shrinking, in addition to maximising reproductive output, it logically follows that evolution will maximise fecundity rate parameter as much as possible given their energy supply. It further follows, then, that. This 

Evidently, the canonical $3/4$ scaling of intake rate restricts and minimises the fecundity rate parameter space and values themselves when shrinking curves are omitted. This is due to the fact that $3/4$ mass-scaling of intake rate is not large enough to overcome even small values of $\rho$, which dominates the fecundity rate term, and additively forms a large loss term with maintenance cost, causing $dm/dt < 0$. The result for $c$ suggests that the theoretical optimum is to devote 17\% of body mass to fecundity per day, which conforms to the existing literature, which suggest 10-25\%. Results also appear unaffected by a changing $\kappa$, suggeting that the decline in fecundity is not does not occur at a large enough magnitude to overcome the investment in reproduction itself.


These results highlight a caveat of the model, which is that the fixed parameter $\alpha$ has a fundamental effect on the feasible parameter space. The aforementioned $\alpha Z$ invariant does, of course, cause the life history model (\cref{LHT_optimisation}) to behave the same regardless of the chosen $\alpha$, since $Z$ compensates. However, this invariant does not discipline the growth model (\cref{luke_model}), which is where predictive power begins to be lost. The results presented reflect $\alpha = 200$, or a fish maturity age of 200 days. However, I predict that adjusting $\alpha$ will have profound effects on the optimum fecundity rate parameter values, due to their relatedness to mass. Suppose $\alpha$ is set at a larger value, e.g. 400 days. This means that for double the amount of time, our hypothetical fish is governed only by \cref{luke_model_juvenile}, which incurs less cost (no reproduction), thus in this time the fish can grow larger. At $\alpha$, when it matures, it has already reached a larger size, thus the reproduction term takes on a much larger value, especially due to $\rho$. This sudden and larger diversion of resources causes $dm/dt < 0$, or shrinking. This is, in fact, how the feasible parameter space in the model is calculated. Suppose, for a given $c, \rho$ combination, a fish is born mature, thus incurring a $cm^{\rho}$ cost for its whole lifespan, if this trajectory's $m_{\infty}$ is smaller than $m_{\alpha}$ when maturity is delayed (e.g. $\alpha = 200$), then this fish will always shrink under this model. When fish mature very early on ($\alpha \rightarrow 0$), they can never reach a size $m_{\alpha} > m_{\infty}$ because they incur these costs from birth. Thus, I predict that at larger values of $\alpha$, only very small values for $\rho$ are possible, if not at all, across a fixed intake rate. In the case that $\rho \approx 0$ and $c \rightarrow 0$, this indicates that reproducing at that age, at all, is not possible, unless another quantity changes, e.g. intake exponent increases. Alternatively, this would indicate that the architecture of the model itself is not realistic, in the sense that fish do not suddenly become mature in the timescale of a single day and incur an immediately large cost. Instead, they either allocate to fecundity from a very early age while remaining sexually immature, termed ``the dissipative processes of respiration''. Or, in reality, the timescale of the juvenile phase is almost negligible in contrast to the timescale of the lifetime, hence they are only governed by \cref{luke_model} \autocite{West2001}. 

\begin{center}
    \begin{minipage}{0.5\linewidth}
        \includegraphics[width=\linewidth]{{../Results/alpha_sensitivity_x=0.75}.pdf}  
        \captionof{figure}{Results of sensitivity analysis of optimum $\rho$ \\ values to $\alpha$ when $am^x = (2.15)m^{0.75}$}
        \label{alpha_sensitivity_0.75}
    \end{minipage}%
    % \hfill
    \begin{minipage}{0.5\linewidth}
        \includegraphics[width=\linewidth]{{../Results/alpha_sensitivity_x=0.85}.pdf}    
        \captionof{figure}{Results of sensitivity analysis of optimum $\rho$ \\ values to $\alpha$ for when $am^x = (2.15)m^{0.85}$}
        \label{alpha_sensitivity_0.85}
    \end{minipage}
\end{center}
\todo{Produce more alpha values for denser plot}
% Need a heatmap of $a = 0.02 $ to prove that West values don't work and validate my change in H2

%%% Conclusion %%%
\section{Conclusion}\thispagestyle{plain}
\lipsum

%%% Bibliography %%%
\addcontentsline{toc}{section}{Bibliography}
\newpage\let\mkbibnamefamily\textsc\printbibliography[title=Bibliography]\thispagestyle{plain} % Sets author names to small caps

%%% SI %%%
\addcontentsline{toc}{section}{Supplementary Information}
% \includepdf[pages=-]{Vassor_L_J_CMEE_Thesis_2019_Supplementary_Information.pdf}

\end{document}