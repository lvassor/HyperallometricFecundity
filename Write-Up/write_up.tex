\documentclass[a4paper, twoside]{report}

\usepackage[english]{babel}
\usepackage[utf8x]{inputenc}
\usepackage[T1]{fontenc}
\usepackage{listings}
\usepackage{hyperref}
\hypersetup{colorlinks=false}
\usepackage{lscape}
\usepackage{subfigure}
\usepackage{amsmath}
\usepackage{graphicx}
\usepackage[colorinlistoftodos]{todonotes}

%% Sets page size and margins
\usepackage[a4paper,top=3cm,bottom=2cm,left=3cm,right=3cm,marginparwidth=1.75cm]{geometry}

\title{Hyperallometric Fecundity}
\author{Luke J. Vassor}
% Update supervisor and other title stuff in title/title.tex

\begin{document}

\begin{titlepage}

    \newcommand{\HRule}{\rule{\linewidth}{0.5mm}} % Defines a new command for the horizontal lines, change thickness here
    
    %----------------------------------------------------------------------------------------
    %	LOGO SECTION
    %----------------------------------------------------------------------------------------
    
    \includegraphics[width=8cm]{./Images/logo.png}\\[1cm] % Include a department/university logo - this will require the graphicx package
     
    %----------------------------------------------------------------------------------------
    
    \center % Center everything on the page
    
    %----------------------------------------------------------------------------------------
    %	HEADING SECTIONS
    %----------------------------------------------------------------------------------------
    \quad\\[1.5cm]
    %\textsc{\LARGE MSc Thesis}\\[1.5cm] % Name of your university/college
    \textsc{\Large Imperial College London}\\[0.5cm] % Major heading such as course name
    \textsc{\large Department of Life Sciences}\\[0.5cm] % Minor heading such as course title
    
    %----------------------------------------------------------------------------------------
    %	TITLE SECTION
    %----------------------------------------------------------------------------------------
    \makeatletter
    \HRule \\[0.4cm]
    { \huge \bfseries \@title}\\[0.4cm] % Title of your document
    \HRule \\[1.5cm]
     
    %----------------------------------------------------------------------------------------
    %	AUTHOR SECTION
    %----------------------------------------------------------------------------------------
    
    \begin{minipage}{0.4\textwidth}
    \begin{flushleft} \large
    \emph{Author:}\\
    \@author % Your name
    \end{flushleft}
    \end{minipage}
    ~
    \begin{minipage}{0.4\textwidth}
    \begin{flushright} \large
    \emph{Supervisor:} \\
    Dr. Samraat Pawar 
    % Uncomment the following lines if there's a co-supervisor
    %\\[1.2em] % Supervisor's Name
    %\emph{Co-Supervisor:} \\
    %Dr. Adam Smith % second marker's name
    \end{flushright}
    \end{minipage}\\[3cm]
    \makeatother
    
    
    %----------------------------------------------------------------------------------------
    %	DATE SECTION
    %----------------------------------------------------------------------------------------
    
    {\large A thesis submitted for the degree of}\\[0.5cm]
    {\large \emph{M.Sc. Computational Methods in Ecology \& Evolution}}\\[0.5cm]
    {\large \today}\\[2cm] % Date, change the \today to a set date if you want to be precise
    
    \vfill % Fill the rest of the page with whitespace
    
\end{titlepage}

\begin{abstract}
    Your abstract goes here. The abstract is a very brief summary of the dissertation's contents. It should be about half a page long. Somebody unfamiliar with your project should have a good idea of what it's about having read the abstract alone and will know whether it will be of interest to them.
\end{abstract}

\renewcommand{\abstractname}{Acknowledgements}
\begin{abstract}
    It is usual to thank those individuals who have provided particularly useful assistance, technical or otherwise, during your project.
\end{abstract}

\tableofcontents
\listoffigures
\listoftables


\section{Introduction}

\bibliographystyle{unsrt}
\bibliography{bibs/sample}
\addcontentsline{toc}{chapter}{Bibliography}

\end{document}