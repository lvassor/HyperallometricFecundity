\documentclass[a4paper]{article} % twoside for paper submission but remove for electronic submission
\title{Reconciling Resource Supply and Hyperallometric Fecundity Scaling in Ontogenetic Growth Models}
\author{Luke Joseph Vassor}

% Packages
% Page Formatting packages
\usepackage[margin=2cm]{geometry}
\usepackage{lipsum} % Generates dummy text
\usepackage{float} % here for H placement parameter
\usepackage{fancyhdr} % rule line and current section displayed at top of pages % \thispagestyle{empty} removes it on desired pages
    \pagestyle{fancy}
    \renewcommand{\footrulewidth}{0.4pt}% default is 0pt
    \renewcommand{\sectionmark}[1]{\markboth{\textbf{\textcolor{blue}{SUPPLEMENTARY INFORMATION}}}{}} % set the \leftmark to section name
    \fancyhead[R]{\nouppercase{\leftmark}} %  sets the right head element ONLY EVER section name, otherwise mark is section/subsection/subsubsection
    \fancyhead[L]{} % sets the left head element to the page number if you use \thepage as argument
    \fancyfoot[C]{\thepage} % sets footer as page number
% Math packages
\usepackage{amsmath}
\usepackage{amssymb}
\usepackage{bm}
% Language packages
% \usepackage[english]{babel}
\usepackage[utf8]{inputenc}    % utf8 support       %!!!!!!!!!!!!!!!!!!!!
\usepackage[T1]{fontenc}       % code for pdf file  %!!!!!!!!!!!!!!!!!!!!
\usepackage{soul}
% \usepackage{listings}

% Other packages
\usepackage{longtable}
\usepackage{hyperref}
    \hypersetup{colorlinks=false}
% \usepackage[colorinlistoftodos]{todonotes}

% Plotting packages
% \usepackage{lscape}
% \usepackage{subfigure}
\usepackage{graphicx}
\usepackage{multicol}
\usepackage{soul}
\usepackage{tabularx}
\usepackage{array}

\makeatletter
\newcommand{\thickhline}{%      % for thick lines in table
    \noalign {\ifnum 0=`}\fi \hrule height 1pt
    \futurelet \reserved@a \@xhline
}
\usepackage{xcolor}
\usepackage{pdflscape}
\newcolumntype{s}{>{\hsize=1\hsize}X}
\newcolumntype{b}{>{\hsize=1\hsize}X}
\newcolumntype{m}{>{\hsize=1\hsize}X}
% Citation packages
\usepackage[maxcitenames=2, backend=biber, dashed=false, style=imperialharvard]{biblatex}
    \addbibresource{./CMEE_Thesis.bib}
    % \renewcommand*\finalnamedelim{\addspace\&\space}
% Force \textit{et al}. to be italicised
\usepackage{xpatch}
\xpretocmd{\cref}{Eq.~}{}{}
    \xpatchbibmacro{name:andothers}{%
        \bibstring{andothers}%
    }{%
    \bibstring[\emph]{andothers}%
    }
    \xpatchbibmacro{Name}{%
        \bibstring{name}
    }{%
    \bibstring[\textbf]{name}%
    }{}{} % leave these empty arguments here, seem to cause issues when ommited
    \xpatchbibmacro{textcite} % combine same author different year citations
  {\setunit{\addcomma}\usebibmacro{cite:extrayear}}
  {\setunit{\compcitedelim}\usebibmacro{cite:labelyear+extrayear}}
  {}
  {}
% Kill "Accessed on" lines in bibliography
\AtEveryBibitem{
    \clearfield{urlyear}
    \clearfield{urlmonth}
    \clearfield{doi}
    \clearfield{url}
    \clearfield{eprint}
    \clearfield{eprinttype}
}
\newcommand\numberthis{\addtocounter{equation}{1}\tag{\theequation}}
\usepackage{tocloft}
\renewcommand{\cftsecleader}{\cftdotfill{\cftdotsep}} % for sections
\usepackage{parskip}
\usepackage[capitalise]{cleveref}
\usepackage{setspace}
\onehalfspacing
\begin{document}

\begin{titlepage}
    
    %----------------------------------------------------------------------------------------
    %	TITLE SECTION
    %----------------------------------------------------------------------------------------
    \makeatletter
    % \vspace{4cm}
    \linespread{1.5} % controls line spacing of title
    {\huge\bfseries\textcolor{blue}{SUPPLEMENTARY INFORMATION}\par}
    \vspace{0.1cm}
    \hrule
    \vspace{0.1cm}
    \hrule
    \center % Center everything on the page   
    \vspace{2cm}
        
    \vspace{2cm} % Title of your document
    \tableofcontents
\end{titlepage}
% \begingroup
% \Large

%% Begin derivation from master equation %%

\section{Ontogenetic Growth Model}
\subsection{Model derivation from the Energy Balance Equation}
\textcite{West2001} developed their previous work on allometric scaling laws \autocite{West1997} to capture growth in their ``general model for ontogenetic growth''. Ontogenetic development is fuelled by metabolism. Incoming energy and materials from the environment are transported through hierarchical branching network systems to supply all cells. These networks become more deeply nested with size and are, thus, self-similar, approximating a fractal, constraining the number of capillaries to scale isometrically with size. These resources are transformed into metabolic energy, which is used for life-sustaining activities. During ontogeny, some fraction of this energy is allocated to the synthesis of new tissue. The rate of energy transformation, then, is the sum of two terms, one of which represents the maintenance of existing biomass, and the other, the synthesis of new tissue. This is expressed by the dynamic conservation of energy equation:
\begin{align}
    B &= N_{c}B_{c} + E_{c}\frac{dN_{c}}{dt} \label{master_balance}
\end{align}
The incoming rate of energy flow, $B$, is the resting metabolic rate of the whole organism at time $t$, $B_c$ is the metabolic rate of a single cell (\textsc{joules/sec}), $E_c$ is the metabolic energy required to create a single cell (\textsc{joules}) and $N_c$ is the total number of cells. Differences between tissues are ignored and some average typical cell is taken as the fundamental unit. The first term, $N_{c}B_{c}$, is the power (\textsc{watts} or \textsc{joules/sec}) required to maintain the total existing biomass of the organsim, denoted background maintenance, a resting metabolic process. The second is the power allocated to the production of new cells and, therefore, to growth. $E_c$, $B_c$, and the mass of a cell, $m_c$ (\textsc{grams}), are assumed to be independent of $m$ remaining constant throughout ontogenetic growth, i.e. these quantities do not vary with size. \cref{master_balance} can be rearranged to find the rate of cell growth at time $t$: 
\begin{align}
    \frac{dN_{c}}{dt} &= \frac{B - N_{c}B_{c}}{E_{c}} \label{cell_rate}
\end{align}
This rearrangement reveals a crucial assumption of the model which is that growth of new cells is derived from \textit{surplus} metabolic energy remaining after cellular processes have been addressed ($N_{c}B_{c}$). That is, these processes are non-negotiable:
\begin{align*}    
    \frac{dN_{c}}{dt} &= \frac{\textsc{Surplus energy}}{\textsc{Cost of creating single cell}}
\end{align*}
At any time $t$ the total body mass $m$ is the number of cells multiplied by the mass of a single cell, $m = N_{c}m_{c}$. Thus, we can multiply equation \cref{cell_rate} by $m_{c}$ to find the rate of whole-organism growth (change in organism mass $m$) at time $t$:
\begin{align*}
    \Bigg(\frac{dN_{c}}{dt}\Bigg)m_c &= \Bigg(\frac{B - N_{c}B_{c}}{E_{c}}\Bigg)m_c \\
    \Bigg(\frac{dN_{c}}{dt}\Bigg)m_c &= \Bigg(\frac{Bm_c - N_{c}m_{c}B_{c}}{E_c}\Bigg) \\
    \textsc{substitute } m &= N_{c}m_{c} \\
    \frac{dm}{dt} &= \Bigg(\frac{Bm_c - mB_{c}}{E_c}\Bigg) \\
    \frac{dm}{dt} &= \Bigg(\frac{Bm_c}{E_c}\Bigg) - \Bigg(\frac{mB_{c}}{E_c}\Bigg) \\
    \frac{dm}{dt} &= \Bigg(\frac{m_c}{E_c}\Bigg)B - \Bigg(\frac{B_{c}}{E_c}\Bigg)m \numberthis \label{growth_pre_sub}
\end{align*}
We can define \cref{growth_pre_sub} qualitatively in terms of cellular-level phenomena. The first term is the mass of a single cell divided by the energy required to build a single cell, (\textsc{mass/joules}), i.e. how many individual cells can be built from a single \textsc{joule} of energy. This is multiplied by the energy inflow ($B$) (\textsc{joules/sec}), thus \textsc{joules} units cancel out, left with only \textsc{mass/time}. Qualitatively this tells us: ``how much mass can I build per second given the energy ($B$) I have available per second.'' If maintenance (second term) did not exist, this first term would represent all growth. The second term represents the metabolic rate of single cell (\textsc{joules/sec}) divided by the energetic cost of building a cell (\textsc{joules}), which is all multiplied by mass, thus units are (\textsc{mass/time}).
\begin{align*}
    \frac{\textsc{mass}}{\textsc{sec}} &= \frac{\textsc{mass}}{\textsc{joules}}\cdot\frac{\textsc{joules}}{\textsc{sec}} - \frac{\frac{\textsc{joules}}{\textsc{sec}}}{\textsc{joules}}\cdot\textsc{mass} \\
    \frac{\textsc{mass}}{\textsc{sec}} &= \frac{\textsc{mass}}{\textsc{\st{joules}}}\cdot\frac{\textsc{\st{joules}}}{\textsc{sec}} - \frac{\frac{\textsc{\st{joules}}}{\textsc{sec}}}{\textsc{\st{joules}}}\cdot\textsc{mass} \\
    \frac{\textsc{mass}}{\textsc{sec}} &= \frac{\textsc{mass}}{\textsc{sec}} - \frac{\textsc{mass}}{\textsc{sec}} \\
\end{align*}
%     \frac{dm}{dt} &= \frac{m_c}{E_c}B_{0}m^{3/4} - \frac{B_{c}}{E_c}m
% \end{align}
Now, Ecological Metabolic Theory assumes that $B$, the resting metabolic rate of the organism at time $t$, scales with mass as $B = B_{0}m^{3/4}$, where $B_0$ is constant within a taxon \autocite{West1997,brown2000-scaling-book, Brown2004}. Substituting this relationship into \cref{growth_pre_sub} leads to the general growth equation.
\begin{align*}
    \frac{dm}{dt} &= \bm{\Bigg(\frac{m_c}{E_c}\Bigg)B_{0}}m^{3/4} - \bm{\Bigg(\frac{B_{c}}{E_c}\Bigg)}m \\
\end{align*}
Since $E_c$, $B_c$ and $m_c$ are independent of mass and $B_0$ is constant within a taxon, we can define two new constants: $a \equiv B_{0}m_{c}/E_{c}$ and $b \equiv B_{c}/E{c}$. Thus, \textcite{West2001} derive the following model for an organism of wet mass $m$ (g), a resting metabolic rate $b$ ($J \cdot d^{-1}$) and a growth rate per unit time of $dm/dt$ ($g \cdot d^{-1}$):
\begin{align*}
    \textsc{substitute }a, b
\end{align*}
\begin{equation}
    \frac{dm}{dt} = am^{3/4} - bm \label{west_model}
\end{equation}
This equation excludes reproduction and would result in sigmoid growth to an asymptotic size, $M$. We can find this asymptotic size in terms of $a$ and $b$ analytically since change in growth at terminal size, $\frac{dm}{dt} = 0$. Thus we set \cref{west_model} to 0 as follows:
\begin{align*}
    aM^{3/4} - bM &= 0 \\
    aM^{3/4} &= bM \\
    \frac{aM^{3/4}}{M} &= b \\
    aM^{-1/4} &= b \\
    \frac{a}{M^{1/4}} &= b \\
    M^{1/4} &= \frac{a}{b} \\
    M &= \Big(\frac{a}{b}\Big)^4
\end{align*}

\subsection{Derivation of intake rate coefficient $a$}
\textcite{West2001} make the following calculation: Assume incoming rate of energy, $B$, scales with standard metabolic rate, such that $B = B_{0}m^{3/4}$, where $B_0$ is constant for a given taxon. $E_T$ is the energy content of mammalian tissue $\approx 7 \cdot 10^6 J \cdot Kg^{-1}$ and $m_c$ is the mass of a single cell $\approx 3 \cdot 10^{-9} g$, thus $E_c$, the energy required to create a cell \textit{in viro} = $E_T \cdot m_c = (7 \cdot 10^6 JKg^{-1})(3 \cdot 10^{-12} Kg) \approx 2.1 \cdot 10^{-5} J$. They take  $B_0 \approx 1.9 \cdot 10^{-2} W$. By substitution:
\begin{align*}
    a &\equiv \frac{B_{0}m_{c}}{E_{c}} \\
    a &= \frac{(1.9 \cdot 10^{-2} W)(3 \cdot 10^{-9} g)}{2.1 \cdot 10^{-5} J}\\
    a &= 2.71 \cdot 10^{-6} g^{1/4}s^{-1} \\
    2.71 \cdot 10^{-6}&g\cdot s^{-1} \cdot 24 \cdot 60 \cdot 60 \ \ \textsc{   seconds to days}\\
    a &\approx 0.25 g^{1/4}day^{-1}
\end{align*}

Bioenergetically, $a$ represents the number of cells grown per unit time given the energy available ($\frac{\textsc{mass}}{\textsc{joules}} \cdot \frac{\textsc{joules}}{\textsc{sec}}$). They claim their theoretical calculation of $a$ is in ``good agreement'' with the model fit estimate for $a$ in their Table 1, but brief observation reveals that this claim only holds true for mammals, not birds and fish (fish are an order of magnitude lower), an intuitive deduction since they used the energy content of mammalian tissue in their calculation, taken from \textcite{peters1983}. However, further investigation makes apparent the intriguing value they use for $B_0$. This quantity remains constant within a taxon, and \textcite{West2001} chose the value known for unicellular organisms, derived from data published by \textcite{Hemmingsen1960}, collected by \textcite{peters1983}, which included different slope (scaling) values for unicellulars, poikilotherms and homeotherms. Given the deliberate choice to use a caloric equivalent value for mammals only, it is perplexing that they chose to ignore the $B_0$ value for homeotherms.

Given my objective of modelling growth and life history in fish, this choice is not only of particular interest but is crucial given that this endeavour mandates accurate values for fixed parameters, such as $a$, in order for free parameters to be biologically realistic. To this end I altered the \textcite{West2001} calculation of $a$ to include values for the same cellular parameters but which are accurate to fish. Tissue energy content $E_T$ for fish was calculated as the average from caloric equivalents across 70 fish species \autocite{Cummins1971, Steimle1980, Hartman1995} (\cref{caloric_equivalents}), converted from $kJ\cdot g^{-1}$ to $JKg^{-1}$ using ``per gram of wet weight'' values, the same metric used by \textcite{West2001}). Repeating the same calculation: $E_T$ is the energy content of fish tissue $= 5.63 \cdot 10^3 J\cdot g^{-1}$ and $m_c$ is the mass of a single cell $\approx 3 \cdot 10^{-9} \textsc{g}$, thus $E_c$, the energy required to create a cell \textit{in viro} = $1.69 \cdot 10^{-5} J$. $B_0 \approx 1.4 \cdot 10^{-1} W$ for poikilotherms. By substitution:
\begin{align*}
    a &\equiv \frac{B_{0}m_{c}}{E_{c}} \\
    a &= \frac{(1.4 \cdot 10^{-1} W)(3 \cdot 10^{-9} g)}{1.69 \cdot 10^{-5} J}\\
    a &= 2.4867 \cdot 10^{-5}g\cdot s^{-1} \\
    2.4867 \cdot 10^{-5}&g\cdot s^{-1} \cdot 24 \cdot 60 \cdot 60 \ \ \textsc{   seconds to days}\\
    a &= 2.1485 g^{1/4}\cdot day^{-1}
\end{align*}

Therefore intake rate of fish is expected to scale as $\approx 2.15g^{1/4}\cdot day^{-1}$ which I use in my parameterisation of the growth model.

\subsection{Introducing Reproduction}
A revised form of \cref{west_model} published by \textcite{Charnov2001}, incorporates an explicit fecundity rate term, given that gonad mass in fish is proportional to body mass \autocite{Charnov2001, Roff1983, peters1983, kozlowski1996}. Crucially, \textcite{Charnov2001} assumed that fecundity rate scales isometrically with size, which is mathematically equivalent to elevating maintenance cost, since the maintenance and reproduction term have equivalent mass-scaling exponents. Until maturity (age $\alpha$, size $m_{\alpha}$), fish incur costs only from maintenance as they are sexually immature, captured by \cref{west_model}. 
\begin{align*}
    \frac{dm}{dt} &= am^{3/4} - bm \ \ \ \ \ \ \ \ \ \ \ \ m < m_{\alpha} \\
    \frac{dm}{dt} &= am^{3/4} - bm - cm \ \ \ \ \ m \geq m_{\alpha} \numberthis \label{charnov2001} \\
    \frac{dm}{dt} &= am^{3/4} - (b+c)m
\end{align*}
where $c \cdot m$ is the reproductive allocation. Since maintenance and reproduction ($bm$ and $cm$) both scale linearly with mass, we can factorise the last two terms to predict that asymptotic size $M$ is now brought down. In this sense, the onset of reproduction can be viewed mathematically as an increase in maintenance cost since $(b+c)$ is a new constant in its own right:
\begin{align*}
    aM^{3/4} - (b+c)M &= 0 \\
    aM^{3/4} &= (b+c)M \\
    \frac{aM^{3/4}}{M} &= (b+c) \\
    aM^{-1/4} &= (b+c) \\
    \frac{a}{M^{1/4}} &= (b+c) \\
    M^{1/4} &= \frac{a}{(b+c)} \\
    M &= \Bigg(\frac{a}{(b+c)}\Bigg)^4
\end{align*}
This reduction highlights the cost to growth reproduction introduces, since there is less energetic scope for growth after $\alpha$. To this end, the earlier a fish matures, and the more fecund it is, its growth rate and asymptotic size will both reduce, reflecting both the timing ($\alpha$) and magnitude ($c\cdot{m}$) of reproduction.

\subsection{Assumptions}
An axiom of the growth model in \cref{west_model} proposed by \textcite{West2001} is that energy intake follows a simple relationship with resting metabolic rate, which scales with body size ($m$) to an exponent of $3/4$, as predicted by \textcite{West1997} and used in the Metabolic Theory of Ecology \autocite{Brown2004} regardless of taxon, resource-environment or dimensionality. The assumptions of resting metabolic rate data are that the study organism is under ``idealised conditions'' i.e. not foraging, reproducing or growing. Consequently, it is puzzling that \textcite{West2001} use a resting metabolic metric for energy intake, which is an active metabolic process and \textcite{Charnov2001} does the same when incorporating reproduction, another active metabolic process. It has been shown by \textcite{Pawar2012} that consumption rate scaling is, therefore, more likely to relate to field metabolic rate. Both scaling exponents are tested in this paper.


\section{Life History Optimisation}
\subsection{Invariant Values}
Across taxa, certain life history variables, representing the timing and magnitude of reproduction form, dimensionless, invariant ratios \autocite{Charnov1993}. These are particularly useful for comparative life histories as these numbers typically remain invariant within a given taxon, e.g. fish. I utilise these to parameterise the life history model below which was used in growth trajectory simulations. In fish, the quantity representing age at maturity multiplied by mortality rate, $\alpha\cdot Z \approx 2$ \autocite{Charnov2001}. I rearrange this to parameterise mortality rate $Z = 2/{\alpha}$.

\subsection{Lifetime Reproduction}
I assume that intake cannot be optimised since this is dependent on what the surrounding environment provides. However, optimisation of reproduction parameters invokes life history theory, which assumes that natural selection optimises strategies, or $c$ and $\rho$, to maximise fitness, where lifetime reproductive output can be used as a proxy for fitness, denoted $R_0$, which can be derived from theoretical evolution studies \autocite{Charnov2001, stearns1992evolution}.

At any time $t$, $b_{t}$ is the \textit{effective} energy allocated by fish to reproduction, the product of the physiological allocation of resources, $cm^{\rho}$ and an efficiency term $h(m)$ representing a declining efficiency of this allocation, known as reproductive senescence, the natural decline in fecundity as fish age \autocite{Stearns2000, Benoit2018, Vrtilek2018}. This decline begins at maturity ($\alpha$) and is controlled by a rate parameter $\kappa$. Fish also experience an extrinsic mortality rate, or actuarial senescence, contained in a surivorship function, $l_t$, which is effectively a declining $\mathbb{P}$(survival to t) \autocite{Charnov1993, Charnov2001, Benoit2018, Laird2010, Reznick2002, Reznick2006}. Thus, the instanteous reproductive output at time $t$ is the product $l_{t}b_{t}$ and the lifetime reproductive output is:
\begin{equation}
    R_{0} = \int_{\alpha}^{\infty}l_{t}b_{t} dt
\end{equation}
Since fish live in a juvenile and adult phase, they are subject to varying mortality rates through ontogeny. Juvenile mortality ($t_0 \rightarrow t_{\alpha}$) controls how many fish are recruited into the adult phase and, since it follows an exponential distribution, $l_t = e^{-Z(t)dt}$ bounded [0,1], it thus acts as a scaling factor, denoted $l_{\alpha}$, for the mature population ($t_{\alpha} \rightarrow t_{\infty}$). For adults survival is \textbf{relative to when maturity is reached}, $l_{t} = e^{-Z(t-\alpha)}$.
\begin{align*}
    R_{0} &= \int cm^{\rho}h(t) \cdot l_{t} dt \\
          &= \int cm^{\rho}(t)^{\rho} e^{-\kappa(t-\alpha)} L_{\alpha}e^{-Z(t-\alpha)} dt \\
          &= l_{\alpha}\int cm^{\rho}(t)^{\rho} e^{-\kappa(t-\alpha)} \cdot e^{-Z(t-\alpha)} dt \\
          &= cl_{\alpha}\int m(t)^{\rho} e^{-(\kappa+Z)(t-\alpha)} dt \\
          &= c\int_{0}^{\alpha}e^{-Z(t)}dt\int_{\alpha}^{\infty} m(t)^{\rho} e^{-(\kappa+Z)(t-\alpha)} dt \numberthis \label{LHT_optimisation}
\end{align*}
To perform this optimisation, the analytical goal is to solve \cref{LHT_optimisation} for values of $c$ and $\rho$ which maximise $R_0$. Since it has no closed-form solution, I simulated this numerically using the \texttt{DifferentialEquations} and \texttt{DiffEqCallbacks} packages in Julia v1.1.1 \autocite{Bezanson2017}, which ran the Rosenbrock optimisation function \autocite{Rosenbrock1960}. The following parameter space was simulated $0.001 < c < 0.4$ and $0.001 < \rho < 1.25$ with 100 linearly-spaced values over a lifespan of $1e6$ days.

\newpage
\begin{landscape}
\pagestyle{empty}
\section{Fish Caloric Equivalents Data}
\begin{longtable}[]{|l|p{1.8cm}|p{2cm}|p{2cm}|p{2.15cm}|p{1.9cm}|l|}
    \caption{Caloric equivalents data digitised from \textcite{Steimle1980} and \textcite{Cummins1971}.} \\
    \hline
    \textbf{Species}                         & \textbf{No. combustions} & \textbf{Dry weight \% ash} & \textbf{Dry weight KJ/g}     & \textbf{Dry Weight ash-free KJ/g}         & \textbf{Wet weight KJ/g} & \textbf{Source}           \\ \hline
    \textit{Acanthocybium solanderi}         & 6                            & 6                          & 23.4                          & 24.9                              &                          & Steimle, 1980                \\ \hline
    \textit{Alosa aestivalis}                & 15                           & 12                         & 22.2                          & 25                                & 7.7                      & Steimle, 1980                \\ \hline
    \textit{Alosa mediocris}                 & 5                            & 11                         & 21.1                          & 23.8                              & 5.5                      & Steimle, 1980                \\ \hline
    \textit{Alosa pseudoharengus}            & 24                           & 12                         & 21.7                          & 24.5                              & 6.4                      & Steimle, 1980                \\ \hline
    \textit{Alosa sapidissima}               & 5                            & 11                         & 19.6                          & 22.1                              & 4.9                      & Steimle, 1980                \\ \hline
    \textit{Ammodytes americanus}            & 20                           & 12                         & 21.7                          & 24.8                              & 6.8                      & Steimle, 1980                \\ \hline
    \textit{Anchoa hepsetus}                 & 24                           & 16                         & 19.9                          & 23.6                              & 5.8                      & Steimle, 1980                \\ \hline
    \textit{Anchoa mitchilli}                & 5                            & 14                         & 21.2                          & 24.6                              & 5.9                      & Steimle, 1980                \\ \hline
    \textit{Brevoortia tyrannus}             & 10                           & 13                         & 21.4                          & 24.6                              & 7.5                      & Steimle, 1980                \\ \hline
    \textit{Caranx chrysos}                  & 5                            & 16                         & 17.7                          & 21.1                              & 5.7                      & Steimle, 1980                \\ \hline
    \textit{Chloroscombrus chrysurus}        & 5                            & 14                         & 23.9                          & 27.9                              & 8.8                      & Steimle, 1980                \\ \hline
    \textit{Citharichthys arctifrons}        & 5                            & 14                         & 19.4                          & 22.5                              & 4.3                      & Steimle, 1980                \\ \hline
    \textit{Clupea harengus}                 &                              &                            & 26.61024                      &                                   & 8.062568                 & Cummins, 1971                \\ \hline
    \textit{Clupea harengus}                 & 20                           & 8                          & 25.1                          & 27.2                              & 10.6                     & Steimle, 1980                \\ \hline
    \textit{Conger oceanicus}                & 6                            & 8                          & 29.8                          & 32.4                              & 9.8                      & Steimle, 1980                \\ \hline
    \textit{Cynoscion regalis}               & 4                            & 13                         & 20.1                          & 23                                & 4.8                      & Steimle, 1980                \\ \hline
    \textit{Enchelyopus cimbrius}            & 5                            & 16                         & 18                            & 21.6                              & 3.6                      & Steimle, 1980                \\ \hline
    \textit{Etrumeus teres}                  & 17                           & 14                         & 20.4                          & 23.6                              & 5.5                      & Steimle, 1980                \\ \hline
    \textit{Gadus morhua}                    & 9                            & 16                         & 18.2                          & 21.6                              & 4.2                      & Steimle, 1980                \\ \hline
    \textit{Hemitripterus americanus}        & 6                            & 16                         & 18.1                          & 21.5                              & 2.5                      & Steimle, 1980                \\ \hline
    \textit{Hippoglossoides platessoides}    & 3                            & 15                         & 17.7                          & 21                                & 4.1                      & Steimle, 1980                \\ \hline
    \textit{Leiostomus xanthurus}            & 11                           & 17                         & 20.1                          & 24                                & 7                        & Steimle, 1980                \\ \hline
    \textit{Limanda ferruginea}              & 12                           & 16                         & 17.6                          & 20.9                              & 4.4                      & Steimle, 1980                \\ \hline
    \textit{Liopsetta putnami}               & 5                            & 17                         & 17.1                          & 20.7                              & 3.8                      & Steimle, 1980                \\ \hline
    \textit{Lophius americanus}              & 6                            & 13                         & 18.5                          & 21.3                              & 1.7                      & Steimle, 1980                \\ \hline
    \textit{Lumpenus maculatus}              & 8                            & 12                         & 20.2                          & 22.2                              & 5.6                      & Steimle, 1980                \\ \hline
    \textit{Macrozoarces americanus}         & 9                            & 13                         & 19.2                          & 22                                & 4.7                      & Steimle, 1980                \\ \hline
    \textit{Melanogrammus aeglefinus}        & 4                            & 17                         & 20.3                          & 24.3                              & 4.5                      & Steimle, 1980                \\ \hline
    \textit{Menidia menidia}                 & 10                           & 13                         & 21.2                          & 24.4                              & 7.3                      & Steimle, 1980                \\ \hline
    \textit{Merluccius bilinearis}           & 10                           & 13                         & 21.3                          & 24.6                              & 4.6                      & Steimle, 1980                \\ \hline
    \textit{Myoxocephalus octodecemspinosus} & 5                            & 17                         & 20.8                          & 25.1                              & 5.4                      & Steimle, 1980                \\ \hline
    \textit{Oecapterus punctatus}            & 5                            & 13                         & 18.8                          & 21.6                              & 6                        & Steimle, 1980                \\ \hline
    \textit{Oncorhynchus garbusha}           &                              &                            & 15.058216                     &                                   & 6.54796                  & Cummins, 1971                 \\ \hline
    \textit{Oncorhynchus garbusha}           &                              &                            & 16.915912                     &                                   & 7.058408                 & Cummins, 1971                 \\ \hline
    \textit{Oncorhynchus keta}               &                              &                            & 15.087504                     &                                   & 6.681848                 & Cummins, 1971                 \\ \hline
    \textit{Oncorhynchus kisutch}            &                              &                            & 14.418064                     &                                   & 5.778104                 & Cummins, 1971                 \\ \hline
    \textit{Oncorhynchus masu}               &                              &                            & 14.493376                     &                                   & 6.552144                 & Cummins, 1971                 \\ \hline
    \textit{Oncorhynchus masu}               &                              &                            & 15.761128                     &                                   & 7.192296                 & Cummins, 1971                 \\ \hline
    \textit{Oncorhynchus nerka}              &                              &                            & 13.99548                      &                                   & 5.405728                 & Cummins, 1971                 \\ \hline
    \textit{Oncorhynchus nerka}              &                              &                            & 14.418064                     &                                   & 5.727896                 & Cummins, 1971                 \\ \hline
    \textit{Oncorhynchus tschawytscha}       &                              &                            & 15.267416                     &                                   & 5.702792                 & Cummins, 1971                 \\ \hline
    \textit{Ophichthus cruentifier}          & 4                            & 14                         & 19.8                          & 22.9                              & 5.8                      & Steimle, 1980                \\ \hline
    \textit{Opisthonema oglinum}             & 5                            & 11                         & 24.8                          & 27.7                              & 6.4                      & Steimle, 1980                \\ \hline
    \textit{Paralichthys oblongus}           & 10                           & 13                         & 22                            & 25.4                              & 5.5                      & Steimle, 1980                \\ \hline
    \textit{Peprilus alepidotus}             & 5                            & 11                         & 22.9                          & 25.8                              & 6.2                      & Steimle, 1980                \\ \hline
    \textit{Peprilus triacanthus}            & 25                           & 11                         & 24.2                          & 27.2                              & 6.2                      & Steimle, 1980                \\ \hline
    \textit{Pollachius virens}               & 5                            & 13                         & 19.2                          & 21.9                              & 5                        & Steimle, 1980                \\ \hline
    \textit{Pomatomus saltatrix}             & 5                            & 20                         & 20.8                          & 25                                & 4.8                      & Steimle, 1980                \\ \hline
    \textit{Prionotus carolinus}             & 5                            & 22                         & 17.7                          & 22.7                              & 4.4                      & Steimle, 1980                \\ \hline
    \textit{Pseudopleuronectes americanus}   & 5                            & 15                         & 16.4                          & 19.4                              & 3.6                      & Steimle, 1980                \\ \hline
    \textit{Salmo salar}                     &                              &                            & 14.882488                     &                                   & 5.83668                  & Cummins, 1971 \\ \hline
    \textit{Salmo salar}                     &                              &                            & 15.31344                      &                                   & 6.163032                 & Cummins, 1971 \\ \hline
    \textit{Sardinella aurita}               & 5                            & 14                         & 19.9                          & 23.1                              & 6                        & Steimle, 1980                \\ \hline
    \textit{Scomber japonicus}               & 15                           & 12                         & 21.6                          & 24.4                              & 6.2                      & Steimle, 1980                \\ \hline
    \textit{Scomberesox saurus}              & 5                            & 9                          & 22.3                          & 24.6                              & 8.5                      & Steimle, 1980                \\ \hline
    \textit{Scophthalmus aquosus}            & 3                            & 13                         & 22.2                          & 25.4                              & 3.1                      & Steimle, 1980                \\ \hline
    \textit{Scotnber scombrus}               & 28                           & 8                          & 24.1                          & 26.2                              & 6                        & Steimle, 1980                \\ \hline
    \textit{Sebastes marinus}                & 15                           & 20                         & 18.5                          & 23.1                              & 4.4                      & Steimle, 1980                \\ \hline
    \textit{Selar crumenopthalmus}           & 5                            & 14                         & 16.4                          & 19                                & 4.9                      & Steimle, 1980                \\ \hline
    \textit{Seriola dumerili}                & 3                            & 15                         & 18.4                          & 21.5                              & 4.6                      & Steimle, 1980                \\ \hline
    \textit{Sphoeroides maculatus}           & 4                            & 15                         & 19.4                          & 22.8                              & 5.1                      & Steimle, 1980                \\ \hline
    \textit{Stenotomus chrysops}             & 14                           & 16                         & 21.5                          & 25.6                              & 6.1                      & Steimle, 1980                \\ \hline
    \textit{Symphurus plagiusa}              & 4                            & 16                         & 19                            & 22.7                              & 6.3                      & Steimle, 1980                \\ \hline
    \textit{Synodus foetens}                 & 5                            & 17                         & 18.3                          & 22.1                              & 4.6                      & Steimle, 1980                \\ \hline
    \textit{Tautogolabrus adspersus}         &                              &                            & 20.41792                      &                                   & 4.426672                 & Cummins, 1971                \\ \hline
    \textit{Tautogolabrus adspersus}         & 5                            & 11                         & 22.2                          & 24.9                              & 6.6                      & Steimle, 1980                \\ \hline
    \textit{Thunnus albacares}               & 6                            &                            & 23.4                          &                                   &                          & Steimle, 1980                \\ \hline
    \textit{Triglops murrayi}                & 5                            & 16                         & 18.1                          & 21.5                              & 3.6                      & Steimle, 1980                \\ \hline
    \textit{Triglops nybelini}               & 5                            & 14                         & 19.4                          & 22.5                              & 5.2                      & Steimle, 1980                \\ \hline
    \textit{Urophycis chuss}                 & 10                           & 14                         & 19.4                          & 22.6                              & 3.8                      & Steimle, 1980                \\ \hline
    \textit{Urophycis regia}                 & 3                            & 10                         & 21.3                          & 23.6                              & 4.7                      & Steimle, 1980                \\ \hline
    \textit{Urophycis tenuis}                & 3                            & 14                         & 19.2                          & 22.3                              & 6.3                      & Steimle, 1980                \\ \hline
    \textit{Xiphias gladius}                 & 59                           & 4                          & 27.5                          & 28.8                              &                          & Steimle, 1980                \\ \hline
\end{longtable}
\pagestyle{plain}
\end{landscape}

\newpage
%%% Table of literature %%%
\section{Ontogenetic Growth Model Literature}

\begin{table}[H]
    \caption{Published literature on Ontogenetic Growth Models including discipline of origin and reproductive scaling. Adapted from \textcite{Barneche2018-reproductive_output}.}
    \begin{tabular}{|l|l|l|l|}
    \hline
    \textbf{Study}                      & \textbf{Model type}       & \textbf{Scaling} & \textbf{Prediction/Assumption} \\ \hline
    \textcite{Gadgil1970}                   & Life history              & Hyperallometric  & Prediction                     \\ \hline
    \textcite{Roff1983}                     & Life history              & Isometric        & Assumption                     \\ \hline
    \textcite{Roff1984}                     & Life history              & Isometric        & Assumption                     \\ \hline
    \textcite{Reiss1985}                    & Mechanistic               & Hypoallometric   & Prediction                     \\ \hline
    \textcite{Kozowski1987-indeterminate}   & Life history              & Variable         & Prediction                     \\ \hline
    \textcite{kozlowski1996}                & Life history              & Variable         & Prediction                     \\ \hline
    \textcite{West2001}                     & Mechanistic               & Isometric        & Assumption                     \\ \hline
    \textcite{Charnov2001}                  & Hybrid                    & Isometric        & Assumption                     \\ \hline
    \textcite{Charnov2002}                  & Hybrid                    & Isometric        & Assumption                     \\ \hline
    \textcite{Lester2004}                   & Life history - Biphasic   & Isometric*       & Assumption                     \\ \hline
    \textcite{Roff2006}                     & Life history              & Isometric        & Assumption                     \\ \hline
    \textcite{Quince2008}                   & Life history - Biphasic   & Isometric*       & Assumption                     \\ \hline
    \textcite{Quince2008b}                  & Life history - Biphasic   & Hyperallometric* & Prediction                     \\ \hline
    \textcite{Pecquerie2009}                & Life history              & Isometric        & Assumption                     \\ \hline
    \textcite{kooijman2010dynamic}          & Mechanistic               & Isometric*       & Assumption                     \\ \hline
    \textcite{Arendt2011}                   & Life history              & Isometry         & Assumption                     \\ \hline
    \textcite{Ohnishi2011}                  & Life history - Biphasic   & Hyperallometric  & Assumption                     \\ \hline
    \textcite{Brunel2013}                   & Mechanistic               & Isometric        & Assumption                     \\ \hline
    \textcite{Charnov2013}                  & Hybrid                    & Isometric        & Assumption                     \\ \hline
    \textcite{Boukal2014}                   & Life history - Biphasic   & Isometric*       & Assumption                     \\ \hline
    \textcite{Kooijman2014a}                & Mechanistic               & Isometric*       & Assumption                     \\ \hline
    \textcite{Minte-Vera2016a}              & Life history - Biphasic   & Isometric*       & Assumption                     \\ \hline
    \textcite{Jusup2017}                    & Mechanistic               & Isometric        & Assumption                     \\ \hline
    \textcite{Mangel2017}                   & Life history              & Hyperallometric  & Assumption                     \\ \hline
    \textcite{Smallegange2017}              & Mechanistic               & Isometric        & Assumption                     \\ \hline
    \textcite{Audzijonyte2018}              & Hybrid                    & Isometric        & Assumption                     \\ \hline
    Beverton and Holt (1957)            & Fisheries                 & Isometric        & Assumption                     \\ \hline
    Scott \textit{et al}. (2006)        & Fisheries                 & Hyperallometric  & Assumption                     \\ \hline
    Jørgensen and Fisken (2006)         & Fisheries                 & Isometric*       & Assumption                     \\ \hline
    Enberg \textit{et al}. (2010)       & Fisheries                 & Isometric        & Assumption                     \\ \hline
    Eikeset \textit{et al}. (2013)      & Fisheries                 & Isometric        & Assumption                     \\ \hline
    Lester \textit{et al}. (2014)       & Fisheries                 & Isometric        & Assumption                     \\ \hline
    Andersen and Beyer (2015)           & Fisheries                 & Isometric        & Assumption                     \\ \hline
    Eikeset \textit{et al}. (2016)      & Fisheries                 & Isometric        & Assumption                     \\ \hline
    Andersen \textit{et al}. (2016)     & Fisheries                 & Isometric        & Assumption                     \\ \hline
    Zimmerman and Jørgensen (2016)      & Fisheries                 & Isometric        & Assumption                     \\ \hline
    Hartvig \textit{et al}. (2011)      & Food web                  & Isometric        & Assumption                     \\ \hline
    Carozza \textit{et al}. (2016)      & Food web                  & Isometric        & Assumption                     \\ \hline
    \end{tabular}
    \label{caloric_equivalents}
\end{table}

%%% Bibliography %%%
\newpage\addcontentsline{toc}{section}{SI References}

\let\mkbibnamefamily\textsc\printbibliography[title=SI References]\thispagestyle{empty} % Sets author names to small caps

\end{document}