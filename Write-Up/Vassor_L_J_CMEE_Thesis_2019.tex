\documentclass[a4paper]{article} % twoside for paper submission but remove for electronic submission
\title{Reconciling Resource Supply and Hyperallometric Fecundity Scaling in Ontogenetic Growth Models}
\author{Luke Joseph Vassor}

% Packages
% Page Formatting packages
\usepackage[margin=2cm]{geometry}
\usepackage{lipsum} % Generates dummy text
\usepackage{titlesec}
    \titleformat*{\section}{\LARGE\bfseries}
    \titleformat*{\subsection}{\Large\bfseries}
    \titleformat*{\subsubsection}{\large\bfseries}
    \titleformat*{\paragraph}{\large\bfseries}
    \titleformat*{\subparagraph}{\large\bfseries}
    \newcommand{\sectionbreak}{\newpage} % forces new page for each section
    \titlespacing*{\section}
{0pt}{20cm plus 1ex minus .2ex}{4.3ex plus .2ex}
\usepackage{fancyhdr} % rule line and current section displayed at top of pages % \thispagestyle{empty} removes it on desired pages
    \pagestyle{fancy}
    \renewcommand{\sectionmark}[1]{\markboth{#1}{}} % set the \leftmark to section name
    \fancyhead[L]{\nouppercase{\leftmark}} %  sets the left head element ONLY EVER section name, otherwise mark is section/subsection/subsubsection
    \fancyhead[R]{} % sets the right head element to the page number if you use \thepage as argument
    \fancyfoot[C]{\thepage} % sets footer as page number

% Math packages
\usepackage{amsmath}

% Language packages
% \usepackage[english]{babel}
\usepackage[utf8]{inputenc}    % utf8 support       %!!!!!!!!!!!!!!!!!!!!
\usepackage[T1]{fontenc}       % code for pdf file  %!!!!!!!!!!!!!!!!!!!!
% \usepackage{listings}

% Other packages
\usepackage{hyperref}
    \hypersetup{colorlinks=false}
% \usepackage[colorinlistoftodos]{todonotes}

% Plotting packages
% \usepackage{lscape}
% \usepackage{subfigure}
\usepackage{graphicx}

% Citation packages
\usepackage[maxcitenames=2, backend=biber, style=imperialharvard]{biblatex}
    \addbibresource{./CMEE_Thesis.bib}
    % \renewcommand*\finalnamedelim{\addspace\&\space}
% Force \textit{et al}. to be italicised
\usepackage{xpatch}
    \xpatchbibmacro{name:andothers}{%
        \bibstring{andothers}%
    }{%
    \bibstring[\emph]{andothers}%
    }
    \xpatchbibmacro{Name}{%
        \bibstring{name}
    }{%
    \bibstring[\textbf]{name}%
    }{}{} % leave these empty arguments here, seem to cause issues when ommited

% Kill "Accessed on" lines in bibliography
\AtEveryBibitem{
    \clearfield{urlyear}
    \clearfield{urlmonth}
    \clearfield{doi}
    \clearfield{url}
    \clearfield{eprint}
    \clearfield{eprinttype}
}

% Add numbers to bibliography
% \DeclareFieldFormat{labelnumberwidth}{[#1]}
% \defbibenvironment{bibliography}  % from numeric.bbx
% {\list
%     {\printtext[labelnumberwidth]{%
%             \printfield{prefixnumber}%
%             \printfield{labelnumber}}}
%     {\setlength{\labelwidth}{\labelnumberwidth}%
%         \setlength{\leftmargin}{\labelwidth}%
%         \setlength{\labelsep}{\biblabelsep}%
%         \addtolength{\leftmargin}{\labelsep}%
%         \setlength{\itemsep}{\bibitemsep}%
%         \setlength{\parsep}{\bibparsep}}%
%     \renewcommand*{\makelabel}[1]{\hss##1}}
% {\endlist}
% {\item}

\begin{document}

\begin{titlepage}
    % Square brackets control size of vertical gap AFTER group
    %----------------------------------------------------------------------------------------
    %	LOGO SECTION
    %----------------------------------------------------------------------------------------
    
    \includegraphics[width=4cm]{./Images/logo.png}\\%[1cm] % Include a department/university logo - this will require the graphicx package
     
    %----------------------------------------------------------------------------------------
    
    \center % Center everything on the page
    
    
    %----------------------------------------------------------------------------------------
    %	TITLE SECTION
    %----------------------------------------------------------------------------------------
    \makeatletter
    \linespread{1.5} % controls line spacing of title
        {\huge{RECONCILING RESOURCE SUPPLY AND HYPERALLOMETRIC FECUNDITY SCALING IN ONTOGENETIC GROWTH MODELS}\par} % leave \par here
    \vspace{2.5cm} % Title of your document

    %----------------------------------------------------------------------------------------
    %	DESCRIPTION SECTION
    %----------------------------------------------------------------------------------------
    % If you want consistent line spacing then need to typeset paragraphy as a single {} group and use \\ for line break
    % Strangely, you need to leave the \ character at the end of the group so that the spacing of the last line from the penultimate line in a group remains constant
    % textsc command sets small capitals (sc)
    \textsc{A thesis submitted in partial fulfilment \\ of the requirements for the degree of \\ Master of Science at Imperial College London \\ by \\ \ }\\[2.5cm]
    \textsc{\Large \@author}\\[2.5cm]
    \textsc{Submitted for the \\ M.Sc. Computational Methods in Ecology \& Evolution \\ Formatted in the journal style of the Potato Journal \\ \ }\\[2cm]

    %----------------------------------------------------------------------------------------
    %	HEADING SECTIONS
    %----------------------------------------------------------------------------------------
    \textsc{Department of Life Sciences \\ Imperial College London \\ \ }\\[1cm]
    \textsc{\today}\\[2cm] % Date, change the \today to a set date if you want to be precise
    
    \vfill % Fill the rest of the page with whitespace
    
\end{titlepage}

%%% Declaration %%%
\section*{Declaration of Originality}\thispagestyle{empty}
    Declaration: The first page inside the cover must provide a brief declaration of the contributions
    made by you and by others to your project. Key points to address are:
    \begin{itemize}
        \item Was the data provided to you or did you collect or assemble it?
        \item Were you responsible for data processing or cleaning, if required?
        \item Were any mathematical models developed by you or by your supervisor?
        \item What role, if any, did your supervisor play in developing the analyses presented?
    \end{itemize}
    I certify that this thesis, and the research to which it refers, are the product of my own work, conducted during the current year of the \emph{M.Sc. Computational Methods in Ecology \& Evolution} at Imperial College London. Any ideas or quotations from the work of other people, published or otherwise, or from my own previous work are fully acknowledged in accordance with the standard referencing practices of the discipline.
    \vspace{3cm}
    \begin{flushright}
        Luke Joseph Vassor \\
        \today
    \end{flushright}

%%% Abstract %%%
\section*{Abstract}\thispagestyle{empty}
    Your abstract goes here. The abstract is a very brief summary of the dissertation's contents. It should be about half a page long. Somebody unfamiliar with your project should have a good idea of what it's about having read the abstract alone and will know whether it will be of interest to them.

%%% Acknowledgements %%%
\section*{Acknowledgements}\thispagestyle{empty}
    It is usual to thank those individuals who have provided particularly useful assistance, technical or otherwise, during your project. \\
    Samraat \\
    Dustin - useful conversations \\
    Tom - ditto \\
    Francis - code help \\
    Pawarlab - conversations and environment \\
    Alex - maths \\
    Course members \\

%%% Table of Contents %%%
\newpage\tableofcontents\thispagestyle{empty}

%%% List of Figures %%%
\newpage\listoffigures\thispagestyle{empty}
\addcontentsline{toc}{section}{List of Figures}

%%% Listof Tables %%%
\newpage\listoftables\thispagestyle{empty}
\addcontentsline{toc}{section}{List of Tables}

%%% Notation %%%
\newpage\section*{Notation}\thispagestyle{empty}
\addcontentsline{toc}{section}{Notation}
\begin{itemize}    
    \item $\frac{dm}{dt}$ = change in mass with (continuous) time
    \item $m$ = mass
    \item $a$ = coefficient (proportion of body mass) for energy/resource intake/acquisition
    \item $b$ = coefficient for maintenance
    \item $c$ = coefficient for reproduction
    \item $y$ = mass scaling exponent for energy intake (almost always = 0.75 (MTE))
    \item $z$ = mass scaling exponent for maintenance (almost always = 1)
    \item $\rho$ = mass scaling exponent for reproduction
    \item $\alpha$ = age at maturity (onset of reproduction)
    \item $m_{\alpha}$ = mass at maturity
    \item $M$ = asymptotic/terminal size
    \item $L(x)$ or $L_x$ = Probability of survival to age $x$
    \item ATR = allocation to reproduction
\end{itemize}

%%% Introduction %%%
\vspace*{250px}
\section{Introduction}\thispagestyle{empty}
    \subsection{Plan}
        Hourglass structure
        \begin{itemize}
            \item Why model at all? Brief Philosophy of modelling - Dustin's paper - Mechanistic and phenomenological
            \item Migration of Biology to theory as a discipline - use miniproject WU
            \item Why model growth? What's the point? 
            \item What can we gain from doing it successfully? 
            \item How long have scientists tried?
            \item What obstacles have they encountered?
            \item Detail the broad approaches - Top-down and bottom-up (Dustin's paper) - Examples of each - compare contrast
            \item Is it possible to reconcile the two?
            \item Why do the others fail to recover reproductive hyperallometry?
        \end{itemize}
    
    \subsection{Lit review}
        For over a century, biologists have attempted to understand ontogenetic growth. That is, the pattern of the growth of an individual throughout its developmental lifetime. It has peaked the interests of scientists across the spectrum of Biology, including theorists, field ecologists and applied scientists, in the hope of answering questions about growth. evolutionary life history theorists, game theorists, metabolic theorists, optimal foraging theorists and fisheries scientists. Why do organisms grow at the rate they grow? Why do they stop growing? What causes them to stop growing? Can they control growth? Why would they control growth? What is the optimal size to grow to? When is the optimal age to mature? All valid questions which many would argue require a theoretical understanding of the constraints on growth and how evolution has worked with these constraints to produce the growth patters we see in organisms.

        The recent migration of biology towards that of a theoretical discipline has seen mathematical biologists attempt to capture growth in the form of sophisticated mathematical equations. In a rapidly changing world, scientific theory will allow us to pierce through the noise in the data we collect and reduce the uncertainty in how the world is constructed and functions. The way in which we do this, traditionally, is to utilise human knowledge to eliminate improbable links from all logical possibilites and suggest the probable value or range of values for a dependent variable. The tools permitting this being quantifiable, mathematisable, testable ideas which are derived from underlying generic principles and built into a predictive framework, called a scientific theory \autocite{peters1983, West2011}. ADD POPPER 1967 and 1972. In order to accurately predict and understand biological phenomena, biology is undergoing a transition towards a discipline whose predictive power is increasingly derived from  biological theory. Mathematical biologists and physicists are now using laws from quantitative disciplines to accurately predict the value of dependent variables, such as laws of geometry \& scaling and laws of thermodynamics \& enzyme kinetics. Geometric laws have become of particular interest to those who seek to model growth, to develop theory and equations which relate an organism's characteristics to its body size. Most body size relations take the form of a power law:

        \begin{equation}
            Y = Y_0 M^b
        \end{equation}

        where $Y$ is the biological characteristic to be predicted, $M$ is animal body mass, and $Y_0$ and $b$ are empirically derived constants. The use of power laws in biology has a venerable history and these laws are writted an ``allometric" equations. If $b = 1$, the scaling is said to be ``isometric", while if $b \neq 1$, the relationship is called allometric, and plots as a curve on linear axes \autocite{brown2000-scaling-book}

        Concern with growth and body size relations is well-founded, especially given the decline of the state of the Earth's biosphere. Given this decline, never has it been so crucial for ecologists to make prudent, accurate predictions about what will happen to life on Earth and prioritise what life to conserve as resources become increasingly stretched and time is of the essence. Ontogenetic growth gives rise to different, ecologically-distinct stages across the lifetime of an organism. Knowledge and powerful theory which allows us to ascertain which stages are most crucial to an organism's conservation, such as ontogenetic growth models and (st)age-based models will allow us to direct resources to conserving not only organisms themselves, but the stages of ontogeny within those organisms which are most crucial to the species' survival. To quote \cite{Bartholomew1981} ``It is only a slight overestimate to say that the most important attribute of an animal, both physiologically and ecologically, is its size.'' The variety of size plays a central role in the fantastic spectrum of niches we can see across the surface of the planet, since most key physiological, ecological and evolutionary life history parameters covary with body size \autocite{peters1983, brown2000-scaling-book,schmidt1984scaling,Marshall2019b}. Larger organisms have lower mass-specific metabolic rates i.e. demand less energy per unit body mass, tend to live for longer and sleep for longer than smaller organisms. They have lower intrinsic rates of increase and population sizes and slower rates of evolution. 

        Ontogenetic growth, the rate of change of size per unit time, has see many quantitative biologists attempt to explain this by relating growth rate to fundamental life history events or to fundamental cellular processes. Ontogenetic growth models typically take the form of a ODE:

        \begin{equation}
            \frac{dm}{dt} = am^y - bm^z
        \end{equation}
                
        \subsection{Understanding Growth and Reproduction}
        Broadly, theory bifurcates into two major branches: mechanistic (``bottom-up'') and phenomenological (``top-down''). Both branches have seen many attempts to understand ontogentic changes in body size.  
        
        Evolutionary life history and game theorists, concerned with growth from a coarser-level, life history perspective, have typically employed optimisation techniques to solve for the optimal age and size values of given life history events, for example age-at-maturation or the onset of reproduction to maximise lifetime reproductive output, e.g. fecundity, given an age/time-dependent mortality constraint. Converseley, metabolic theorists, concerned with growth at a granular level have used laws from thermodynamics and enzyme kinetics as a first-principles approach to growth problems through the lens of Metabolic Theory and Dynamic Energy Budget (DEB). Under these paradigms, everything an organism does in its lifetime, including how it grows, is ultimately governed by the metabolic energy that flows into its system. 

        \subsubsection{Bottom-Up Mechanistic Approaches}
        The most venerable theory, which has since provided the foundation for many extant growth models to be built from, was the P\"{u}tter balance equation, now commonly associated with the von Bertalanffy growth function, which estimates the rate of increase in mass as the difference between anabolism and catabolism per unit time. \autocite{Putter1920, vonBert1938, VonBertalanffy1957,Marshall2019b}
        von Bertalanffy: ``there will be growth so long as building up prevails over breaking down; the organism reaches a steady state if and when both processes are equal. We may express this in a general formula: 
        \begin{equation}
            \frac{dW}{dt} = \eta W^m - \kappa W^n \text{\qquad ''}
        \end{equation}

        Through time, the mechanistic interpretation of the VBGF has varied, now broadly being considered as the difference in rates of energy assimilation and expenditure or the flow of energy in minus energy use for maintenance of existing tissues (metabolism). Implicit in these models is the assumption that surplus energy is optimally allocated to growth \textit{after} non-negotiable maintenance of existing cells. This is derived from rearranging the conservation of energy equation presented by \cite{West2001}:

        \begin{align}
            B &= \sum_c \Bigg[N_{c}B_{c} + E_{c}\frac{dN_{c}}{dt}\Bigg] \\
            \frac{dN_{c}}{dt} &= \frac{B - N_{c}B_{c}}{E_{c}} \\
            \text{Rate of new cells} &= \frac{\text{Surplus energy}}{\text{Cost of creating single cell}}
        \end{align}

        \begin{itemize}            
            \item $B$ = incoming rate of energy flow, which is the average resting metabolic rate of the whole organism at time $t$
            \item $B_c$ = the metabolic rate of a single cell
            \item $E_c$ = the metabolic energy required to create a cell
            \item $N_c$ = the total number of cells
            \item $\sum\limits_c$ = over all types of tissue, assuming a typical cell as the fundamental unit
        \end{itemize}
        
        The VGBF triggered a model-construction philosophy that focuses on ``bottom-up'' drivers of growth, typically based on fundamental laws previously established in other sciencies, such as thermodynamics and enzyme kinetics, combining biological knowledge, intuition and mathematics to explain observed ecological phenomena. Within this paradigm, body size is the end result of mechanistic constraints on resource supply and demand and the flow of free energy. This has given rise to well-known frameworks, such as Dynamic Energy Budget (DEB) theory, Metabolic Theory of Ecology (MTE) \& the Ontogenetic Growth Model (OGM) and Pauly's limiting gill surface model (LGSM). Although having different foci, each model is predicated on the assumption that the size scaling of resource acquisition is shallower than that of resource use, such that these two relationships eventually converge on a terminal size and growth ceases, when an organism is ``spending'' as much energy as it is acquiring. 
        
        What is puzzling, however, is that despite being an uncontroversially fundamental life history event, the VBGF, and the majority of models which have evolved from it do not consider reproduction. It has was indicated more than 20 years ago that mechanistic inferences made from the results of these models are problematic since they do not account for reproduction \autocite{Day1997, Marshall2019b}. Reproduction incurs large energetic costs and increases an organism's mortality risk, with some highly fecund vertebrates consisting of 75\% reproductive tissue \autocite{Parker2018}. Sexual maturity and the consequent onset of reproduction therefore sees different mechanistic dynamics play out relative to those in the preceding juvenile phase as resources are shunted from growth to reproduction \autocite{Day1997}. The lack of an explicit term in most mechanistic models to represent allocation of energy to reproduction consequently means they make a simple but critical assumption: that reproduction is proportionate to body size - in other words, reproduction energy output scales isometrically with size. These models assume that loss of energy to reproduction occurs from birth to death ($t_0 \rightarrow t_{\infty}$) as a part of energy loss to maintenance ($-bm^z$), which scales with mass to an exponent of 1, meaning reproduction remains a constant fraction of total body size throughout ontogeny.

        \subsubsection{Top-Down Life History Approaches}
        Life-history theory assumes fundamentally that evolution maximises an organism's reproductive output, and that strategies concerning the timing of life-history events are the product of optimising trade-offs among different traits which compete for limited resources \autocite{Day1997, Stearns1989, stearns1992evolution}.      

        In contrast to mechanistic approaches, life-history theory typically makes simplifying assumptions with regard to why certain biological traits scale with mass. \autocite{Day1997, Kozowski1987-indeterminate} For instance, while growth rate, resource acquisition ($am^y$), realised through consumption rate, and maintenance metabolism ($bm^z$) all scale with body size in particular ways \autocite{peters1983,Werner1988,brown2000-scaling-book}, phenomenological, life-history models typically remain agnostic as to \textit{why} such scaling occurs. The chain of causality in the scientific approach here is inherently reversed. The concern is not with how the mechanisms behind scaling laws explain ontogenetic growth, but instead with how evolution has selected for particular life-history strategies for reproduction and growth - and therefore size - \textit{given} that these scaling laws exist \autocite{Danko2017}. Mechanistic: Given these observed strategies, what are the mechanisms and how do they work. Phenom: Given these mechanisms exist, what is the best strategy (bayesian vs frequentist) ``From this perspective, life-history models are ‘top-down’ – they are driven by selection on body size, and the underlying physiology evolves in response to these selection pressures. Such phenomenological models of growth are good at predicting changes in size, based on variation in external context across populations or between closely related species (e.g., [34,35]).''
        
        Thus, size at maturity is predicted to be inversely related to mortality rate under a simple life-history model [30].

        \subsubsection{Hybrids}
        Whilst these two branches of theory tend to operate in isolation, several attempts have been made at combining them. Broadly classed as biphasic models (reviewed in \autocite{Wilson2018}), some account for the mechanistic drivers of growth pre- and post-maturity through seperate equations. 
        
        ''Similarly, links between DEB theory and integral projection models consider the demographic consequences of different physiologies [37]. These hybrid approaches seek to maximise their relevance and explanatory potential by drawing the different strengths of both the mechanistic and phenomenological schools of thought about how and why organisms grow. Thus, models of growth occur along a continuum, from exclusively mechanistic to exclusively phenomenological, but most tend to make assumptions about the costliness of reproduction.''
        
        \subsubsection{Uniphasic \& Biphasic}
        Uniphasic models (i.e. a single equation applying to the whole lifetime) which lack an explicit reproductive allocation term are guilty of making the misguided assumption that reproductive allocation scales isometrically with mass given that it falls under maintenance. Uniphasic models which include a specific reproductive term then assume that there is no switch of internal physiological dynamics at the onset of reproduction. And really, without accounting for allometric scaling of reproduction with mass (i.e. isometric) they essentially are commiting the same mathematical crime, since the maintenance and reproduction then share a common mass term which can be factored out. This was accounted for by \cite{Charnov2001}.

        Other have sought to account for inflections in growth due to a shifting of resources by fitting biphasic models 

        \subsubsection{A glitch in the matrix}

        The biological literature suggests that both branches typically assume that reproductive output scales isometrically with body size, e.g. a single 2kg female has the same reproductive output as two 1kg females. However, a recent meta-analysis of marine fishes covering 342 species and 15 orders has contradicted this assumption, with larger mothers reproducing disproportionately more for 95\% of species, i.e. reproductive hyperallometry whereby a single 2kg female reproduces more than two 1kg females \autocite{Barneche2018d}. It has since been argued that reproductive hyperallometry profoundly challenges mechanistic theories of growth, which should be revised accordingly, and that this hyperallometry drives growth trajectories in ways largely unanticipated by current theories PLAG \autocite{Marshall2019b}. In their opinion paper, \cite{Marshall2019b} correctly proclaim that these recent results beg two questions:
        \begin{enumerate}
            \item How reasonable is an assumption of isometric reproductive output (isometric reproductive scaling)?
            \item If reproduction is not isometric, how does this alter our understanding of growth?
        \end{enumerate}


        The most familiar example of allometry is simple geometric scaling. If we have spheres or any objects of self-similar shapes, we can describe changes in surface area, $A$, or volume, $V$, as a function of a linear dimension, the radius, $r$, as follows: $A = \pi r^2$ and $V = \frac{4}{3}\pi r^3$. And if the objects can maintain a constant density as they vary in size then $M \propto V$, and we can express their linear dimensions, $l$, or surface areas as functions of their mass
        \begin{equation}
            l = c_{1}M^{1/3} \; \text{and} \; c_{2}M^{2/3}
        \end{equation}
        where the values of the normalisation constants $c_1$ and $c_2$ depend on the units of measurement. Since these same equations apply to any shape, if organisms preserve self-similar shapes as they vary in size, then their linear dimensions should vary as the 1/3 and their surface areas as the 2/3 powers of their mass.
        
    \subsection{Rationale}
        \begin{itemize}
            \item We know fish display hyperallometric fecundity with mass
            \item This extends the ``General life history problem'' - should I wait to get bigger?
            \item Next issue is that when we simulate growth curves with high $\rho$ values, we encounter shrinking fish, because $\rho$ dominates and $\frac{dm}{dt} < 0$
            \item Where is the energy coming from to compensate for this hypothetical shrinking since we know they don't do this in reality over ontogeny
            \item Intake rate scaling may be the answer
        \end{itemize}
    \begin{table}[]
        \begin{tabular}{|l|l|l|l|l|}
        \hline
        \textbf{Study}                          & \textbf{Model type}       & \textbf{Scaling} & \textbf{Prediction/Assumption} & \textbf{Refs} \\ \hline
        \autocite{Gadgil1970}                   & Life history              & Hyperallometric  & Prediction                        & (40)          \\ \hline
        \autocite{Roff1983}                     & Life history              & Isometric        & Assumption                        & (41)          \\ \hline
        \autocite{Roff1984}                     & Life history              & Isometric        & Assumption                        & (42)          \\ \hline
        \autocite{Reiss1985}                    & Mechanistic               & Hypoallometric   & Prediction                        & (43)          \\ \hline
        \autocite{Kozowski1987-indeterminate}   & Life history              & Variable         & Prediction                        & (44)          \\ \hline
        \autocite{kozlowski1996}                & Life history              & Variable         & Prediction                        & (45)          \\ \hline
        \autocite{West2001}                     & Mechanistic               & Isometric        & Assumption                        & (46)          \\ \hline
        \autocite{Charnov2001}                  & Hybrid                    & Isometric        & Assumption                        & (47)          \\ \hline
        \autocite{Charnov2002}                  & Hybrid                    & Isometric        & Assumption                        & (48)          \\ \hline
        \autocite{Lester2004}                   & Life history - Biphasic   & Isometric*       & Assumption                        & (49)          \\ \hline
        \autocite{Roff2006}                     & Life history              & Isometric        & Assumption                        & (50)          \\ \hline
        \autocite{Quince2008}                   & Life history - Biphasic   & Isometric*       & Assumption                        & (51)          \\ \hline
        \autocite{Quince2008b}                  & Life history - Biphasic   & Hyperallometric* & Prediction                        & (51)          \\ \hline
        \autocite{Pecquerie2009}                & Life history              & Isometric        & Assumption                        & (52)          \\ \hline
        \autocite{kooijman2010dynamic}          & Mechanistic               & Isometric*       & Assumption                        & (53)          \\ \hline
        \autocite{Arendt2011}                   & Life history              & Isometry         & Assumption                        & (54)          \\ \hline
        \autocite{Ohnishi2011}                  & Life history - Biphasic   & Hyperallometric  & Assumption                        & (55)          \\ \hline
        \autocite{Brunel2013}                   & Mechanistic               & Isometric        & Assumption                        & (56)          \\ \hline
        \autocite{Charnov2013}                  & Hybrid                    & Isometric        & Assumption                        & (57)          \\ \hline
        \autocite{Boukal2014}                   & Life history - Biphasic   & Isometric*       & Assumption                        & (58)          \\ \hline
        \autocite{Kooijman2014a}                & Mechanistic               & Isometric*       & Assumption                        & (59)          \\ \hline
        \autocite{Minte-Vera2016a}              & Life history - Biphasic   & Isometric*       & Assumption                        & (60)          \\ \hline
        \autocite{Jusup2017}                    & Mechanistic               & Isometric        & Assumption                        & (61)          \\ \hline
        \autocite{Mangel2017}                   & Life history              & Hyperallometric  & Assumption                        & (62)          \\ \hline
        \autocite{Smallegange2017}              & Mechanistic               & Isometric        & Assumption                        & (63)          \\ \hline
        \autocite{Audzijonyte2018}              & Hybrid                    & Isometric        & Assumption                        & (63)          \\ \hline
        Beverton and Holt (1957)                & Fisheries                 & Isometric        & Assumption                        & (2)           \\ \hline
        Scott \textit{et al}. (2006)            & Fisheries                 & Hyperallometric  & Assumption                        & (64)          \\ \hline
        Jørgensen and Fisken (2006)             & Fisheries                 & Isometric*       & Assumption                        & (65)          \\ \hline
        Enberg \textit{et al}. (2010)           & Fisheries                 & Isometric        & Assumption                        & (66)          \\ \hline
        Eikeset \textit{et al}. (2013)          & Fisheries                 & Isometric        & Assumption                        & (67)          \\ \hline
        Lester \textit{et al}. (2014)           & Fisheries                 & Isometric        & Assumption                        & (68)          \\ \hline
        Andersen and Beyer (2015)               & Fisheries                 & Isometric        & Assumption                        & (69)          \\ \hline
        Eikeset \textit{et al}. (2016)          & Fisheries                 & Isometric        & Assumption                        & (70)          \\ \hline
        Andersen \textit{et al}. (2016)         & Fisheries                 & Isometric        & Assumption                        & (12)          \\ \hline
        Zimmerman and Jørgensen (2016)          & Fisheries                 & Isometric        & Assumption                        & (71)          \\ \hline
        Hartvig \textit{et al}. (2011)          & Food web                  & Isometric        & Assumption                        & (72)          \\ \hline
        Carozza \textit{et al}. (2016)          & Food web                  & Isometric        & Assumption                        & (73)          \\ \hline
        \end{tabular}
    \end{table}
%%% Methods %%%
\section{Methods}\thispagestyle{empty}
\lipsum

%%% Results %%%
\section{Results}\thispagestyle{empty}
\lipsum

%%% Discussion %%%
\section{Discussion}\thispagestyle{empty}
\lipsum

%%% Conclusion %%%
\section{Conclusion}\thispagestyle{empty}
\lipsum

%%% Bibliography %%%
\addcontentsline{toc}{section}{Bibliography}
\newpage\let\mkbibnamefamily\textsc\printbibliography[title=Bibliography]\thispagestyle{empty} % Sets author names to small caps

%%% Appendices %%%
\newpage\section*{Appendix A}\thispagestyle{empty}
\lipsum

\end{document}\grid
