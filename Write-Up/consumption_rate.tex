\section{Incorporating Consumption Rate}
At any time $t$, growth rate is considered as the difference between supply and expenditure. Thus we can incorporate intake scaling into this supply term.
We know that at time $t$, organisms consume resource at a rate proportional to their size:
\begin{align*}
    C \propto m
\end{align*}
This scales with size, thus we introduce a scaling exponent, $\gamma$:
\begin{align}
    C(t) &= C_{0}m^{\gamma} \label{time_scaling}
\end{align}
Consumption rate takes units \textsc{individuals/area/time} or which is the product of prey encountered (individuals/area) and handling time by the consumer, thus is a function of time. Fish consume resources for a time period, or the duration of a foraging bout. To calculate total intake (mass) in a foraging bout we can integrate \cref{time_scaling} with respect to time, where $t_0$ is an arbitrary time point at the commencement of a bout, and $t_1$ marks the end of a bout:
\begin{align*}
    C_{tot} &= \int_{t_{0}}^{t_{1}}C_{0}m^{\gamma}dt \\
    C_{tot} &= C_{0}m^{\gamma}t \numberthis \label{intake_amount}
\end{align*}
We assume that, as is ubiquitous across biological rates, foraging time scales with mass to some scaling exponent $\psi$ i.e. $t = t_{0}m^{\psi}$ which we can substitute into \cref{intake_amount}:
\begin{align*}
    \textsc{substitute } t &= t_{0}m^{\psi} \\
    C_{tot} &= C_{0}m^{\gamma}t_{0}m^{\psi} \\
    C_{tot} &= C_{0}t_{0}m^{\gamma + \psi} \numberthis \label{total_intake}
\end{align*}
where $C_{tot}$ has dimensions mass. I can assume that fish begin to allocate to growth \textbf{after} a foraging bout or that growth during a bout is negligible due to a disconnect in timescales between foraging (a matter of hours) and lifetime growth (a matter of years). This presupposition means total intake term is biologically identical to $a$ in equation (3). Assimilation of nutrients via digestion can never be 100\% efficient and we hence add an efficieny term, $\varepsilon$ to \cref{total_intake}, bounded $[0,1]$, which captures carbon loss of ingested resource during digestion due to thermodynamic constraints. \cref{total_intake} becomes:
\begin{align}
    \varepsilon C_{0}t_{0}m^{\gamma + \psi} \label{intake_term}
\end{align} 
We can now multiply this distribution term by our total energy intake term from equation (10):
In order to substitute our new term \cref{intake_term} we must consider the dimensions of the original growth equation \cref{west_model} on both sides, which must balance out.
\begin{align}
    \frac{dm}{dt} &= am^{3/4} - bm \\
    \frac{\textsc{grams}}{\textsc{sec}} &= \frac{\textsc{grams}}{\textsc{joules}}\cdot \frac{\textsc{joules}}{\textsc{sec}} - \frac{\frac{\textsc{joules}}{\textsc{sec}}}{\textsc{joules}}\cdot \textsc{grams} \\
    \frac{\textsc{grams}}{\textsc{sec}} &= \frac{\textsc{grams}}{\st{\textsc{joules}}}\cdot \frac{\st{\textsc{joules}}}{\textsc{sec}} - \frac{\frac{\st{\textsc{joules}}}{\textsc{sec}}}{\st{\textsc{joules}}}\cdot \textsc{grams} \\
    \frac{\textsc{grams}}{\textsc{sec}} &= \frac{\textsc{grams}}{\textsc{sec}} - \frac{\textsc{grams}}{\textsc{sec}} \\
    \textsc{substitute } & \varepsilon C_{0}t_{0}m^{\gamma + \psi} \\
    \frac{dm}{dt} &= C_{0}t_{0}\varepsilon am^{\gamma + \psi}\Big(m^{-\frac{1}{4}}\Big) - bm - cm^{\rho}
\end{align}
Let our new intake term, $a_0 = C_{0}t_{0}\varepsilon a$ and multiply out our new energy intake term:
\begin{align}
    \frac{dm}{dt} &= a_{0}m^{\gamma + \psi -\frac{1}{4}} - bm - cm^{\rho}
\end{align}
One can now observe that if consumption rate scales with mass for amount and time as $3/4$, as MTE would suggest (limited by resting metabolic rate), intake rate would scale superlinearly:
\begin{align}
    \frac{dm}{dt} &= a_{0}m^{\frac{3}{4} + \frac{3}{4} -\frac{1}{4}} - bm - cm^{\rho} \\
    \frac{dm}{dt} &= a_{0}m^{\frac{5}{4}} - bm - cm^{\rho}
\end{align}
Further, it has also been shown that the assumption of resting metabolic rate scaling may also be false in consumption rate scaling, as active MR has been shown to scale more steeply. If it is the case that consumption scales with active MR then it is likely that the total exponent value for $a_{0}m^\phi$ may be even larger.
