%% beamerthemeImperialPoster v1.0 2016/10/01
%% Beamer poster theme created for Imperial College by LianTze Lim (Overleaf)
%% LICENSE: LPPL 1.3
%%
%% This is the example poster demonstrating use
%% of the Imperial College Beamer Poster Theme
\documentclass[xcolor={table}]{beamer}
%% Possible paper sizes: a0, a0b, a1, a2, a3, a4 (although Imperial College posters are usually A0 or A1).
%% Possible orientations: portrait, landscape
%% Font sizes can be changed using the scale option.
\usepackage[size=a0,orientation=portrait,scale=1.55]{beamerposter}
\usepackage{pgfplots}
\pgfplotsset{width=25cm,compat=1.9}

\usetheme{ImperialPoster}

%% Four available colour themes
\usecolortheme{ImperialWhite} % Default
% \usecolortheme{ImperialLightBlue}
% \usecolortheme{ImperialDarkBlue}
% \usecolortheme{ImperialBlack}

\title{Explicitly Accounting for Resource Supply \& Hyperallometric Fecundity Scaling in Ontogenetic Growth Models}

\author{\mainauthor{Luke Vassor},\Tsup{1} Tom Clegg,\Tsup{1} Diego Barneche,\Tsup{2} Van Savage,\Tsup{3} Dustin Marshall,\Tsup{4} Samraat Pawar\Tsup{1}}

\institute{\Tsup{1}Department of Life Sciences, Silwood Park, Imperial College London \Tsup{2}College of Life and Environmental Science, University of Exeter \Tsup{3}Departments of Biomathematics and Ecology and Evolutionary Biology, UCLA \Tsup{4}Centre for Geometric Biology, Monash University, Australia}

\addbibresource{../../../Write-Up/CMEE_Thesis.bib}
% \usepackage[maxcitenames=2, backend=biber, style=imperialharvard]{biblatex}
% \addbibresource{../../../Write-Up/CMEE_Thesis.bib}


\begin{document}
\begin{frame}[fragile=singleslide,t]\centering

\maketitle

\begin{columns}[onlytextwidth,T]

%%%% First Column
\begin{column}{.47\textwidth}

\begin{block}{The ``General Life History Problem''}
    \begin{itemize}
        \item The onset of reproduction brings new costs to an organism. Life history theory assumes that evolution selects for strategies that maximise lifetime reproduction.
        \item The general life-history problem asks: ``\textit{given that reproduction propagates genetic material but also bears cost (e.g. increased parental mortality), how should reproduction be optimally distributed over ontogeny to maximise output?}"
        \item A large metastudy \autocite{Barneche2018d} has now extended this problem by showing that the total reproductive output of female fish scales hyperallometrically with body size (across species) i.e. $R \propto m^{>1}$, 
        \begin{itemize}
            \item 1 x 2kg fish reproduces > 2 x 1kg fish
        \end{itemize}
        \item This potentially resets a crucial theoretical assumption of many existing growth models - that reproduction scales \textit{isometrically} with body size i.e. $R \propto m^{1}$. 
    \end{itemize}
    %This begs a new question: ``given that fish are indeterminate growers, and that they can increase reproductive output by delaying maturation and investing in somatic growth, when is the optimal time to mature, and what is the optimal exponent? given a mortality rate"
\end{block}

\begin{block}{The model}
    \begin{itemize}
        \item Energy intake scales hypoallometrically with size $= am^{3/4}$%, constrained by the fractal-like architecture of a branching capillary network, 
        \item Cellular maintenance scales isometrically with size $= bm$ 
        \item Reproduction begins at maturity, age $\alpha$, size $m_{\alpha}$ and scales hyperallometrically with size $= cm^{\rho}$
        \item Surplus energy is optimally allocated to growth
        \item Optimising reproductive output ($R_{T}$ is constrained by probability of survival at age $x$, which decays exponentially $= L_x$
    \end{itemize}
    \begin{align*}
        \frac{dm}{dt} &= am^{3/4} - bm \; \; \; \; \; \; \; \; \; \; \; \; \; \; \; m < m_{\alpha} \\
        \frac{dm}{dt} &= am^{3/4} - bm - cm^{\rho} \; \; \; \;\; m > m_{\alpha} \\
        \text{Maximise } R_{T} &= \int_{\alpha}^{\infty} L_{x}cm_{x}^{\rho}dx
    \end{align*}
\end{block}

\begin{figure}
\includegraphics[width=\hsize]{../../../Results/optimisation.pdf}
\caption{Fitness surface for lifetime reproductive output ($R$) as a function of age at maturity ($\alpha$) and allometric scaling exponent of reproduction ($\rho$). Maximum reproduction results from highest exponent and intermediate maturation age.}
\end{figure}

% \begin{table}
% \begin{tabularx}{\linewidth}{  X  X  }
% \toprule
% \textbf{Table header (bold)} & \textbf{Table header (bold)} \\
% \midrule
% Table header (normal) & Table header (normal) \\
% \midrule
% Table header (normal) & Table header (normal) \\
% \midrule
% Table header (normal) & Table header (normal) \\
% \bottomrule
% \end{tabularx}
% \end{table}

\begin{block}{Acknowledgements}
Lorem ipsum dolor sit amet, consectetuer adipiscing elit. In nunc nisl, pharetra et, lacinia non, pellentesque sit amet, tellus. Sed tempus. In egestas. Maecenas aliquam libero vitae leo. Donec consectetuer. Pellentesque ultrices feugiat enim. Morbi tempus tortor non metus. Quisque vitae metus. Nulla justo.
\end{block}

\end{column}


%%%% Second Column
\begin{column}{.47\textwidth}

\begin{block}{Scaling of intake rate}
    \begin{itemize}
        \item Size-scaling laws are ubiquitous among biological rates
        \item This extends to consumption rate \autocite{Pawar2012}
        \begin{itemize}
            \item Different scalings for 2D/3D foraging
            \item Different scalings for resource-saturated/depleted environments
            \item 
        \end{itemize}
    \end{itemize}
    \begin{tikzpicture}
        \begin{axis}[
            ymode=log,
            axis lines = left,
            xlabel = $x$,
            ylabel = {$f(x)$},
        ]
        %Below the red parabola is defined
        \addplot [
            domain=0:10, 
            samples=100, 
            color=red,
        ]
        {x^0.75};
        \addlegendentry{$\beta = 3/4$}
        %Here the blue parabloa is defined
        \addplot [
            domain=0:10, 
            samples=100, 
            color=blue,
            ]
            {x^1.06 - 10};
        \addlegendentry{$\beta = 1.06$}
         
        \end{axis}
        \end{tikzpicture}
\end{block}

\begin{block}{Incorporating into OGM}
\begin{itemize}
    \item Current growth models assume that energy intake is fixed scales to $\frac{3}{4}$.
    \item In nunc nisl, pharetra et, lacinia non, pellentesque sit amet, tellus. Sed tempus.
    \item In egestas.
    \item Maecenas aliquam libero vitae leo. Donec consectetuer. 
    \item Pellentesque ultrices feugiat enim. Morbi tempus tortor non metus. 
    \item Nulla justo. Pellentesque laoreet dui a felis bibendum blandit. Quisque commodo. 
\end{itemize}
\end{block}

% \begin{table}
% \begin{tabularx}{\linewidth}{  X  X  }
% \toprule
% \textbf{Table header (bold)} & \textbf{Table header (bold)} \\
% \midrule
% Table header (normal) & Table header (normal) \\
% \midrule
% Table header (normal) & Table header (normal) \\
% \midrule
% Table header (normal) & Table header (normal) \\
% \bottomrule
% \end{tabularx}
% \end{table}

\begin{sidefigure}
\includegraphics[width=\hsize]{example-image-golden-upright}
\caption{Lorem ipsum dolor sit amet, consectetuer adipiscing elit. In nunc nisl, pharetra et, lacinia non, pellentesque sit amet, tellus. Sed tempus. In egestas. Maecenas aliquam libero vitae leo. Donec consectetuer. Pellentesque ultrices feugiat enim. Morbi tempus tortor non metus. Quisque vitae metus. Nulla justo. Pellentesque laoreet dui a felis bibendum blandit. Quisque commodo. Cras luctus vestibulum leo. Quisque varius mauris vel diam. Aenean quis erat. Vestibulum mauris. Suspendisse potenti. Vivamus consectetuer massa et nunc. Aliquam ornare risus eu risus. Suspendisse molestie urna vitae augue.

Aliquam erat volutpat. Duis at sapien vestibulum mi sollicitudin rutrum.}
\end{sidefigure}


\let\mkbibnamefamily\textsc\printbibliography

\end{column}
\end{columns}


\end{frame}
\end{document}