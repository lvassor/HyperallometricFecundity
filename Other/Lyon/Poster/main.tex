%% beamerthemeImperialPoster v1.0 2016/10/01
%% Beamer poster theme created for Imperial College by LianTze Lim (Overleaf)
%% LICENSE: LPPL 1.3
%%
%% This is the example poster demonstrating use
%% of the Imperial College Beamer Poster Theme
\documentclass[xcolor={table}]{beamer}
%% Possible paper sizes: a0, a0b, a1, a2, a3, a4 (although Imperial College posters are usually A0 or A1).
%% Possible orientations: portrait, landscape
%% Font sizes can be changed using the scale option.
\usepackage[size=a0,orientation=portrait,scale=1.8]{beamerposter}
\usepackage{pgfplots}
\pgfplotsset{width=25cm,compat=1.9}
\usepackage{tabularx}
\usepackage{enumitem}
\usepackage{caption}
\captionsetup{skip=0pt,belowskip=0pt}
\usetheme{ImperialPoster}
\newcolumntype{Y}{>{\centering\arraybackslash}X}
\renewcommand*{\bibfont}{\footnotesize}
\AtEveryBibitem{[\printfield{labelnumber}]\addspace}%Numbers in the bib
% reducing space around equations
% \setlength{\abovedisplayshortskip}{-1pt}
% \setlength{\belowdisplayshortskip}{-1pt}

% \AtBeginDocument{%
%    \setlength\abovedisplayskip{-2pt}
%    \setlength\belowdisplayskip{-2pt}}

% \setlist[itemize]{noitemsep, topsep=0pt} % reduce spacing before itemize
% \makeatletter
% \renewbibmacro*{textcite}{%
%   \iffieldequals{namehash}{\cbx@lasthash}
%     {\mkbibsuperscript{\supercitedelim}}
%     {\cbx@tempa
%      \ifnameundef{labelname}
%        {\printfield[citetitle]{labeltitle}}
%        {\printnames{labelname}}}%
%   \ifnumequal{\value{citecount}}{1}
%     {}
%     {}%
%   \mkbibsuperscript{\usebibmacro{cite}}%
%   \savefield{namehash}{\cbx@lasthash}%
%   \gdef\cbx@tempa{\addspace\multicitedelim}}%

% \DeclareCiteCommand{\textcite}
%   {\let\cbx@tempa=\empty
%    \undef\cbx@lasthash
%    \iffieldundef{prenote}
%      {}
%      {\BibliographyWarning{Ignoring prenote argument}}%
%    \iffieldundef{postnote}
%      {}
%      {\BibliographyWarning{Ignoring postnote argument}}}
%   {\usebibmacro{citeindex}%
%    \usebibmacro{textcite}}
%   {}
%   {}
% \makeatother
%% Four available colour themes
\usecolortheme{ImperialWhite} % Default
% \usecolortheme{ImperialLightBlue}
% \usecolortheme{ImperialDarkBlue}
% \usecolortheme{ImperialBlack}

\title{Reconciling Resource Supply \& Hyperallometric Fecundity Scaling in Ontogenetic Growth Models}

\author{\mainauthor{Luke Vassor},\Tsup{1} Tom Clegg,\Tsup{1} Diego Barneche,\Tsup{2} Dustin Marshall,\Tsup{3} Van Savage, \Tsup{4} Samraat Pawar\Tsup{1}}
% \vspace*{1pt}
\institute{\Tsup{1}Department of Life Sciences, Silwood Park, Imperial College London \Tsup{2}College of Life and Environmental Science, University of Exeter \Tsup{3}Centre for Geometric Biology, Monash University \Tsup{4}Departments of Biomathematics and Ecology and Evolutionary Biology, UCLA}

\addbibresource{../../../Write-Up/CMEE_Thesis.bib}
% \usepackage[maxcitenames=2, backend=biber, style=imperialharvard]{biblatex}
% \addbibresource{../../../Write-Up/CMEE_Thesis.bib}


\begin{document}
\begin{frame}[fragile=singleslide,t]\centering

\maketitle

\begin{columns}[onlytextwidth,T]

%%%% First Column
\begin{column}{.47\textwidth}

\begin{block}{The ``General Life History Problem''}
    \begin{itemize}
        \item ``\textit{Given that reproduction bears cost (e.g. increased parent mortality), how should it be optimally distributed over ontogeny to maximise output?}"
        \item Life history theory assumes that evolution selects for strategies which maximise lifetime reproduction
        \item A large metastudy \autocite{Barneche2018d} has extended this problem by showing that \textbf{total reproductive output} in female fish scales \textbf{hyper}allometrically with size - bigger mothers are disproportionately more fecund i.e. $R \propto m^{>1}$
        \\ \quad \quad \ \ \ 1 x 2kg \ \ \ \ \ \qquad \qquad \quad \quad 2 x 1kg
        \includegraphics[width=0.8\hsize,trim=0 0 0 1.1cm, clip]{./fish_figure.png}
    \end{itemize}
    % ,trim=0cm 0cm 0cm 0.8cm,clip
    %This begs a new question: ``given that fish are indeterminate growers, and that they can increase reproductive output by delaying maturation and investing in somatic growth, when is the optimal time to mature, and what is the optimal exponent? given a mortality rate"
\end{block}


\begin{block}{Shrinking fish and ``Missing Energy''}
    \begin{itemize}
        \item Simulated growth curves with high fecundity scaling exponents suggests fish should shrink at maturity
        \item Where do they obtain the energy to prevent shrinking?
    \end{itemize}
\end{block}
\begin{figure}
    \vspace*{-1.4cm} % reduce white space before figure
    \includegraphics[width=\hsize]{../../../Results/growth_curve_rho_good.pdf}
    \caption{Ontogenetic growth curves for different maturation ages (dashed lines: $\alpha = 100, 200, 300$) and different reproductive exponents ($\rho$). Large $\rho$ values produce shrinking fish.}
\end{figure}

\begin{block}{Size-Scaling of intake rate}
    \begin{itemize}
        \item Size-scaling laws of biological rates are ubiquitous \autocite{peters1983}
        \item This extends to consumption rate - different size scalings for different environments i.e. $C \propto m^\gamma$ \autocite{Pawar2012}
    \end{itemize}

    % \begin{tikzpicture}
    %     \begin{axis}[
    %         ymode=log,
    %         axis lines = left,
    %         xlabel = $x$,
    %         ylabel = {$f(x)$},
    %     ]
    %     %Below the red parabola is defined
    %     \addplot [
    %         domain=0:10, 
    %         samples=100, 
    %         color=red,
    %     ]
    %     {x^0.75};
    %     \addlegendentry{$\beta = 3/4$}
    %     %Here the blue parabloa is defined
    %     \addplot [
    %         domain=0:10, 
    %         samples=100, 
    %         color=blue,
    %         ]
    %         {x^1.06 - 10};
    %     \addlegendentry{$\beta = 1.06$}
         
    %     \end{axis}
    %     \end{tikzpicture}
\begin{table}
    \setlength\abovecaptionskip{20pt}
    \setlength\belowcaptionskip{-30pt}
    \caption{Size-scaling exponents of consumption rate in different foraging dimensions and resource states. 2D v 3D \& Empirical v Predicted (parentheses) are significantly different in all cases \autocite{Pawar2012}}
    \renewcommand{\arraystretch}{0.9}% Tighter
    \begin{tabularx}{\linewidth}{l  Y  Y  }
    \toprule
    \rowcolor{blue!20}    & \multicolumn{2}{c}{\textbf{Consumption rate scaling exponent}}                                        \\
    \midrule
    \rowcolor{blue!20}& \multicolumn{1}{c}{\textbf{Scarce resources}} & \multicolumn{1}{c}{\textbf{Abundant resources}} \\
    \midrule
    \rowcolor{blue!5}2D & 0.85 $\pm$ 0.05 (0.78)                      & 0.85 $\pm$ 0.05 (0.78)                        \\
    \midrule
    \rowcolor{blue!20}3D & 1.06 $\pm$ 0.06 (1.16)                      & 1.00 $\pm$ 0.06 (1.16)                       \\              
    \bottomrule
    \end{tabularx}
\end{table}
\end{block}
% \begin{table}
% \begin{tabularx}{\linewidth}{  X  X  }
% \toprule
% \textbf{Table header (bold)} & \textbf{Table header (bold)} \\
% \midrule
% Table header (normal) & Table header (normal) \\
% \midrule
% Table header (normal) & Table header (normal) \\
% \midrule
% Table header (normal) & Table header (normal) \\
% \bottomrule
% \end{tabularx}
% \end{table}

% \begin{block}{Acknowledgements}
%     \textbf{Author contributions:} L.J.V., S.P. and T.C. conceived the study. V.M.S. performed the majority of mathematics. D.R.B. provided insight into and data on fish growth. D.J.M. 
% \end{block}

\end{column}


%%%% Second Column
\begin{column}{.47\textwidth}

\begin{block}{Ontogenetic Growth Models}
    \begin{itemize}
        \item Extant models typically assume energy is optimally allocated to somatic growth \textbf{after} maintenance and reproduction (post-maturation) costs
        \item Under metabolic theory, energy intake is assumed to scale \textbf{hypo}allometrically with mass i.e. $I \propto m^{3/4}$ \autocite{West2001}
    \end{itemize}
    \begin{align*}
        \frac{dm}{dt} &= am^{3/4} - bm - cm^{\rho} \; \; \; \;\; m > m_{\alpha}
    \end{align*}
    \begin{itemize}[noitemsep]
        \item We theorise that the ``missing energy'' causing shrinkage is obtained via the superlinear scaling of intake rate (Table 1)
        \item In a 3D environment with scarce resources, intake would scale as $I = I_{0}m^{1.06}$
        \item Organisms can switch between environments 
        \item Given this intake rate and a survivorship probability, organisms can then optimise their age-at-maturtiy ($\alpha$) and fecundity allometry ($\rho$), extending the general life history problem
    \end{itemize}
    % \begin{align*}
    %     \text{Maximise } R_{T} &= \int_{\alpha}^{\infty} L_{x}cm_{x}^{\rho}dx
    % \end{align*}
\begin{figure}
    \vspace*{-1.4cm} % reduce white space before figure
    \includegraphics[width=\hsize,trim=0cm 0cm 0cm 0.8cm,clip]{../../../Results/optimisation_square.pdf}
    \caption{Fitness surface for lifetime reproductive output ($R$) as a function of age at maturity ($\alpha$) and allometric scaling exponent of reproduction ($\rho$). ATR = allocation to reproduction.}
\end{figure}
\end{block}
\let\mkbibnamefamily\textsc\printbibliography

\end{column}
\end{columns}


\end{frame}
\end{document}