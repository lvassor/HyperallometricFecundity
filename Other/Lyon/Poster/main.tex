%% beamerthemeImperialPoster v1.0 2016/10/01
%% Beamer poster theme created for Imperial College by LianTze Lim (Overleaf)
%% LICENSE: LPPL 1.3
%%
%% This is the example poster demonstrating use
%% of the Imperial College Beamer Poster Theme
\documentclass[xcolor={table}]{beamer}
%% Possible paper sizes: a0, a0b, a1, a2, a3, a4 (although Imperial College posters are usually A0 or A1).
%% Possible orientations: portrait, landscape
%% Font sizes can be changed using the scale option.
\usepackage[size=a0,orientation=portrait,scale=1.55]{beamerposter}
\usepackage{pgfplots}
\pgfplotsset{width=25cm,compat=1.9}
\usepackage{tabularx}
\usetheme{ImperialPoster}


\makeatletter
\renewbibmacro*{textcite}{%
  \iffieldequals{namehash}{\cbx@lasthash}
    {\mkbibsuperscript{\supercitedelim}}
    {\cbx@tempa
     \ifnameundef{labelname}
       {\printfield[citetitle]{labeltitle}}
       {\printnames{labelname}}}%
  \ifnumequal{\value{citecount}}{1}
    {}
    {}%
  \mkbibsuperscript{\usebibmacro{cite}}%
  \savefield{namehash}{\cbx@lasthash}%
  \gdef\cbx@tempa{\addspace\multicitedelim}}%

\DeclareCiteCommand{\textcite}
  {\let\cbx@tempa=\empty
   \undef\cbx@lasthash
   \iffieldundef{prenote}
     {}
     {\BibliographyWarning{Ignoring prenote argument}}%
   \iffieldundef{postnote}
     {}
     {\BibliographyWarning{Ignoring postnote argument}}}
  {\usebibmacro{citeindex}%
   \usebibmacro{textcite}}
  {}
  {}
\makeatother
%% Four available colour themes
\usecolortheme{ImperialWhite} % Default
% \usecolortheme{ImperialLightBlue}
% \usecolortheme{ImperialDarkBlue}
% \usecolortheme{ImperialBlack}

\title{Explicitly Accounting for Resource Supply \& Hyperallometric Fecundity Scaling in Ontogenetic Growth Models}

\author{\mainauthor{Luke Vassor},\Tsup{1} Tom Clegg,\Tsup{1} Diego Barneche,\Tsup{2} Van Savage,\Tsup{3} Dustin Marshall,\Tsup{4} Samraat Pawar\Tsup{1}}

\institute{\Tsup{1}Department of Life Sciences, Silwood Park, Imperial College London \Tsup{2}College of Life and Environmental Science, University of Exeter \Tsup{3}Departments of Biomathematics and Ecology and Evolutionary Biology, UCLA \Tsup{4}Centre for Geometric Biology, Monash University, Australia}

\addbibresource{../../../Write-Up/CMEE_Thesis.bib}
% \usepackage[maxcitenames=2, backend=biber, style=imperialharvard]{biblatex}
% \addbibresource{../../../Write-Up/CMEE_Thesis.bib}


\begin{document}
\begin{frame}[fragile=singleslide,t]\centering

\maketitle

\begin{columns}[onlytextwidth,T]

%%%% First Column
\begin{column}{.47\textwidth}

\begin{block}{The ``General Life History Problem''}
    \begin{itemize}
        \item ``\textit{Given that reproduction bears cost (e.g. increased parental mortality), how should reproduction be optimally distributed over ontogeny to maximise output?}"
        \item Life history theory assumes that evolution selects for strategies which maximise lifetime reproduction.
        \item A large metastudy \autocite{Barneche2018d} has extended this problem by showing that the \textbf{total reproductive output} of female fish scales \textbf{hyper}allometrically with body size - bigger mothers reproduce disproportionately more i.e. $R \propto m^{>1}$
        \includegraphics[width=0.8\hsize]{./fish_figure.png}
    \end{itemize}
    %This begs a new question: ``given that fish are indeterminate growers, and that they can increase reproductive output by delaying maturation and investing in somatic growth, when is the optimal time to mature, and what is the optimal exponent? given a mortality rate"
\end{block}


\begin{block}{Shrinking fish and ``Missing Energy''}
    \begin{itemize}
        \item Simulating growth curves (Figure 1) with high reproductive exponents suggests fish should be shrinking after maturity
        \item Where are they obtaining the extra energy to prevent shrinking?
    \end{itemize}
\end{block}

%%%%% Growth Curves Figure %%%%%
\begin{figure}
    \includegraphics[width=\hsize]{../../../Results/growth_curve_rho.pdf}
    \caption{Ontogenetic growth curves for different maturity ages (dashed lines: $\alpha = 100, 200, 300$) and different reproductive exponents ($\rho$). Large $\rho$ values produce shrinking fish.}
\end{figure}

\begin{block}{Scaling of intake rate}
    \begin{itemize}
        \item Size-scaling laws are ubiquitous among biological rates data \autocite{peters1983}
        \item This extends to consumption rate --- different size-scalings for different environments i.e. $C \propto m^\gamma$ \autocite{Pawar2012}
    \end{itemize}
    % \begin{tikzpicture}
    %     \begin{axis}[
    %         ymode=log,
    %         axis lines = left,
    %         xlabel = $x$,
    %         ylabel = {$f(x)$},
    %     ]
    %     %Below the red parabola is defined
    %     \addplot [
    %         domain=0:10, 
    %         samples=100, 
    %         color=red,
    %     ]
    %     {x^0.75};
    %     \addlegendentry{$\beta = 3/4$}
    %     %Here the blue parabloa is defined
    %     \addplot [
    %         domain=0:10, 
    %         samples=100, 
    %         color=blue,
    %         ]
    %         {x^1.06 - 10};
    %     \addlegendentry{$\beta = 1.06$}
         
    %     \end{axis}
    %     \end{tikzpicture}
\end{block}

%%%%%% Intake Rate Exponents %%%%%%
\begin{table}
    \begin{tabularx}{\linewidth}{l  X  X  }
    \toprule
        & \multicolumn{2}{c}{\textbf{Consumption rate}}                                        \\
    \midrule
        & \multicolumn{1}{c}{\textbf{Scarce resources}} & \multicolumn{1}{c}{\textbf{Abundant resources}} \\
    \midrule
     2D & 0.85 $\pm$ 0.05                      & 0.85 $\pm$ 0.05                        \\
    \midrule
     3D & 1.06 $\pm$ 0.06                      & 1.00 $\pm$ 0.06                        \\              
    \bottomrule
    \end{tabularx}
\end{table}

% \begin{table}
% \begin{tabularx}{\linewidth}{  X  X  }
% \toprule
% \textbf{Table header (bold)} & \textbf{Table header (bold)} \\
% \midrule
% Table header (normal) & Table header (normal) \\
% \midrule
% Table header (normal) & Table header (normal) \\
% \midrule
% Table header (normal) & Table header (normal) \\
% \bottomrule
% \end{tabularx}
% \end{table}

\begin{block}{Acknowledgements}
    \textbf{Author contributions:} L.J.V., S.P. and T.C. conceived the study. V.M.S. performed the majority of mathematics. D.R.B. provided insight into and data on fish growth. D.J.M. 
\end{block}

\end{column}


%%%% Second Column
\begin{column}{.47\textwidth}

\begin{block}{Ontogenetic Growth Models}
    \begin{itemize}
        \item Current mechanistic growth models typically assume that energy is optimally dedicated to somatic growth \textbf{after} maintenance and reproduction (post-maturation) costs
        \item Under metabolic theory, most assume that energy intake scales with \textbf{hypo}allometrically with mass as $I \propto m^{3/4}$ \autocite{West2001}
    \end{itemize}
    \begin{align*}
        \frac{dm}{dt} &= am^{3/4} - bm - cm^{\rho} \; \; \; \;\; m > m_{\alpha} \\
        % aosoossjejje \uparrow \uparrow \uparrow \\
    \end{align*}
    \begin{itemize}
        \item We theorise that the ``missing energy'' causing shrinkage is obtained via the superlinear scaling of intake rate (Table 1)
        \item In a 3D environment with scarce resources, intake would scales as $I = I_{0}m^{1.06}$
        \item Organisms can switch between environments 
        \item Given this intake rate and a survivorship probability, organisms can then optimise their age-at-maturtiy ($\alpha$) and fecundity allometry ($\rho$), extending the general life history problem
    \end{itemize}
    % \begin{align*}
    %     \text{Maximise } R_{T} &= \int_{\alpha}^{\infty} L_{x}cm_{x}^{\rho}dx
    % \end{align*}
\end{block}

\begin{figure}
\includegraphics[width=\hsize]{../../../Results/optimisation_square.pdf}
\caption{Fitness surface for lifetime reproductive output ($R$) as a function of age at maturity ($\alpha$) and allometric scaling exponent of reproduction ($\rho$). Maximum reproduction results from highest exponent and intermediate maturation age.}
\end{figure}

\let\mkbibnamefamily\textsc\printbibliography

\end{column}
\end{columns}


\end{frame}
\end{document}