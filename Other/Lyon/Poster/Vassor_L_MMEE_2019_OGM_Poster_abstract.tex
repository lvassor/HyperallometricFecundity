\documentclass{article}
\usepackage[margin=2cm]{geometry}
\usepackage{graphicx}
\usepackage{lineno}
\usepackage{authblk}
\usepackage{setspace}
\usepackage{abstract}
\makeatletter
\renewcommand{\maketitle}{\bgroup\setlength{\parindent}{0pt}
\begin{flushleft}
  \Large{\textbf{\@title}}

  \large{\@author}
\end{flushleft}\egroup
}
\makeatother


\title{\textbf{\huge Explicitly Accounting for Resource Supply in Ontogenetic Growth Models and addressing the problem of allometric fecundity scaling}}
\author[1]{Luke Vassor}
\author[1]{Tom Clegg}
\author[2]{Diego Barneche}
\author[3]{Van Savage​}
\author[1]{Samraat Pawar}

\affil[1]{Department of Life Sciences, Silwood Park Campus, Imperial College London}
\affil[2]{College of Life and Environmental Science, University of Exeter}
\affil[3]{Departments of Biomathematics and Ecology and Evolutionary Biology, UCLA}
\vspace{15mm}

\begin{document}

\maketitle
\doublespacing

\section*{Abstract}
\linenumbers
The last century has seen many biologists attempt to successfully capture growth through ontogeny in the form of sophisticated mathematical models. Mechanistic approaches in developing these models have yielded several venerable biological theories which describe growth across different taxa and both determinate and indeterminate growth schedules. These include bottom-up approaches utilising first priniciples such as Metabolic Theory and Dynamic Energy Budget (DEB) theory and arguably more top-down approaches such as Evolutionary Life History Theory and Game Theory. In the former, while the "master" growth equation remains undisputed as a theoretical foundation, we believe there to be a gap in the literature, specifically how energy supply, realised through resource consumption affects ontogenetic growth. To this end, can the dispersal of growth data around the existing models be explained by the differential scaling of resource consumption rates with body mass found in previous studies? In this meta-study we address this gap by using a large consumer-resource dataset and results from a previous study to incorporate energy acquisition via consumption into an existing growth model. Further, we investigate the implications of the hyperallometry of consumption rate mass scaling on the scaling of allocation of energy/mass to reproduction, which was found to scale superlinearly in recent literature.


\textbf{Keywords}: Mechanistic, Model, DEB, Life History, Optimal Foraging, ESS

\end{document}