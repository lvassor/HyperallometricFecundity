\documentclass[handout]{beamer}
%\documentclass[presentation]{beamer}

\usecolortheme{Imperial}
 
\usepackage[utf8]{inputenc}
\usepackage[UKenglish]{babel}
\usepackage{booktabs}
\usepackage{caption}
\usepackage{subcaption}
\usepackage{graphicx}
\usepackage{amsmath}
\usepackage{amsfonts}
\usepackage{amssymb}
\usepackage{epstopdf}
\usepackage{pgfplots}
\usepackage{tabularx}
\usepackage{array}

% complying UK date format, i.e. 1 January 2001
\usepackage{datetime}
\let\dateUKenglish\relax
\newdateformat{dateUKenglish}{\THEDAY~\monthname[\THEMONTH] \THEYEAR}

% Imperial College Logo, not to be changed!
\institute{\includegraphics[height=0.7cm]{Imperial_1_Pantone_solid.pdf}}

% -----------------------------------------------------------------------------




%Information to be included in the title page:
\title{Explicitly incorporating resource supply into an Ontogenetic Growth Model}

\subtitle{Theoretical Ecology}

\author{Luke Vassor \\ \textit{M.Sc. CMEE}}

\date{\today}



\begin{document}
 
\frame{\titlepage}


\begin{frame}
	\frametitle{Background - Growth}
	\begin{itemize}
		\item Organisms grow from birth to an asymptotic size, typically sigmoidal
		\item Metabolic Theory: Growth rate slows because as we grow larger, we need to maintain more and more cells
		\item Maintenance rate scales more steeply than energy supply
		\item Growth ``scope'' is energy left over
		\item Different growth schedules:
		\begin{itemize}
			\item Determinate - eventually usage = supply
			\item Indeterminate - usage tends towards supply over lifetime
		\end{itemize}
	\end{itemize}
	% \begin{table}[h]
	% 	\centering
	% 	\begin{tabular}{p{0.2\textwidth} p{0.7\textwidth}} 
	% 		\toprule
	% 		\multicolumn{2}{p{0.9\textwidth}}{TableTitle} \\
	% 		\midrule
	% 		CapA       & lots of examplesA \\

	% 		CapB       & lots of examplesB \\
			
	% 		CapC       & lots of examplesC \\
	% 		\bottomrule
	% 	\end{tabular} 
	% \end{table}

	\begin{block}{General model}
		\begin{align*}
			\frac{dm}{dt} = am^{3/4} - bm
		\end{align*}
	\end{block}
\end{frame}


\begin{frame}
	\frametitle{Scaling of Metabolic Rate \& Intake Rate}
	\begin{columns}[b]
		\column{0.0125\textwidth}
		\column{0.4875\textwidth}
			\centering
			\begin{figure}
				\includegraphics[width=\textwidth]{determinate.pdf} \
				% remove the 'draft' keyword, when replacing with final figure!
				\caption{Determinate growth}
			\end{figure}
		\column{0.4875\textwidth}
			\centering
			\begin{figure}
				\includegraphics[width=\textwidth]{indeterminate.pdf} \
				% remove the 'draft' keyword, when replacing with final figure!
				\caption{Indeterminate growth}
			\end{figure}
		\column{0.0125\textwidth}
	\end{columns}
\end{frame}


\begin{frame}
	\frametitle{Background - Life History Theory}
	\begin{itemize}
		\item Life History Theory assumes that evolution will select for the strategy which maximises fitness - i.e. optimising the timing and magnitude of life history events.
		\item Lifetime reproductive output is a direct measure of fitness
		\item Organism dedicates a fraction of mass to fecundity at time $t$, $b(t)$ and has a $\mathbb{P}$(Survival to $t$), $L(t)$
		\item Maximising fitness means maximising the quanitity:
	\end{itemize}
	\begin{equation*}
		R_0 = \int_{\alpha}^{\infty}L(t)b(t) dt
	\end{equation*}

	% \begin{columns}[]
	
	% 	\column{0.0125\textwidth}
	% 	\column{0.4875\textwidth}
	% 		\centering
	% 		\begin{figure}
	% 			\includegraphics[draft,width=\textwidth]{fooC.pdf} \
	% 			% remove the 'draft' keyword, when replacing with final figure!
	% 			\caption{Caption of Figure C}
	% 		\end{figure}
	% 	\column{0.4875\textwidth}
	% 		Some text and a bullet point list
	% 		\begin{itemize}
	% 			\item ItemA
	% 			\item ItemB
	% 			\item ItemC
	% 			\item ItemD
	% 		\end{itemize}			
	% 	\column{0.0125\textwidth}
	% \end{columns}
\end{frame}


\begin{frame}
	\frametitle{Fitness - Lifetime Reproductive Output}
	\bigskip
	\begin{figure}
		\includegraphics[width=0.8\textwidth, height=0.5\textwidth]{fecundity.pdf}
		% remove the 'draft' keyword, when replacing with final figure!
		\caption{Lifetime reproductive output from maturity to $t_{\infty}$}
	\end{figure}
\end{frame}

\begin{frame}
	\frametitle{``Thick Fish''}
	\begin{itemize}
		\item Barneche \textit{et al.}: ``Fish fecundity scales hyperallometrically with size''
	\end{itemize}
	\begin{block}{Proposed model}
		\begin{align*}
			\frac{dm}{dt} = am^{3/4} - bm - cm^{\rho}
		\end{align*}
	\end{block}
	\begin{itemize}
		\item This extends the ``general life history problem'' - how and when should I reproduce?
	\end{itemize}
\end{frame}

\begin{frame}
	\frametitle{How can you maximise lifetime reproduction?}
	\begin{itemize}
		\item Which strategies are optimal to maximise:
	\end{itemize}
	\begin{equation*}
		R_0 = \int_{\alpha}^{\infty}L(t)b(t) dt
	\end{equation*}
	\begin{itemize}
		\item Adjust $\alpha$ - maturity age, but $\alpha Z$ invariant
		\item Adjust $c$ - allocation to reproduction, $10-25\%$
		\item Adjust $\rho$ - scaling of reproduction with mass, $1.29$
	\end{itemize}
\end{frame}

\begin{frame}
	\frametitle{Shrinking fish}
	\begin{itemize}
		\item When we incorporate reproductive hyperallometry into our model simulation, we get shrinking fish
	\end{itemize}
	\begin{columns}[b]
		\column{0.0125\textwidth}
		\column{0.4875\textwidth}
			\centering
			\begin{figure}
				\includegraphics[width=\textwidth]{growth_curves.pdf} \
				% remove the 'draft' keyword, when replacing with final figure!
				\caption{Lifetime ontogenetic growth curves (\textit{Gadus morhua})}
			\end{figure}
		\column{0.4875\textwidth}
			\centering
			\begin{figure}
				\includegraphics[width=\textwidth]{reproductive_allocation.pdf} \
				% remove the 'draft' keyword, when replacing with final figure!
				\caption{Cumulative allocation to reproduction (\textit{Gadus morhua})}
			\end{figure}
		\column{0.0125\textwidth}
	\end{columns}
\end{frame}

\begin{frame}
	\frametitle{``Missing'' Energy}
	\begin{itemize}
		\item Where is the energy shortfall coming from?
		\item Pawar \textit{et al.}'s Intake rate mass scaling laws
	\end{itemize}
	\begin{table}
		\tiny
		\caption{Size-scaling exponents of consumption rate in different foraging dimensions and resource states. 2D v 3D \& Empirical v Predicted (parentheses) are significantly different in all cases (Pawar \textit{et al.,} 2012)}
		\begin{tabularx}{\linewidth}{|X|X|X|}
		\toprule
		& \multicolumn{2}{c}{\textbf{Consumption rate scaling exponent}}                                        \\
		\midrule
		& \multicolumn{1}{c}{\textbf{Scarce resources}} & \multicolumn{1}{c}{\textbf{Abundant resources}} \\
		\midrule
		2D & 0.85 $\pm$ 0.05 (0.78)                      & 0.85 $\pm$ 0.05 (0.78)                        \\
		\midrule
		3D & 1.06 $\pm$ 0.06 (1.16)                      & 1.00 $\pm$ 0.06 (1.16)                       \\              
		\bottomrule
		\end{tabularx}
	\end{table}
\end{frame}


\begin{frame}
	\frametitle{Objective: New Model which Incorporates Intake Scaling}
	\begin{columns}[]
		\column{0.0125\textwidth}
		\column{0.4875\textwidth}
			\centering
			\begin{figure}
				\includegraphics[width=\textwidth]{../../Results/optimisation_square.pdf}\
				% remove the 'draft' keyword, when replacing with final figure!
				\caption{LH fitness surface}
			\end{figure}
		\column{0.4875\textwidth}
		\begin{align*}
			\frac{dm}{dt} = am^{3/4} - bm - cm^{\rho}
		\end{align*}
			\begin{itemize}
				\item Fish aren't acquiring enough energy to reproduce AND maintain size/grow
				\item Incorporate scaling laws into $am^{3/4}$
				\item See if reproductive hyperallometry ($\rho > 1$) emerges
			\end{itemize}			
		\column{0.0125\textwidth}
	\end{columns}
\end{frame}

\begin{frame}
	\small
	\frametitle{Model derivation}
	\begin{align*}
		\text{Intake rate} &= I_{0}m^{\gamma} \\
		\text{Total intake} &= I_{0}m^{\gamma}*t \\
		\text{Total intake} &= I_{0}m^{\gamma}t_{0}m^{\psi} \\
		\text{Total intake, } I_{tot} &= I_{0}t_{0}m^{\gamma + \psi}
	\end{align*}
	\begin{align*}
		\frac{dm}{dt} &= I_{tot}\epsilon\Big(am^{-\frac{1}{4}}\Big) - bm - cm^{\rho} \\
		\frac{dm}{dt} &= I_{0}t_{0}m^{\gamma + \psi}\epsilon\Big(am^{-\frac{1}{4}}\Big) - bm - cm^{\rho} \\
		\frac{dm}{dt} &= I_{0}t_{0}\epsilon am^{\gamma + \psi}\Big(m^{-\frac{1}{4}}\Big) - bm - cm^{\rho} \\
		\text{let } a_0 &= I_{0}t_{0}\epsilon a \\
		\frac{dm}{dt} &= a_{0}m^{\gamma + \psi -\frac{1}{4}} - bm - cm^{\rho} \\
	\end{align*}
\end{frame}
\end{document}

