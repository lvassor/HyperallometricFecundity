\documentclass[handout]{beamer}
%\documentclass[presentation]{beamer}

\usecolortheme{Imperial}
 
\usepackage[utf8]{inputenc}
\usepackage[UKenglish]{babel}
\usepackage{booktabs}
\usepackage{caption}
\usepackage{subcaption}
\usepackage{graphicx}
\usepackage{amsmath}
\usepackage{amsfonts}
\usepackage{amssymb}
\usepackage{epstopdf}
\usepackage{pgfplots}
\usepackage{tabularx}
\usepackage{array}
\usepackage{xpatch}

\usepackage[maxcitenames=2, backend=biber, dashed=false, style=imperialharvard]{biblatex}
    \addbibresource{../../Write-Up/CMEE_Thesis.bib}
    \setlength\bibitemsep{0.5\baselineskip}
\renewbibmacro*{textcite}{%
  \ifnameundef{labelname}
    {\iffieldundef{shorthand}
       {\usebibmacro{cite:label}%
        \setunit{%
          \global\booltrue{cbx:parens}%
          \addspace\bibopenparen}%
        \ifnumequal{\value{citecount}}{1}
          {\usebibmacro{prenote}}
          {}%
        \usebibmacro{cite:labelyear+extrayear}}
       {\usebibmacro{cite:shorthand}}}
    {\printnames{labelname}%
     \setunit{%
       \global\booltrue{cbx:parens}%
       \addspace\bibopenparen}%
     \ifnumequal{\value{citecount}}{1}
       {\usebibmacro{prenote}}
       {}%
     \usebibmacro{citeyear}}}
    % \renewcommand*\finalnamedelim{\addspace\&\space}

% Force \textit{et al}. to be italicised

    \xpatchbibmacro{name:andothers}{%
        \bibstring{andothers}%
    }{%
    \bibstring[\emph]{andothers}%
    }
    \xpatchbibmacro{Name}{%
        \bibstring{name}
    }{%
    \bibstring[\textbf]{name}%
    }{}{} % leave these empty arguments here, seem to cause issues when ommited
    \xpatchbibmacro{textcite} % combine same author different year citations
  {\setunit{\addcomma}\usebibmacro{cite:extrayear}}
  {\setunit{\compcitedelim}\usebibmacro{cite:labelyear+extrayear}}
  {}
  {}
% Kill "Accessed on" lines in bibliography
\AtEveryBibitem{
    \clearfield{urlyear}
    \clearfield{urlmonth}
    \clearfield{doi}
    \clearfield{url}
    \clearfield{eprint}
    \clearfield{eprinttype}
}

% complying UK date format, i.e. 1 January 2001
\usepackage{datetime}
\let\dateUKenglish\relax
\newdateformat{dateUKenglish}{\THEDAY~\monthname[\THEMONTH] \THEYEAR}

% Imperial College Logo, not to be changed!
\institute{\includegraphics[height=0.7cm]{logo.png}}

% -----------------------------------------------------------------------------




%Information to be included in the title page:
\title{Realistic Intake Rate Scaling Allows For Hyperallometric Fecundity Rate and Later Maturity In Fish}

\subtitle{Theoretical Ecology}

\author{Luke Vassor \\ \textit{M.Sc. CMEE}}

\date{\today}



\begin{document}
 
\frame{\titlepage}


\begin{frame}
	\frametitle{Ontogenetic Growth}
	\begin{itemize}
		\item Organisms acquire resources from the environment to grow new biomass
		\item Costs to address: maintaining cells, reproduction
		\item Ecological Metabolic Theory: Growth rate slows because as we grow larger, we need to maintain more and more cells
		\item Life History Theory: Growth rates change throughout ontogeny because evolution has selected these timings to maximise reproductive output
	\end{itemize}
	% \begin{table}[h]
	% 	\centering
	% 	\begin{tabular}{p{0.2\textwidth} p{0.7\textwidth}} 
	% 		\toprule
	% 		\multicolumn{2}{p{0.9\textwidth}}{TableTitle} \\
	% 		\midrule
	% 		CapA       & lots of examplesA \\

	% 		CapB       & lots of examplesB \\
			
	% 		CapC       & lots of examplesC \\
	% 		\bottomrule
	% 	\end{tabular} 
	% \end{table}
\end{frame}


\begin{frame}
	\frametitle{Power laws: Metabolic Rate \& Intake Rate}
		\centering
		\begin{figure}
			\includegraphics[width=\textwidth]{./Figures/presentation_growth_schedules.pdf} \
			% remove the 'draft' keyword, when replacing with final figure!
			\caption{Maintenance (metabolic) rate scales with mass more steeply than intake rate, causing growth to slow or terminate.}
		\end{figure}
\end{frame}

\begin{frame}
	\frametitle{Power laws: Fecundity Rate}
	\begin{itemize}
		\item New results from \textcite{Barneche2018-reproductive_output}: fish data suggest that the rate of allocation to fecundity may scale hyperallometrically
		\item Bigger fish mothers are disproportionately more fecund
	\end{itemize}
		\centering
		\begin{figure}
			\includegraphics[width=\textwidth]{./Figures/presentation_growth_schedules.pdf} \
			% remove the 'draft' keyword, when replacing with final figure!
			\caption{Maintenance (metabolic) rate scales with mass more steeply than intake rate, causing growth to slow or terminate.}
		\end{figure}
\end{frame}

\begin{frame}
	\frametitle{Capturing this in a model}
	\begin{itemize}
		\item Intake rate scales less steeply than maintenance: $am^{3/4}$
		\item $3/4$ exponent originates from how metabolic rate scales with mass - metabolic rate decreases with size, so larger organisms need less fuel per unit of mass
		\item Maintenance scales as: $bm^{1}$
		\item $1$ exponent as double the cells = double the maintenance costs
		\item Reproduction scales as: $cm^{\rho}$
	\end{itemize}

	\begin{block}{Growth model}
		\begin{align*}
			\frac{dm}{dt} &= am^{3/4} - bm \ \ \ \ \ \ \ \ \ \ \ \ \ \ \ \ \ m < m_{\alpha}\\
			\frac{dm}{dt} &= am^{3/4} - bm - cm^{\rho} \ \ \ \ \  \ \ \ m \geq m_{\alpha}
		\end{align*}
	\end{block}
\end{frame}

\begin{frame}
	\frametitle{Life History Theory}
	\begin{itemize}
		\item Life History Theory assumes that evolution will select for the strategy which maximises fitness
		\begin{itemize}
			\item Natural selection optimises the timing and magnitude of life history events.
		\end{itemize}
		\item Lifetime reproductive output is a direct measure of fitness and is impacted by several factors:
		\begin{enumerate}
			\item Organism dedicates a fraction of mass to fecundity at any time 
			\item Also has a probability of survival wich decreases with time
			\item Also has a decreasing efficiency in allocating to reproduction with age
		\end{enumerate}
	\end{itemize}
\end{frame}

\begin{frame}
	\frametitle{Capturing this in a model}
	\begin{itemize}
		\item Reproduction is the fraction of mass dedicated and scales with mass: $cm^{\rho}$
		\item Probability of survival decreases with time: $l(t)$
		\item Efficiency of fecundity decreases with time: $h(t)$
		\item The product of these is the reproductive output at time $t$, so the lifetime reproductive output is the integral of this quantity from maturity to death
	\end{itemize}
	\begin{block}{Life history model}
		\begin{align*}
			R_0 = \int_{\alpha}^{\infty}l(t)cm^{\rho}h(t) dt
		\end{align*}
	\end{block}
\end{frame}

% \begin{frame}
% 	\frametitle{Fitness - Lifetime Reproductive Output}
% 	\bigskip
% 	\begin{figure}
% 		\includegraphics[width=0.8\textwidth, height=0.5\textwidth]{./Figures/fecundity.pdf}
% 		% remove the 'draft' keyword, when replacing with final figure!
% 		\caption{Lifetime reproductive output from maturity to $t_{\infty}$}
% 	\end{figure}
% \end{frame}
\begin{frame}
	\frametitle{EMT \& LHT}
	\begin{itemize}
		\item To test whether fecundity rate scales hyperallometrically, I allowed $c$ and $\rho$ to vary
		\item See if hyperallometry ($\rho > 1$) emerges naturally
	\end{itemize}
	\begin{block}{Final Model}
		\begin{align*}
			\frac{dm}{dt} &= am^{3/4} - bm \ \ \ \ \ \ \ \ \ \ \ \ \ \ \ \ \ m < m_{\alpha}\\
			\frac{dm}{dt} &= am^{3/4} - bm - cm^{\rho} \ \ \ \ \  \ \ \ m \geq m_{\alpha}\\
			R_0 &= \int_{\alpha}^{\infty}l(t)cm^{\rho}h(t) dt
		\end{align*}
	\end{block}
\end{frame}

\begin{frame}
	\frametitle{``Missing'' Energy}
	\begin{itemize}
		\item Where is the energy shortfall coming from?
		\item Pawar \textit{et al.}'s Intake rate mass scaling laws
	\end{itemize}
	\begin{table}
		\tiny
		\caption{Size-scaling exponents of consumption rate in different foraging dimensions and resource states. 2D v 3D \& Empirical v Predicted (parentheses) are significantly different in all cases (Pawar \textit{et al.,} 2012)}
		\begin{tabularx}{\linewidth}{|X|X|X|}
		\toprule
		& \multicolumn{2}{c}{\textbf{Consumption rate scaling exponent}}                                        \\
		\midrule
		& \multicolumn{1}{c}{\textbf{Scarce resources}} & \multicolumn{1}{c}{\textbf{Abundant resources}} \\
		\midrule
		2D & 0.85 $\pm$ 0.05 (0.78)                      & 0.85 $\pm$ 0.05 (0.78)                        \\
		\midrule
		3D & 1.06 $\pm$ 0.06 (1.16)                      & 1.00 $\pm$ 0.06 (1.16)                       \\              
		\bottomrule
		\end{tabularx}
	\end{table}
\end{frame}
\end{document}