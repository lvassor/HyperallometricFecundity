\documentclass[handout]{beamer}
%\documentclass[presentation]{beamer}

\usecolortheme{Imperial}
 
\usepackage[utf8]{inputenc}
\usepackage[UKenglish]{babel}
\usepackage{booktabs}
\usepackage{caption}
\usepackage{subcaption}
\usepackage{graphicx}
\usepackage{amsmath}
\usepackage{amsfonts}
\usepackage{amssymb}
\usepackage{epstopdf}
\usepackage{pgfplots}
\usepackage{tabularx}
\usepackage{array}
\usepackage{xpatch}

\usepackage[maxcitenames=2, backend=biber, dashed=false, style=imperialharvard]{biblatex}
    \addbibresource{../../Write-Up/CMEE_Thesis.bib}
    \setlength\bibitemsep{0.5\baselineskip}
\renewbibmacro*{textcite}{%
  \ifnameundef{labelname}
    {\iffieldundef{shorthand}
       {\usebibmacro{cite:label}%
        \setunit{%
          \global\booltrue{cbx:parens}%
          \addspace\bibopenparen}%
        \ifnumequal{\value{citecount}}{1}
          {\usebibmacro{prenote}}
          {}%
        \usebibmacro{cite:labelyear+extrayear}}
       {\usebibmacro{cite:shorthand}}}
    {\printnames{labelname}%
     \setunit{%
       \global\booltrue{cbx:parens}%
       \addspace\bibopenparen}%
     \ifnumequal{\value{citecount}}{1}
       {\usebibmacro{prenote}}
       {}%
     \usebibmacro{citeyear}}}
    % \renewcommand*\finalnamedelim{\addspace\&\space}

% Force \textit{et al}. to be italicised

    \xpatchbibmacro{name:andothers}{%
        \bibstring{andothers}%
    }{%
    \bibstring[\emph]{andothers}%
    }
    \xpatchbibmacro{Name}{%
        \bibstring{name}
    }{%
    \bibstring[\textbf]{name}%
    }{}{} % leave these empty arguments here, seem to cause issues when ommited
    \xpatchbibmacro{textcite} % combine same author different year citations
  {\setunit{\addcomma}\usebibmacro{cite:extrayear}}
  {\setunit{\compcitedelim}\usebibmacro{cite:labelyear+extrayear}}
  {}
  {}
% Kill "Accessed on" lines in bibliography
\AtEveryBibitem{
    \clearfield{urlyear}
    \clearfield{urlmonth}
    \clearfield{doi}
    \clearfield{url}
    \clearfield{eprint}
    \clearfield{eprinttype}
}
\newcommand{\putat}[3]{\begin{picture}(0,0)(0,0)\put(#1,#2){#3}\end{picture}}
% complying UK date format, i.e. 1 January 2001
\usepackage{datetime}
\let\dateUKenglish\relax
\newdateformat{dateUKenglish}{\THEDAY~\monthname[\THEMONTH] \THEYEAR}
% \usepackage{eso-pic}
% \newcommand\AtPagemyUpperLeft[1]{\AtPageLowerLeft{%
% \put(\LenToUnit{0.9\paperwidth},\LenToUnit{0.9\paperheight}){#1}}}
% \AddToShipoutPictureFG{
%   \AtPagemyUpperLeft{{\includegraphics[width=.5cm,keepaspectratio]{./Figures/fish.png}}}
% }%
% Imperial College Logo, not to be changed!
\institute{\includegraphics[height=0.7cm]{logo.png}}

% -----------------------------------------------------------------------------




%Information to be included in the title page:
\title{Realistic Intake Rate Scaling Predicts Hyperallometric Fecundity Rate and Later Maturity In Fish}

\subtitle{Theoretical Ecology}

\author{Luke Vassor \\ \textit{M.Sc. CMEE}}

\date{\today}



\begin{document}
 
\frame{\titlepage}


\begin{frame}
	\frametitle{Ontogenetic Growth}
	\begin{itemize}
		\item Organisms acquire resources from the environment to grow new biomass
		\item Costs to address: maintaining cells, reproduction
		\item Ecological Metabolic Theory: Growth rate slows because as we grow larger, we need to maintain more and more cells
		\item Energy intake and maintenance scale with mass differently
	\end{itemize}
	\centering
	\includegraphics[height=1cm]{./Figures/fish.png}\hspace{0.8cm}\includegraphics[height=1.5cm]{./Figures/fish.png}\hspace{0.8cm}\includegraphics[height=2cm]{./Figures/fish_fecund.png}
\end{frame}


\begin{frame}
	\frametitle{Power laws: Why Growth Slows}
		\centering
		\begin{figure}
			\includegraphics[width=\textwidth]{./Figures/presentation_growth_schedules.pdf} \
			% remove the 'draft' keyword, when replacing with final figure!
			\caption{Maintenance (metabolic) rate scales with mass more steeply than intake rate, causing growth to slow or terminate.}
		\end{figure}
\end{frame}
\begin{frame}
	\frametitle{Power laws: Intake Rate Update}
	\putat{230}{32}{\includegraphics[height=1cm]{./Figures/resource.png}}\putat{260}{25}{\includegraphics[width=2.5cm]{./Figures/fish.png}}
	\begin{itemize}
		\item Most models assume that intake rate relates to resting metabolic rate
		\item $MR \propto m^{3/4}$ - increased efficiency at larger sizes
		% \item Therefore the assumption is that larger organisms acquire disproportionately less energy, so intake rate scales with mass as: $am^{3/4}$
		\item Resting MR carries restrictive assumptions
		\item Intake rate more likely related to field metabolic rate, which scales as: $am^{0.85}$ \autocite{Weibel2004, Pawar2012}
		\item Consumption rate scales as steeply as $am^{1.06}$
		\item I used the traditional 0.75 and updated 0.85 scaling to see if this increases the likelihood of fecundity rate hyperallometry
	\end{itemize}
\end{frame}
\begin{frame}
	\frametitle{Power laws: Fecundity Rate}
	\putat{215}{-5}{\includegraphics[height=2cm]{./Figures/fish_fecund.png}}
	\begin{itemize}
		\item New results from \textcite{Barneche2018-reproductive_output}: fish data suggest that fecundity output scales hyperallometrically 
		\item Bigger fish mothers are disproportionately more fecund - not an instantaneous rate
	\end{itemize}
		\begin{columns}[b]
			\column{0.0125\textwidth}
			\column{0.4875\textwidth}
			\vspace{-0.5cm}
				\centering
				\begin{figure}
					\includegraphics[width=0.85\textwidth]{{./Figures/diego_hyperallometry}.pdf} \
					% remove the 'draft' keyword, when replacing with final figure!
					\caption{Hyperallometric fecundity on linear axes}
				\end{figure}
			\column{0.4875\textwidth}
				\centering
				\begin{figure}
					\includegraphics[width=\textwidth]{{./Figures/diego_hyperallometry_log}.pdf} \
					% remove the 'draft' keyword, when replacing with final figure!
					\caption{Hyperallometry means costs scale steeper}
				\end{figure}
			\column{0.0125\textwidth}
		\end{columns}
\end{frame}

\begin{frame}
	\frametitle{Capturing this in a model}
	\begin{columns}[b]
		\column{0.0125\textwidth}
		\column{0.4875\textwidth}
		\begin{itemize}
			\item Intake rate scales as: $am^{0.75}$ or $am^{0.85}$
			\item Maintenance scales as: $bm^{1}$
			% \item $1$ exponent as double the cells = double the maintenance costs
			\item Reproduction scales as: $cm^{\rho}$
		\end{itemize}
		\begin{block}{Biphasic Growth model}
			\begin{align*}
				\frac{dm}{dt} &= am^{0.75} - bm \\ %\ \ \ \ \ \ \ m < m_{\alpha}\\
				\frac{dm}{dt} &= am^{0.75} - bm - cm^{\rho} %\ \ \ \ \  \ \ \ m \geq m_{\alpha}
			\end{align*}
		\end{block}
		\column{0.4875\textwidth}
		\centering
		\begin{figure}
			\includegraphics[width=\textwidth]{../../Results/single_curve_with_onset.pdf}   \
				% remove the 'draft' keyword, when replacing with final figure!
			\caption{Example growth curve with onset of maturity at $\alpha = 200$ days}
		\end{figure}
		\column{0.0125\textwidth}
	\end{columns}
\end{frame}

\begin{frame}
	\frametitle{Capturing this in a model}
	\begin{columns}[b]
		\column{0.0125\textwidth}
		\column{0.4875\textwidth}
		\begin{itemize}
			\item I allowed $c$ and $\rho$ to vary in my simulations
			\item See if hyperallometry ($\rho > 1$) emerges naturally
			\item Remove shrinking growth curves - not biologically realistic
		\end{itemize}
		\begin{block}{Maximising Fitness}
			\begin{align*}
				R_0 &= \int_{\alpha}^{\infty}l(t)cm^{\rho}h(t) dt
			\end{align*}
		\end{block}
		\column{0.4875\textwidth}
		\centering
		\begin{figure}
			\includegraphics[width=\textwidth]{../../Results/single_curve_with_onset.pdf}   \
				% remove the 'draft' keyword, when replacing with final figure!
			\caption{Example growth curve with onset of maturity at $\alpha = 200$ days}
		\end{figure}
		\column{0.0125\textwidth}
	\end{columns}
\end{frame}
		% \item Probability of survival decreases with time: $l(t)$
		% \item Efficiency of fecundity decreases with time: $h(t)$

\begin{frame}
	\frametitle{Results - Varying Intake Rate Scaling}
	\begin{columns}[b]
		\column{0.0125\textwidth}
		\column{0.4875\textwidth}
			\centering
			\begin{figure}
				\includegraphics[width=\textwidth]{{../../Results/opt_hm_Alph=200.00_a=2.15_x=0.75_k=0.01_tex}.pdf} \
				% remove the 'draft' keyword, when replacing with final figure!
				\caption{Optimum fecundity rate parameters $c,\rho$ when \\ intake rate scaling = 0.75}
			\end{figure}
		\column{0.4875\textwidth}
			\centering
			\begin{figure}
				\includegraphics[width=\textwidth]{{../../Results/opt_hm_Alph=200.00_a=2.15_x=0.85_k=0.01_tex}.pdf} \
				% remove the 'draft' keyword, when replacing with final figure!
				\caption{Optimum fecundity rate parameters $c,\rho$ when \\ intake rate scaling = 0.85}
			\end{figure}
		\column{0.0125\textwidth}
	\end{columns}
\end{frame}

\begin{frame}
	\frametitle{Results - Varying Age-At-Maturity, $\alpha$}
	\begin{columns}[b]
		\column{0.0125\textwidth}
		\column{0.4875\textwidth}
			\centering
			\begin{figure}
				\includegraphics[width=\textwidth]{{../../Results/alpha_sensitivity_x=0.75}.pdf}   \
				% remove the 'draft' keyword, when replacing with final figure!
				\caption{Intake rate scaling of 0.75}
			\end{figure}
		\column{0.4875\textwidth}
			\centering
			\begin{figure}
				\includegraphics[width=\textwidth]{{../../Results/alpha_sensitivity_x=0.85}.pdf}  \
				% remove the 'draft' keyword, when replacing with final figure!
				\caption{Intake rate scaling of 0.85}
			\end{figure}
		\column{0.0125\textwidth}
	\end{columns}
\end{frame}

\begin{frame}
	\frametitle{Realistic maturity ages}
	\begin{itemize}
		\item Empirically, fish have been observed to mature after years of growth (in the order of thousands of days)
		\begin{itemize}
			\item Cod: 2-4 years \autocite{OBrien1993, Rochet2001, Knickle2013}
			\item Chinook Salmon: 2-5 years \autocite{groot1991pacific}
			\item Yellowtail flounder: 2-5 years \autocite{OBrien1993}
		\end{itemize}
		\item Maturity ages this late necessitate steeper intake rate scaling
	\end{itemize}
\end{frame}

\begin{frame}
	\frametitle{Future direction}
	\begin{itemize}
		\item ``Reverse-engineer'' consumption rate scaling from \textcite{Barneche2018-reproductive_output} data 
		\item Collect data on continuous rate of fecundity allocation - very difficult to do
	\end{itemize}
\end{frame}
\begin{frame}
	\frametitle{Acknowledgements}
	\begin{itemize}
		\item Dr Samraat Pawar (Supervisor, ICL) 
		\item Tom Clegg (PhD Student, ICL)
		\item Prof Van Savage (UCLA)
		\item Dr Diego Barneche (University of Exeter)
		\item Prof Dustin Marshall (Monash University)
	\end{itemize}
\end{frame}
\end{document}
