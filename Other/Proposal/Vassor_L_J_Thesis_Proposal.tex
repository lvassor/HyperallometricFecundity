\documentclass[11pt]{article}
\usepackage[margin=2cm]{geometry}
\usepackage{graphicx}
\usepackage{amsmath}
\usepackage{lineno}
\usepackage{setspace}
\usepackage{apacite}
\usepackage{gensymb}
\usepackage{amssymb}
\usepackage{wrapfig}
\usepackage{multicol}
\usepackage{pgfgantt}
\usepackage{geometry} % to change margins
\usepackage{pdflscape} % provides the landscape environment
\usepackage{ragged2e} % provides \RaggedLeft
\usepackage{pdfpages}
\usepackage{xcolor}
\usepackage{tabu}


\doublespacing

\title{\LARGE{THESIS PROPOSAL} \\ \textbf{Accounting for Resource Supply in Ontogenetic Growth Models}}
\author{Luke Vassor \\ CID: 01607235}
\date{}
\begin{document}
    \maketitle    
    \begin{center}
        \begin{multicols}{2}
        \begin{singlespacing}
            Primary Supervisor: Dr Samraat Pawar \\
            \textbf{Department of Life Sciences \\
            Imperial College London \\
            London SW7 2AZ \\
            United Kingdom \\
            s.pawar@imperial.ac.uk \\}
            Possible Secondary Supervisor: Dr Diego Barneche \\
            \textbf{College of Life and Environmental Sciences \\
            University of Exeter \\
            Penryn TR10 9FE \\ 
            United Kingdom \\
            d.barneche@exeter.ac.uk \\}
        \end{singlespacing}
        \end{multicols}
        \vspace{3mm}
        \includegraphics[width=3in]{./Images/Imperial_College_London_crest.pdf} \\
        \vspace{3mm}
        
        A thesis proposal submitted in partial fulfilment of the requirements for the degree of Master of Science at Imperial College London \\
        Submitted for the M.Sc. Computational Methods in Ecology \& Evolution
    \end{center}
    \newpage
    \textbf{Keywords: Ontogenetic, Growth, Reproduction, Geometric Biology, Mechanistic}
    \section{Background, Project Idea \& Proposed Questions}
    \linenumbers
        Mechanistic insight into the growth of individuals is integral to understanding the growth responses of populations, communities and ecosystems due to the scalable nature of such mechanisms from individuals up to higher levels of organisation. Ontogenetic growth models consider the allocation of energy (that is, the abilitiy to do work) acquired through metabolism, to producing new biomass at different stages of life. Fundamentally, organisms are considered to portion their metabolic energy across two processes in these models: the growth of new somatic cells, and maintenance of existing cells. Several existing growth models explain this through ``bottom-up" drivers including the von Bertalanffy growth function (VGBF), dynamic energy budget (DEB) theory and the Ontogentic Growth Model (OGM) \shortcite{vonBert1938, vonBert1957, kooijman_2000, G.B.2001, Hou2002,Marshall2019a}. The typical growth curve exhibited by individuals is sigmoidal over ontogeny such that they undergo rapid growth at early (st)ages which then slows approaching an asymptotic adult size \shortcite{Barneche2018a}. The mechanistic interpretation under this paradigm is that as we age, maintenance of existing cells dominates new biomass production, thus we cease to grow. This is a broad life-history trait which comes as an apparent consequence of evolution optimising trade-offs between competing traits and limited resources, as explained by life-history theory i.e. given the niche of a specific organism, it is energetically most efficient to undergo different life history events at specific times in the organism's lifetime \shortcite{StearnsEvoLifeHistories,Marshall2019a,Pettersen2019}. 
        
        While size-dependence of growth is well understood, however, well-established bottom-up models fail to sufficiently consider reproduction \shortcite{Barneche2018a,Marshall2019a}. Reproduction is energetically costly, and as noted by \shortcite{Barneche2018a}, is crucially assumed in presiding mechanistic models to scale \textit{isometrically} with size. That is, individuals allocate a constant proportion of energy to reproduction, realised through egg number, size and energy content, as they grow (beyond maturity). However recent results from a large-scale meta-analysis indicate that, in fact, reproductive energy allocation scales hyperallometrically with size \shortcite{Barneche2018,Marshall2019a}. That is, larger mothers are more fecund than smaller mothers. This potential reset to theoretical assumptions has profound implications for conservation and stock management since these growth models underlie much of current policy.

        Further, existing models fail to explicitly include the effects of resource supply. Most implicitly assume that resource supply is met throughout growth or ontogeny and assume a simple relationship between consumption rate and metabolic rate, such that consumption rate therefore also scales with size to an exponent $\beta = 0.75$ \shortcite{pawar2012}. By extension this also implies that no metabolic energy is expended on failed foraging attempts because an assumption of modelling metabolic rate is idealised conditions of no foraging, reproduction or growth \shortcite{peters1983, jungers1986,alexander2003principles}. However, it has been documented that there is substantial variation in resource encounter rate data as consumers tend to encounter resources more frequently in 3-D space, with an exponent of $\beta \approx 1.06$, a superlinear relationship with size \shortcite{pawar2012}. This finding has strong consequences for organisms foraging in 3-D space, such as pelagic fish, and sheds light on the apparent gap in the growth model literature, which has thus far been unable to incorporate the supply of resources to an organism as it grows as an explicit term, given the assumed matched scaling with metabolic rate. In light of the aforementioned recent literature suggesting a superlinear scaling of reproductive energy allocation with size, this highlights the novelty of this project's aims, to link two schools of thought: metabolic theory and optimal foraging theory. The similarity in superlinear scaling between reproductive energy output and 3-D resource encounter rate with size is somewhat suggestive and begs new consideration with respect to dominant growth models. The aim of this project will be to build upon the Charnov (2006) Ontogenetic Growth Model by incorporating an explicit resource supply term which considers scaling of encounter rates, using theoretical developments from \shortcite{pawar2012}. If we are successful in doing this, reproductive allocation could then be built into the model as a parameter or used as validation as the model itself, the potential finding being a linear relationship between reproductive energy allocation and resource supply. That is, once resource supply is incorporated it can then be varied to observe the effects on life history. \color{white}{\shortcite{Charnov2006}} \color{black} %hacky fix to in-text citation with not brackets


    \section{Anticipated Output \& Outcomes}
        Assuming a successful project, based on informed speculation, we hope to discover a linear relationship between resource supply and reproductive allocation. However, this is predicated on the assumption that the extra energy gained from hyperallometrically scaled resource encounter rate as an individual gets bigger is then devoted to extra reproductive output. In reality, we may be underestimating the amount of energy that is devoted to foraging and maintenance of exisiting tissue. This would be equally an informative finding of the project if it is the case.
    \section{Project feasibility \& Gantt Chart}
        Discussion within pawarlab confirms that developing the model to such an extent is feasible in the time given. Potential for project aims to grow as time passes, especially given visits of field experts to Silwood Park in Spring \& Summer months. See below for estimated projects phase durations.
        \vspace{3mm}
        \includegraphics[width=7.5in]{./gantt_chart.pdf}
    \section{Itemized budget}
        Given the theoretical nature of the project, little extra resources should be required:
        \begin{table}[ht]
            \caption{Anticipated costs}
            \begin{tabular}{|l|l|r|}
            \hline
            \textbf{Item}              & \textbf{Description}                                      & \textbf{Cost \pounds} \\ \hline
            External hard-drive        & To store/backup large ontogenetic growth dataset          & 89.98         \\ \hline
            Visit secondary supervisor & ICL to University of Exeter Penryn, Super Off-Peak Return & 78.60         \\ \hline
            \end{tabular}
        \end{table}
    \section{Cited References}

\bibliographystyle{apacite}
\bibliography{./proposal.bib}

\newpage

\large{\textbf{``I have seen and approved the proposal and the budget''}} \\ \\
\normalsize{
Approved: \includegraphics[width=2in]{./samsig.pdf} \\

\hspace*{0mm}\phantom{Approved: }Date: 29/4/2019

\hspace*{0mm}\phantom{Approved: }Dr Samraat Pawar

\hspace*{0mm}\phantom{Approved: }Primary Supervisor
}


\end{document}
